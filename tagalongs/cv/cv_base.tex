% LaTeX file for resume
% This file uses the resume document class (res.cls)
\let\nofiles\relax
\documentclass{res}
%\usepackage{helvetica} % uses helvetica postscript font (download helvetica.sty)
%\usepackage{newcent}   % uses new century schoolbook postscript font
\newsectionwidth{0pt}  % So the text is not indented under section headings
\usepackage{fancyhdr}  % use this package to get a 2 line header
\usepackage{graphicx}
\usepackage{hyperref}
\renewcommand{\headrulewidth}{0pt} % suppress line drawn by default by fancyhdr
\setlength{\headheight}{24pt} % allow room for 2-line header
\setlength{\headsep}{24pt}  % space between header and text
\setlength{\headheight}{24pt} % allow room for 2-line header
\pagestyle{fancy}     % set pagestyle for document
\rhead{ {\it P. Talbot}\\{\it p. \thepage} } % put text in header (right side)
\cfoot{}                                     % the foot is empty
\topmargin=-0.5in % start text higher on the page

\begin{document}
\thispagestyle{empty} % this page has no header
\name{Paul W. Talbot\\[12pt]}% the \\[12pt] adds a blank line after name

%\address{{\bf School Address} \\ Farris Engineering Center, Suite 209 \\
%  University of New Mexico \\   Albuquerque, NM 87106}

%\address{{\bf Home Address} \\ 3643 B Deloy Dr. \\Idaho Falls, ID 83401 \\ (509) 713-2842}
\address{3201 Florian Ave. \\Idaho Falls, ID 83401 \\ (509) 713-2842}




\begin{resume}

%\section{\centerline{OBJECTIVE}}
%\vspace{8pt} % provide vertical space between section title and contents
%A summer position with the MOOSE team at Idaho National Laboratory working on numerical methods.

\vspace{0.2in}
\section{\centerline{PROFESSIONAL EXPERIENCE}}
\vspace{8pt}

{\sl Idaho National Laboratory, Idaho Falls, ID} \\[2pt]
Integrated Energy Systems (IES) \hfill    2017 - Present
\begin{itemize} \itemsep -2pt
  \item \emph{PI and lead developer, HERON (github.com/idaholab/HERON)} \\
  Led a small team of developers in implementing HERON, a Python-based object-oriented plugin of RAVEN
  for stochastic techno-economic optimization of grid-energy systems. Planned, designed, and deployed
  software under continuous-deployment version control and software quality assurance. Provided training
  and deployed plugin for analyses of novel integrated energy system technologies in several markets,
   with partner industries, other national laboratories, and universities.
  \item \emph{2021 LDRD awarded: Signal processing for cybersecurity} \\
  Managed a 3-year \$1.5M project including lab and university researchers to explore using high/low
  order correlation for false data injection detection in physical process signals, including
  toolset software development and demonstration on experimental systems.
  \item \emph{Stochastic gradient descent optimization R\&D} \\
  Extended RAVEN software to perform system optimization using stochastic gradient descent algorithms,
  including forward and central differencing as well as Simultaneous Perturbation Stochastic Approximation
  (SPSA) gradient descent. Extended existing algorithms to solve stochastic system optimization problems.
  \item \emph{Point of contact for multiple awarded university-led (NEUP) proposals}
\end{itemize} \vspace{-6pt}

Uncertainty Quantification \hfill    2014 - 2017
\begin{itemize} \itemsep -2pt
  \item \emph{Sparse grid collocation for generalized polynomial chaos} \\
  As part of doctorate work, extended RAVEN sampling and reduced-order model methods to include
  sparse grid collocation sampling for multidimensional polynomial surrogate models. Demonstrated
  strengths and weaknesses of algorithm on several analytic as well as real-world problems.
  \item \emph{High-dimension model reduction} \\
  Implemented HDMR to accelerate polynomial fitting using sparse grid collocation, including both
  static and adaptive sampling strategies.
  \item \emph{Agile development, maintenance, refactoring, quality assurance}
  \item \emph{Continuous integration deployment}
  \item \emph{Senior developer, RAVEN (github.com/idaholab/raven)} \\
  Trained new users and developers in RAVEN as well as orchestrated signifcant new development and reworks
  of several systems in RAVEN as needs adjusted, managing performance in both memory and speed.
  Responded frequently to user queries, offering guidance or implementing code changes. Reviewed
  proposed code changes for software quality assurance, iterating with developers to improve code
  contributions.
  \item \emph{Python, LaTeX, C++, Conda, Bash}
\end{itemize} \vspace{-6pt}

{\sl Internships} %\\[2pt]
\begin{itemize} \itemsep -2pt
  \item 2012 INL, MOOSE Deployment packages and regression testing
  \item 2011 LANL, Discrete maximum principle for iMC equations
  \item 2010 INL, MARMOT Frenkel pair distribution R\&D
  \item 2009 AREVA, BLEU effective enrichment research
  \item 2008 AREVA, CASMO4 and MICROBURN-B2, benchmarking
\end{itemize} \vspace{-6pt}

\vspace{0.2in}
\section{\centerline{EDUCATION}}
\vspace{8pt}
{\sl Doctor of Philosophy}, Nuclear Engineering \\
University of New Mexico, Albuquerque, New Mexico, GPA 4.08 \hfill December 2016 \\
Thesis - Advanced Stochastic Collocation Methods for Polynomial Chaos in RAVEN
\begin{itemize} \itemsep -2pt
\item Researched, implemented multidimensional sparse grid sampling techniques
\item Distribution-specific polynomial fitting for uncertain variables
\item Vast statistical convergence improvement demonstrated for continuous responses
\item Development in Object-Oriented Python (RAVEN framework, github.com/idaholab/raven)
\item Software quality assurance, version control (Git)
\item Intership, Idaho National Laboratory
\end{itemize}

{\sl Master of Science}, Nuclear Engineering \\
Oregon State University, Corvallis, Oregon, GPA 3.75 \hfill March 2013 \\
Thesis - Extending the Discrete Maximum Principle for the IMC Equations
\begin{itemize} \itemsep -2pt
\item Research, implemented theoretical maximum for implicit Monte Carlo
\item Improved solve strategy for nonlinear photon transport
\item Intership, Los Alamos National Laboratory
\end{itemize}

{\sl Bachelor of Science}, Physics \\
BYU-Idaho, Rexburg, Idaho, GPA 3.84     \hfill    April 2010
\begin{itemize} \itemsep -2pt
\item Investigate down-blended vs fresh UO2 performance in BWR assembly
\item CASMO4, MICROBURN-B2, ALLADIN benchmarking
\item Power shapes, pin peaking, design limits
\item Intership, AREVA NP in Richland, WA
\end{itemize}

\vspace{0.2in}
\section{\centerline{ COMPUTING SKILLS }}
\vspace{8pt}
\center{
Extensive use of Python (Conda, Pandas, Xarray, SKLearn, Statsmodels, OOP/Functional), Git, Bash\\
Experience with C++, MatLab, Visual Basic}

\vspace{0.2in}
\section{\centerline{SAMPLE PUBLICATIONS}}
\vspace{15pt}
\begin{itemize}
  \item K. Frick, D. Wendt, P. W. Talbot, et al, ``Technoeconomic Assessment of Hydrogen Cogeneration via High Temperature Steam Electrolysis with a Light-Water Reactor,''
    Applied Energy 2022; vol 306 part B, pp. 118044. https://doi.org/10.1016/j.apenergy.2021.118044
  \item Y. Li, A. Sundaram, HS. Abdel-Khalik, P. W. Talbot, ``Real-Time Monitoring for Detection of Adversarial Subtle Process Variations,''
    Nuclear Science and Engineering 2022; pp. 1024. https://doi.org/10.1080/00295639.2021.1997041
  \item D. McDowell, P. W. Talbot, et al, ``A Technical and Economic Assessment of LWR Flexible Operation for Generation and Demand Balancing to Optimize Plant Revenue'',
    INL report INL/EXT-21-65443, 2021
  \item R. R. Flanagan, P. W. Talbot, et al, ``Isolating cloud induced noise to improve generation of synthetic surface solar irradiances,''
    Advances in Applied Energy 2021, vol 3, pp. 100045. https://doi.org/10.1016/j.adapen.2021.100045
  \item P. W. Talbot, D. McDowell, et al, ``Evaluation of Hybrid FPOG Applications in Regulated and Deregulated Markets using HERON'',
    INL report INL/EXT-20-60968, 2020
  \item P. W. Talbot, C. Rabiti, et al, ``Correlated Synthetic Time Series Generation using Fourier and ARMA signal processing,''
    Int. J. Energy Res. 2020; 1-12. https://doi.org/10.1002/er.5115
  \item A. Epiney, C. Rabiti, P. Talbot, et al, "Economic analysis of a nuclear hybrid energy system in a stochastic environment including wind turbines in an electricity grid",
    Applied Energy 2020; 260, 114227
  \item P. W. Talbot, et al, "Analysis of Differential Financial Impacts on LWR Load-Following Operations", INL
	  report INL/EXT-19-55614, 2019
  \item K. Frick, P. Talbot, et al, "Evaluation of Hydrogen Production Feasibility for a Light Water Reactor
	  in the Midwest", INL report INL/EXT-19-55395, 2019
  \item A. Epiney, C. Rabiti, P. Talbot, et al, "Case Study: Nuclear-Renewable-Water Integration in Arizona", INL report INL/EXT-18-51369, 2018
  \item C. Rabiti, A. Epiney, P. W. Talbot, et al, "Status Report on Modeling and Simulation Capabilities for Nuclear-Renewable Hybrid Energy Systems", INL Report INL/EXT-17-43441, 2017
  \item P. W. Talbot, ``Advanced Stochastic Collocation Methods for Polynomial Chaos in RAVEN,'', Ph. D.
    Dissertation, Department of Nuclear Engineering, University of New Mexico, December 2016
  \item P. W. Talbot, C. Wang, et al, ``Multistep Input Reduction for High
    Dimensional Uncertainty Quantification in RAVEN Code,'' ANS PHYSOR 2016
  \item P. W. Talbot, K. Gamble, et al, ``Time-Dependent Sensitivity Analysis of OECD Benchmark using BISON
    and RAVEN,'' 2016 ANS winter conference transactions
  \item P. W. Talbot, A. K. Prinja, C. Rabiti, ``Adaptive Sparse-Grid Stochastic Collocation Uncertainty
    Quantification Convergence for Multigroup Diffusion,'' 2016 ANS annual conference transactions
  \item C. Wang, P. W. Talbot, et al, ``An efficient Sampling-Based Method for Sensitivity and
    Uncertainty Analysis through RAVEN,'' 2016 ANS annual conference transactions
  \item P. W. Talbot, A. K. Prinja, C. Rabiti, ``High Density Model Reduction Uncertainty Quantification
    for Multigroup Diffusion Neutronics,'' 2015 ANS M\&C topical conference transactions
  \item P. W. Talbot, A. K. Prinja, ``Sparse-Grid Stochastic Collocation Uncertainty Quantification Convergence
    for Multigroup Diffusion,'' 2014 ANS winter conference transactions
  \item P. W. Talbot, ``Extending the Discrete Maximum Principle for the IMC equations,'' Oregon State University masters thesis, September 2012
  \item P. W. Talbot, A. B. Wollaber, T. Palmer, ``Implementing a Discrete Maximum Principle for the IMC Equations," 2012 ANS general conference transactions, M \& C division
\end{itemize}
\vspace{10pt}
\begin{itemize}
  \item ORCID: \href{https://orcid.org/0000-0002-9672-9044}{0000-0002-9672-9044}
  \item OSTI: \url{https://www.osti.gov/search/orcid:0000000296729044}
  \item Publons (reviews): \url{https://publons.com/researcher/3839497/paul-talbot/}
\end{itemize}

\vspace{0.2in}
\section{\centerline{MEMBERSHIPS}}
\vspace{-5pt} % reduce space between section title and contents
\begin{itemize}
  \item Reviewer:
  \begin{itemize}
    \item \emph{Applied Energy}
    \item \emph{Energies}
    \item \emph{Mathematics}
    \item \emph{Nuclear Science and Engineering}
    \item \emph{Nuclear Technology}
  \end{itemize}
  \item Conference reviewer: ANS, ANS M\&C\\
  \item Technical Program Committee, ANS M\&C 2019\\
  \item Member, American Nuclear Society
 \end{itemize}

 \begin{center}
	 \emph{References available on request.}
 \end{center}

\end{resume}
\end{document}



%%%%%% EXCESS
\vspace{0.2in}
\section{\centerline{REFERENCES}}
% \vspace{-5pt} % reduce space between section title and contents
\begin{itemize}
  \item Cristian Rabiti, Relationship Manager, Idaho National Laboratory, cristian.rabiti@inl.gov
  \item Andrea Alfonsi, Technical Lead, Idaho National Laboratory, andrea.alfonsi@inl.gov
  \item Anil Prinja, Department Head, University of New Mexico, prinja@unm.edu
 \end{itemize}
 \vspace{0.2in}
%\section{\centerline{INTERESTS}}
%\vspace{-5pt}
%\begin{center}
%scripting and coding, Jazz, family, rock climbing
%\end{center}















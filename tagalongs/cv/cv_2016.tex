% LaTeX file for resume
% This file uses the resume document class (res.cls)
\let\nofiles\relax
\documentclass{res}
%\usepackage{helvetica} % uses helvetica postscript font (download helvetica.sty)
%\usepackage{newcent}   % uses new century schoolbook postscript font
\newsectionwidth{0pt}  % So the text is not indented under section headings
\usepackage{fancyhdr}  % use this package to get a 2 line header
\renewcommand{\headrulewidth}{0pt} % suppress line drawn by default by fancyhdr
\setlength{\headheight}{24pt} % allow room for 2-line header
\setlength{\headsep}{24pt}  % space between header and text
\setlength{\headheight}{24pt} % allow room for 2-line header
\pagestyle{fancy}     % set pagestyle for document
\rhead{ {\it P. Talbot}\\{\it p. \thepage} } % put text in header (right side)
\cfoot{}                                     % the foot is empty
\topmargin=-0.5in % start text higher on the page

\begin{document}
\thispagestyle{empty} % this page has no header
\name{Paul W. Talbot\\[12pt]}% the \\[12pt] adds a blank line after name

%\address{{\bf School Address} \\ Farris Engineering Center, Suite 209 \\
%  University of New Mexico \\   Albuquerque, NM 87106}

%\address{{\bf Home Address} \\ 3643 B Deloy Dr. \\Idaho Falls, ID 83401 \\ (509) 713-2842}
\address{3643 B Deloy Dr. \\Idaho Falls, ID 83401 \\ (509) 713-2842}



\begin{resume}

%\section{\centerline{OBJECTIVE}}
%\vspace{8pt} % provide vertical space between section title and contents
%A summer position with the MOOSE team at Idaho National Laboratory working on numerical methods.

\vspace{0.2in}
\section{\centerline{EDUCATION}}
\vspace{8pt}
{\sl Doctor of Philosophy}, Nuclear Engineering \\
University of New Mexico, Albuquerque, New Mexico, GPA 4.08 \hfill December 2016

{\sl Master of Science}, Nuclear Engineering \\
Oregon State University, Corvallis, Oregon, GPA 3.75 \hfill March 2013 %\\
%Thesis - Extending the Discrete Maximum Principle for the IMC Equations

{\sl Bachelor of Science}, Physics \\ % \sl will be bold italic in
					 % New Century Schoolbook (or
					 % any postscript font) and
					 % just slanted in Computer
					 % Modern (default) font
BYU-Idaho, Rexburg, Idaho, GPA 3.84     \hfill    April 2010

\vspace{0.2in}
\section{\centerline{PROFESSIONAL EXPERIENCE}}
\vspace{8pt}

{\sl Idaho National Laboratory, Idaho Falls, ID} \\[2pt]
RAVEN project \hfill    Fall 2014 - Present

 \begin{itemize} \itemsep -2pt
  \item High-dimension model reduction method implementation
  \item Sparse grid collocation method implementation
  \item BISON uncertainty quantification
    \begin{itemize}
      \item Short PWR rods, failure probabilities
      \item Full PWR rods, sensitivity analysis
      \item Coupling with neutronics (MAMMOTH)
    \end{itemize}
  \item Python, C++
\end{itemize} \vspace{-6pt}


%{\sl Idaho National Laboratory, Idaho Falls, ID} \\[2pt]
MOOSE projects \hfill    Summers 2010, 2012

 \begin{itemize} \itemsep -2pt
  \item Worked within multiphysics object-oriented software environment (MOOSE)
  \item Used method of manufactured solutions to test functionality
  \item Optimized polynomial fits for interstitials and voids in MARMOT
  \item C++, Python
\end{itemize} \vspace{-6pt}


{\sl Los Alamos National Laboratory, Los Alamos, NM} \hfill        Summer 2011 \\
     CCS-2  \hfill

   \begin{itemize} \itemsep -2pt % reduce space between items
   \item Extrapolated existing pseudo-analytic single-dimensional discrete maximum principle for the implicit Monte Carlo equations governing radiative heat transfer to include multiple dimensions, non-equilibrium conditions, and mutligroup energies.
  \item Implemented predictive capacity into use codes at LANL to predict boundedness in choices of spatial and time discretization.
 \end{itemize}


{\sl AREVA, NP} \\[2pt]
BWR Neutronics \hfill    Summers 2008, 2009
 \begin{itemize}
 \item  Assisted in benchmarking software version update
  \item Used simulation codes CASMO4 and MICROBURN-B2
  \item  Researched effect of BLEU fuel in Browns Ferry Unit 2 rector
  %\item Used simulation codes ALADDIN, CASMO3G, CASMO4
 \end{itemize}

\vspace{0.2in}
\section{\centerline{ COMPUTING SKILLS }}
\vspace{8pt}
\center{
Experienced with Python, C++, Git, Bash, MatLab, Visual Basic\\
Some experience in Java, Javascript, Fortran, C}

\vspace{0.2in}
\section{\centerline{PUBLICATIONS}}
\vspace{15pt}
\begin{itemize}
  \item P. W. Talbot, ``Advanced Stochastic Collocation Methods for Polynomial Chaos in RAVEN,'', Ph. D.
    Dissertation, Department of Nuclear Engineering, University of New Mexico, December 2016
  \item P. W. Talbot, C. Wang, et al, ``Multistep Input Reduction for High
    Dimensional Uncertainty Quantification in RAVEN Code,'' ANS PHYSOR 2016
  \item P. W. Talbot, K. Gamble, et al, ``Time-Dependent Sensitivity Analysis of OECD Benchmark using BISON
    and RAVEN,'' 2016 ANS winter conference transactions
  \item P. W. Talbot, A. K. Prinja, C. Rabiti, ``Adaptive Sparse-Grid Stochastic Collocation Uncertainty
    Quantification Convergence for Multigroup Diffusion,'' 2016 ANS annual conference transactions
  \item C. Wang, P. W. Talbot, et al, ``An efficient Sampling-Based Method for Sensitivity and
    Uncertainty Analysis through RAVEN,'' 2016 ANS annual conference transactions
  \item P. W. Talbot, A. K. Prinja, C. Rabiti, ``High Density Model Reduction Uncertainty Quantification
    for Multigroup Diffusion Neutronics,'' 2015 ANS M\&C topical conference transactions
  \item P. W. Talbot, A. K. Prinja, ``Sparse-Grid Stochastic Collocation Uncertainty Quantification Convergence
    for Multigroup Diffusion,'' 2014 ANS winter conference transactions
  \item P. W. Talbot, ``Extending the Discrete Maximum Principle for the IMC equations,'' Oregon State University masters thesis, September 2012
   \item P. W. Talbot, A. B. Wollaber, T. Palmer, ``Implementing a Discrete Maximum Principle for the IMC Equations," 2012 ANS general conference transactions, M \& C division
   \item M. R. Tonks, D. Gaston, P. C. Millett, D. Andrs, P. W. Talbot, ``An object-oriented finite element framework for multiphysics phase field simulations," J. Computational Materials Science, Vol. 50 issue 3, January 2011
   \end{itemize}


\vspace{0.2in}
\section{\centerline{MEMBERSHIPS}}
\vspace{-5pt} % reduce space between section title and contents
\begin{center}
      American Nuclear Society - Alpha Nu Sigma \\
      Society of Physics Students \\
 \end{center}
 \vspace{0.2in}

\vspace{0.2in}
%\section{\centerline{INTERESTS}}
%\vspace{-5pt}
%\begin{center}
%scripting and coding, Jazz, family, rock climbing
%\end{center}

\end{resume}
\end{document}














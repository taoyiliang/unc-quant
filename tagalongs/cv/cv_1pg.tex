% LaTeX file for resume
% This file uses the resume document class (res.cls)
\let\nofiles\relax
\documentclass{res}
%\usepackage{helvetica} % uses helvetica postscript font (download helvetica.sty)
%\usepackage{newcent}   % uses new century schoolbook postscript font
\newsectionwidth{0pt}  % So the text is not indented under section headings
\usepackage{fancyhdr}  % use this package to get a 2 line header
\usepackage{graphicx}
\usepackage{hyperref}
\renewcommand{\headrulewidth}{0pt} % suppress line drawn by default by fancyhdr
\setlength{\headheight}{24pt} % allow room for 2-line header
\setlength{\headsep}{24pt}  % space between header and text
\setlength{\headheight}{24pt} % allow room for 2-line header
\pagestyle{fancy}     % set pagestyle for document
\rhead{ {\it P. Talbot}\\{\it p. \thepage} } % put text in header (right side)
\cfoot{}                                     % the foot is empty
\topmargin=-0.5in % start text higher on the page

\begin{document}
\thispagestyle{empty} % this page has no header
\name{Paul W. Talbot\\[12pt]}% the \\[12pt] adds a blank line after name

%\address{{\bf School Address} \\ Farris Engineering Center, Suite 209 \\
%  University of New Mexico \\   Albuquerque, NM 87106}

%\address{{\bf Home Address} \\ 3643 B Deloy Dr. \\Idaho Falls, ID 83401 \\ (509) 713-2842}
\address{3201 Florian Ave. \\Idaho Falls, ID 83401 \\ (509) 713-2842}




\begin{resume}

%\section{\centerline{OBJECTIVE}}
%\vspace{8pt} % provide vertical space between section title and contents
%A summer position with the MOOSE team at Idaho National Laboratory working on numerical methods.
\vspace{0.2in}
\section{\centerline{EDUCATION}}
\vspace{8pt}
{\sl Doctor of Philosophy}, Nuclear Engineering \\
University of New Mexico, Albuquerque, New Mexico, GPA 4.08 \hfill December 2016 \\
Thesis - Advanced Stochastic Collocation Methods for Polynomial Chaos in RAVEN
% \begin{itemize} \itemsep -2pt
% \item Researched, implemented multidimensional sparse grid sampling techniques
% \item Distribution-specific polynomial fitting for uncertain variables
% \item Vast statistical convergence improvement demonstrated for continuous responses
% \item Development in Object-Oriented Python (RAVEN framework, github.com/idaholab/raven)
% \item Software quality assurance, version control (Git)
% \item Intership, Idaho National Laboratory
% \end{itemize}

{\sl Master of Science}, Nuclear Engineering \\
Oregon State University, Corvallis, Oregon, GPA 3.75 \hfill March 2013 \\
Thesis - Extending the Discrete Maximum Principle for the IMC Equations
% \begin{itemize} \itemsep -2pt
% \item Research, implemented theoretical maximum for implicit Monte Carlo
% \item Improved solve strategy for nonlinear photon transport
% \item Intership, Los Alamos National Laboratory
% \end{itemize}

{\sl Bachelor of Science}, Physics \\
BYU-Idaho, Rexburg, Idaho, GPA 3.84     \hfill    April 2010
% \begin{itemize} \itemsep -2pt
% \item Investigate down-blended vs fresh UO2 performance in BWR assembly
% \item CASMO4, MICROBURN-B2, ALLADIN benchmarking
% \item Power shapes, pin peaking, design limits
% \item Intership, AREVA NP in Richland, WA
% \end{itemize}


%\vspace{0.2in}
\section{\centerline{PROFESSIONAL EXPERIENCE}}
\vspace{8pt}

{\sl Idaho National Laboratory, Idaho Falls, ID} \\[2pt]
Integrated Energy Systems (IES) \hfill    2017 - Present
\begin{itemize} \itemsep -2pt
  \item \emph{Software architect, FORCE (github.com/idaholab/FORCE)}
  \item \emph{2021 LDRD awarded: Signal processing for cybersecurity}
  \item \emph{Stochastic gradient descent optimization R\&D}
\end{itemize} \vspace{-6pt}

Uncertainty Quantification \hfill    2014 - 2017
\begin{itemize} \itemsep -2pt
  \item \emph{Sparse grid collocation for generalized polynomial chaos}
  \item \emph{High-dimension model reduction}
  \item \emph{Continuous integration deployment}
  \item \emph{Senior developer, RAVEN (github.com/idaholab/raven)}
\end{itemize} \vspace{-6pt}

{\sl Internships} %\\[2pt]
\begin{itemize} \itemsep -2pt
  \item 2012 INL, MOOSE Deployment packages and regression testing
  \item 2011 LANL, Discrete maximum principle for iMC equations
  \item 2010 INL, MARMOT Frenkel pair distribution R\&D
  \item 2009 AREVA, BLEU effective enrichment research
  \item 2008 AREVA, CASMO4 and MICROBURN-B2, benchmarking
\end{itemize} \vspace{-6pt}

\vspace{10pt}
{\sl Publications} %\\[2pt]
\begin{itemize}
  \item ORCID: \href{https://orcid.org/0000-0002-9672-9044}{0000-0002-9672-9044}
  \item OSTI: \url{https://www.osti.gov/search/orcid:0000000296729044}
  \item Publons (reviews): \url{https://publons.com/researcher/3839497/paul-talbot/}
\end{itemize}

\vspace{0.2in}
% \section{\centerline{MEMBERSHIPS}}
% \vspace{-5pt} % reduce space between section title and contents
% \begin{itemize}
%   \item Reviewer:
%   \begin{itemize}
%     \item \emph{Applied Energy}
%     \item \emph{Energies}
%     \item \emph{Mathematics}
%     \item \emph{Nuclear Science and Engineering}
%     \item \emph{Nuclear Technology}
%   \end{itemize}
%   \item Conference reviewer: ANS, ANS M\&C\\
%   \item Technical Program Committee, ANS M\&C 2019\\
%   \item Member, American Nuclear Society
%  \end{itemize}

%  \begin{center}
% 	 \emph{References available on request.}
%  \end{center}

\end{resume}
\end{document}



%%%%%% EXCESS
\vspace{0.2in}
\section{\centerline{REFERENCES}}
% \vspace{-5pt} % reduce space between section title and contents
\begin{itemize}
  \item Cristian Rabiti, Relationship Manager, Idaho National Laboratory, cristian.rabiti@inl.gov
  \item Andrea Alfonsi, Technical Lead, Idaho National Laboratory, andrea.alfonsi@inl.gov
  \item Anil Prinja, Department Head, University of New Mexico, prinja@unm.edu
 \end{itemize}
 \vspace{0.2in}
%\section{\centerline{INTERESTS}}
%\vspace{-5pt}
%\begin{center}
%scripting and coding, Jazz, family, rock climbing
%\end{center}















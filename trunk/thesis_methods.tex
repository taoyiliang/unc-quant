\section{gPC: Generalized Polynomial Chaos}
In general, stochastic processes can be represented efficiently by a basis consisting of an orthogonal set of polynomials, especially if chosen correctly.  While homogeneous chaos only makes use of Hermite polynomials, a more generalized polynomial chaos (gPC) intelligently selects basis polynomials based on weighting functions.
\begin{center}
\begin{tabular}{c c c c}
Polynomial & Random Distr. & Weighting & Span \\ \hline
Legendre & Uniform & 1/2 & [-1,1] \\
Hermite & Normal &  $\exp(-x^2)/\sqrt{2\pi}$ & $(-\infty,\infty)$ \\
Laguerre & Gamma & $x^{k-1}\exp(-x)/\Gamma(k)$ & $[0,\infty)$
\end{tabular}
\end{center}
Consider an uncertain (and therefore treated as stochastic) process $U(p;\zeta)$ that is a function of its ``certain'' input parameters and phase space $p$ as well as uncertain parameters $\zeta$.  In general, $\zeta$ may be the combination of many ($\zeta_1,\zeta_2,...,\zeta_n,...,\zeta_N$) if $U$ depends on many uncertain parameters.  We wish to expand $U$ in terms of one of the polynomial bases in order to quantify its uncertainty.  The polynomial basis is chosen based on the form and span of the uncertainty, as shown in the table.  For any case, $U$ is expanded as
\begin{equation}
U(p;\zeta)\approx\sum_{i=1}^{I} c_i B_i(\zeta),
\end{equation}
where the approximation is because of term truncation at $P_t<\infty$, $c_i$ are polynomial coefficients, and $B_i$ is the polynomial of order $i$ that best fits the uncertainty in $U$.  Since the polynomials are known, we can solve for the unknown coefficients using the orthogonality of the basis polynomials as
\begin{equation}
c_i=\frac{(U(p;\zeta),B_i(\zeta))}{(B_i(\zeta)^2)},
\end{equation}
using $(\cdot)$ as inner product notation
\begin{equation}
\big(f(x),g(x) \big)\equiv \int_S f(x)g(x)dx,
\end{equation}
where $S$ is the support of $x$.


\section{SCM: Stochastic Collocation Method}
The stochastic collocation method (SCM) makes use of quadrature sets to sample from the random space generated by uncertainty.  We can make use of quadrature sets consisting of roots of the same polynomials used as basis functions in order to calculate the inner product for the gPC coefficients,
\begin{align}
(U,B_i)&\equiv\int U(\zeta) B_i(\zeta)d\zeta, \\
  &=\left( \int d\zeta_1 \int d\zeta_2 ... \int d\zeta_N\right)
        U(p;\zeta_1,\zeta_2,...,\zeta_N)B_i(\zeta_1,\zeta_2,...,\zeta_N), \\
 &\approx\left(\sum_{m_1=1}^{M_1}w_{m_1}\sum_{m_2=1}^{M_2}w_{m_2}...\sum_{m_N=1}^{M_N}w_{m_N}\right)
        U(p;\zeta_{m_1},\zeta_{m_2},...,\zeta_{m_N})B_i(\zeta_{m_1},\zeta_{m_2},...,\zeta_{m_N}),
\end{align}
where $w_{m_n}$ are weights obtained from quadrature sets corresponding to the polynomial basis chosen.  The quadrature set may or may not have the same level of truncation as the polynomial expansion; that is, $M_n$ need not be the same as $M_1$ or $I$.

We can further modify the inner product calculation by finding the coefficient term at node $\zeta_m\equiv(\zeta_{1,m_1},...,\zeta_{n,m_n},...,\zeta_{N,m_N})$ so that
\begin{align}
c_i&=\left(\sum_{m_1=1}^{M_1}w_{m_1}\sum_{m_2=1}^{M_2}w_{m_2}...\sum_{m_N=1}^{M_N}w_{m_N}\right)
        c_i(\zeta_{1,m_1},...,\zeta_{n,m_n},...,\zeta_{N,m_N}),\\
  &= \left(\sum_{m_1=1}^{M_1}w_{m_1}\sum_{m_2=1}^{M_2}w_{m_2}...\sum_{m_N=1}^{M_N}w_{m_N}\right)
        c_{i,m},\\
c_{i,m}&\equiv U(p;\zeta_m)B_i(\zeta_m),
\end{align}
where $c_{i,m}$ is the coefficient to the $i$-th order basis polynomial corresponding to a single sample realization $m$ of $U(p;\zeta)$.  Furthermore, we bring weights inward and multiply them to obtain weights that also correspond to a single realization $m$ of $U$, so that
\begin{equation}
w_m=\prod_{h=1}^N w_{m_h},
\end{equation}
\begin{equation}
c_i=\left(\sum_{m_1=1}^{M_1}...\sum_{m_n=1}^{M_n}...\sum_{m_N=1}^{M_N}\right)w_m c_{i,m}.
\end{equation}
This form of exansion is the ``full tensor product expansion,'' since it uses all possible combinations of polynomial orders for each uncertain variable.  Methods for reducing the number of combinations necessary will be discussed TODO.

\subsection{Statistics}
One way to compare the validity of stochastic collocation with generalized polynomial chaos expansion with the more standard Monte Carlo uncertainty quantification approach is to compare moments.  For TODO REASONS we expect the first and second moment to be preserved in the polynomial expansion of $U(\zeta)$.

The first moment, the mean, is obtained by finding $U(\zeta)$ at the expected value.  Since the coefficients of the expansion are constants, this is equivalent to evaluating the expansion at the expected values of the basis polynomials.  In turn, these polynomials are evaluated at the expected value from the distribution of the uncertain variable.  We use angle brackets $\expv{U(\zeta)}$ to indicate the expected value of $U(\zeta)$,
\begin{align}
\expv{U(\zeta)} &= \sum_{i}U_i\expv{B_i(\zeta)},\\
  &=\sum_i U_i B_i(\expv{\zeta}),\\
  &=\sum_i U_i B_i(\expv{\zeta_1},\expv{\zeta_2},\cdots,\expv{\zeta_N}).
\end{align}
Assuming the expansion expressed in terms of standardized distributions and only the normal and uniform distributions are considered, the expected value of all $\zeta_n$ is zero, so only the first basis function term survives,
\begin{align}
\expv{U(\zeta)} &=\sum_i U_i B_i(\expv{\zeta_1},\expv{\zeta_2},\cdots,\expv{\zeta_N}),\\
  &=U_0B_0.
\end{align}

The second moment is calculated as the expected value of the square of the function, and is used to compute the variance of the distribution,
\begin{align}
\expv{U^2(\zeta)}=\sum_i U_i B_i(\langle \zeta^2\rangle).
\end{align}
For the normal and uniform distribution, general expressions for moments of $\zeta_n$ are shown in Table \ref{tab:moments}.  For verification, moments can also be calculated by using Monte Carlo samples of $\zeta_n$ from its uncertainty distriubtion, and accumulated as shown in Table \ref{tab:moments}.
\begin{table}[h!]
\centering
\begin{tabular}{c | c | c || c}
Moment & Uniform $\zeta_n\in(a,b)$ & Normal $\zeta_n\in\mathcal{N}(\mu,\sigma^2)$ & Monte Carlo\\[3pt] \hline
1 $\langle\zeta_n\rangle$ & $\frac{1}{2}(b+a)$ & $\mu$ & $\frac{1}{N}\sum_j \zeta_{j}$\\[5pt]
2 $\langle\zeta_n^2\rangle$& $\frac{1}{2}(b^2+a^2)$ & $\mu^2+\sigma^2$ & $\frac{1}{N}\sum_j \zeta^2_{j}$
\end{tabular}
\caption{Moments for uniform, normal distributions}
\label{tab:moments}
\end{table}\\
The two statistics of interest, the expected value $\mu$ and population variance $\sigma^2$, are given by
\begin{align}
\mu &= \langle \zeta_n \rangle,\\
\sigma^2&=\langle \zeta^2_n \rangle - \langle \zeta_n\rangle^2.
\end{align}

\subsection{Constructing Multidimensional Bases}
We now give examples of expanding a multivariate function in multiple bases.  In future sections we explore alternate distributions and polynomials, as well as mapping uncertain spaces onto the [0,1] normalized shifted Legendre polynomial space; for now, we assume all random variables $\zeta_n$ are already expressed as uncertain variables with values $\in[0,1]$.

\subsubsection{Polynomials and Distributions}
We consider a set of eight typical uncertainty distributions and their corresponding polynomials and quadrature.  We summarize them in the table below, taken from TODO CITE Xiu and Karniadakis.  
\begin{center}
\begin{tabular}{c|c|c|c}
 & Unc. Distribution & Basis Polynomials & Support \\ \hline\hline
Continuous & Normal & Hermite & $(-\infty,\infty)$ \\
 & Gamma & Laguerre & $[0,\infty)$ \\
 & Beta & Jacobi & $[a,b]$ \\
 & Uniform & Legendre & $[a,b]$ \\ \hline
Discrete & Poisson & Charlier & \{0,1,2,...\}\\
 & Binomial & Krawtchouk & \{0,1,...,N\}\\
 & Negative Binomial & Meixner & \{0,1,2,...\}\\
 & Hypergeometric & Hahn & \{0,1,...,N\}
\end{tabular}
\end{center}
FIXME get rid of the discontinuous ones?  They're not in scipy.\\
Definitions and examples of these distributions are included in the appendix.

\subsubsection{Example: Single-Dimensional expansion, uniform}
Starting with the simplest case, we consider a function of single uniform-uncertainty variable $\zeta=\zeta_1$,
\begin{equation}
f(\zeta)=a+b\zeta, \hspace{10pt}\zeta\in[-1,1],
\end{equation}
where $a$ and $b$ are arbitrary scalars.  We expand $f(\zeta)$ in orthonormalized Legendre polynomials,
\begin{align}
f(\zeta)&=\sum_{i=0}^\infty f_i P_i(\zeta),\\
 &= \sum_{i=0}^1 f_i P_i(\zeta).
\end{align}
We can truncate the sum at 1 term because we know a priori $f(\zeta)$ is order 1 in $\zeta$, so it can be exactly represented by Legendre polynomials of up to order 1; in general, this is not known and perfect accuracy can only be guaranteed with infinite terms.  Using the orthogonality of the normalized Legendre polynomials, we find the coefficients $f_i$ given by
\begin{equation}
f_i=\into f(\zeta)P_i(\zeta)d\zeta.
\end{equation}
We can approximate the integral with Gauss-Legendre quadrature,
\begin{align}
f_i&=\sum_{\ell=0}^\infty w_\ell f(\zeta_\ell) P_i(\zeta_\ell),\\
 &=\sum_{\ell=0}^1 w_\ell f(\zeta_\ell) P_i(\zeta_\ell),
\end{align}
where once again, because we know the Legendre polynomial order is no greater than 1 and $f(\zeta)$ is order 1, the integral has maximum order 3 and Legendre quadrature can exactly integrate polynomials of order $2n-1$.  It is straightforward to insert the values from the Legendre quadrature set and see that the coefficients obtained are
\begin{equation}
f_0=a\sqrt{2},\hspace{10pt} f_1=b\sqrt{\frac{2}{3}}.
\end{equation}
If we reconstruct $f(\zeta)$ using these coefficients and the first three normalized Legendre polynomials, we obtain our original function $a+b\zeta$.

\subsubsection{Example: Single-Dimensional expansion, normal}
Starting with the simplest case, we consider a function of single uniform-uncertainty variable $\zeta=\zeta_1$,
\begin{equation}
f(\zeta)=a+b\zeta, \hspace{10pt}\zeta=\mathcal{N}(0,1),
\end{equation}
where $\mathcal{N}(0,1)$ indicates a Gaussian normal distribution with mean $\mu=0$ and standard deviation $\sigma=1$, and $a$ and $b$ are arbitrary scalars again.  We expand using orthonormal (physician's) Hermite polynomials,
\begin{equation}\label{expand_normal}
f(\zeta)=\sum_{i=0}^1 f_i H_i(\zeta).
\end{equation}
The Hermite polynomials are orthogonal with respect to the weighting function $W(x)=e^{-x^2}$, so that
\begin{equation}
\intf W(\zeta) H_m(\zeta) H_n(\zeta) d\zeta = \delta_{mn},
\end{equation}
The term $W(x)$ did not arise in the Legendre polynomial orthogonality discussion because it is unity, and can be omitted.  To find the expansion coefficients $f_i$, we integrate both sides of Eq. \ref{expand_normal} with respect to the weighting function and orthonormal Hermite polynomial,
\begin{align}
\intf W(\zeta)H_i(\zeta) f(\zeta) d\zeta &= \sum_{i}f_i\intf \rho(\zeta) H_i(\zeta)H_j(\zeta)d\zeta,\\
  &= f_i.
\end{align}
Before applying quadrature, it is important to note that Gauss-Hermite quadrature approximates integrals of the form
\begin{equation}
\intf W(x) f(x) dx = \sum_{\ell=0}^\infty w_\ell f(x_\ell).
\end{equation}
Often, if $f(x)$ is known a priori, it's convenient to remove $W(x)$ from $f(x)$ (say, for example, if $f(x)=x^2 e^{-x^2}$).  However, with our non-intrusive approach, we have to adjust $f(x)$ with $W^{-1}(x)$ in order to use Gauss-Hermite quadrature accurately.  For clarity, we redefine everything under the integral as
\begin{equation}
g(\zeta)\equiv W(\zeta)H_i(\zeta)f(\zeta),
\end{equation}
\begin{equation}
f_i = \intf g(\zeta)d\zeta.
\end{equation}
Because we intend to use Gauss-Hermite quadrature, we multiply by $1=W(\zeta)W^{-1}(\zeta)$,
\begin{align}
f_i&=\intf W(\zeta)\frac{g(\zeta)}{W(\zeta)}d\zeta,\\
  &=\sum_\ell^L w_\ell \frac{g(\zeta_\ell)}{W(\zeta_\ell)},\\
  &=\sum_\ell^L w_\ell \frac{W(\zeta)H_i(\zeta)f(\zeta)}{W(\zeta)},\\
    &=\sum_\ell^L w_\ell H_i(\zeta)f(\zeta).
\end{align}
In this example, we again truncate at 1st-order expansion and use order 2 quadrature,
\begin{align}
f(\zeta)=a+b\zeta&=\sum_{i=0}^{1} f_iH_i(\zeta),\\
  &=f_0H_0(\zeta) + f_1H_1(\zeta).
\end{align}
The coefficients are obtained by
\begin{align}
f_0 &= w_1H_0(\zeta_1)f(\zeta_1) + w_2H_0(\zeta_2)f(\zeta_2),\\
  &= \frac{\sqrt{\pi}}{2}\frac{1}{\sqrt[4]{\pi}}\left(a-\frac{b}{\sqrt{2}}\right) + \frac{\sqrt{\pi}}{2}\frac{1}{\sqrt[4]{\pi}}\left(a+\frac{b}{\sqrt{2}}\right),\\
  &= a\sqrt[4]{\pi}.
\end{align}
\begin{align}
f_1 &= w_1H_1(\zeta_1)f(\zeta_1) + w_2H_1(\zeta_2)f(\zeta_2),\\
  &= \frac{\sqrt{\pi}}{2}\frac{(-1)}{\sqrt[4]{\pi}}\left(a-\frac{b}{\sqrt{2}}\right) + \frac{\sqrt{\pi}}{2}\frac{1}{\sqrt[4]{\pi}}\left(a+\frac{b}{\sqrt{2}}\right),\\
  &= b\frac{\sqrt[4]{\pi}}{\sqrt{2}}.
\end{align}
Recombining,
\begin{align}
f(\zeta)=a+b\zeta&=\sum_{i=0}^{1} f_iH_i(\zeta),\\
  &=f_0H_0(\zeta) + f_1H_1(\zeta),\\
  &= a\sqrt[4]{\pi}\frac{1}{\sqrt[4]{\pi}} + b\frac{\sqrt[4]{\pi}}{\sqrt{2}}\frac{2\zeta}{\sqrt{2\sqrt{\pi}}},\\
  &= a+b\zeta.
\end{align}




\newpage
\subsubsection{Example: Shifted Range}
When the uncertain variable is on a non-standard range, it is simple to express the variable in terms of a standardly-distributed variable and make use of the standard weigths and measures.  We use the same example as the single-dimensional case above, but using standard Legendre polynomials and uniformly uncertain variables $\zeta\in(a,b),\xi\in(-1,1)$, and define
\begin{equation}
\sigma \equiv \frac{b-a}{2} \text{ (range)}, \hspace{30pt} \mu\equiv \frac{a+b}{2} \text{ (mean)}.
\end{equation}
\begin{align}
f(\zeta)&=a+b\zeta,\hspace{30pt}\zeta\in[a,b],\\
  &=\sum_i f_i P_i(\xi),\hspace{10pt}\xi\in[-1,1],
\end{align}
\begin{equation}
\zeta=\sigma\xi+\mu,\hspace{20pt}\xi=\frac{\zeta-\mu}{\sigma},
\end{equation}
\begin{align}
f_i &= \into f(\zeta(\xi))P_i^*(\xi) d\xi,\\
    &=\into f(\sigma \xi+\mu) P_i(\xi)d\xi,\\
    &\approx \sum_\ell w_\ell f(\sigma \xi_\ell+\mu)P_i(\xi_\ell).
\end{align}
In the specific example case we can truncate at two terms,
\begin{equation}
f_i=\sigma w_1 f(\sigma x_1+\mu)P_i(x_1) + w_2 f(\sigma x_2+\mu)P_i(x_2).
\end{equation}
The two coefficients are
\begin{align}
f_0 &= (1)\left[a+b\left(\frac{-\sigma}{\sqrt{3}}+\mu\right)\right]\left(\frac{1}{\sqrt{2}}\right) + (1)\left[a+b\left(\frac{\sigma}{\sqrt{3}}+\mu\right)\right] \left(\frac{1}{\sqrt{2}}\right),\\
  &=\frac{2}{\sqrt{2}}(a+b\mu),\\
  &=\sqrt{2}(a+b\mu).
\end{align}
\begin{align}
f_1 &= (1)\left[a+b\left(\frac{-\sigma}{\sqrt{3}}+\mu\right)\right]\left(\sqrt{\frac{3}{2}}\frac{(-1)}{\sqrt{3}}\right) + (1)\left[a+b\left(\frac{\sigma}{\sqrt{3}}+\mu\right)\right]\left(\sqrt{\frac{3}{2}}\frac{1}{\sqrt{3}}\right),\\
  &=b\sqrt{\frac{2}{3}}(\sigma-\mu).
\end{align}
Reconstructing the original function,
\begin{align}
f(\zeta)=a+b\zeta&=\sum_i f_iP_i(\xi),\\
  &=f_0P_0(\xi) + f_1P_1(\xi),\\
  &=(a+b\mu) + (b\sigma\xi),\\
  &=a+b\mu+b\sigma\left(\frac{\zeta-\mu}{\sigma}\right),\\
  &=a+b\zeta.
\end{align}

\newpage
\subsubsection{Example: Multivariate Expansion}
We now consider multidimensional function of $\zeta_1,\zeta_2$,
\begin{equation}
f(\zeta)\equiv f(\zeta_1,\zeta_2)=(a-b\zeta_1)(c-d\zeta_2),
\end{equation}
where $(a,b,c,d)$ are arbitrary scalars.  We expand each dimension in normalized shifted Legendre polynomials,
\begin{equation}
f(\zeta)=\sum_{i_1=0}^\infty \sum_{i_2=0}^\infty f_{i_1,i_2}\tilde P_{i_1}(\zeta_1)\tilde P_{i_2}(\zeta_2),
\end{equation}
where $f_{i_1,i_2}$ is the combined coefficient for the multivariate polynomial term.  The coefficients can be obtained in the same manner as the single dimension expansion,
\begin{equation}
f_{i_1,i_2}=\intz\intz f(\zeta)\tilde P_{i_1}(\zeta_1)\tilde P_{i_2}(\zeta_2)d\zeta_1d\zeta_2,
\end{equation}
and approximated with Legendre quadrature
\begin{align}
f_{i_1,i_2}&=\sum_{\ell_1=0}^\infty w_{\ell_1} \sum_{\ell_2=0}^\infty w_{\ell_2}
     f(\zeta_{1,\ell_1},\zeta_{1,\ell_2})\tilde P_{i_1}(\zeta_{1,\ell_1})\tilde P_{i_2}(\zeta_{2,\ell_2}),\\
  &=\sum_{\ell_1=0}^\infty \sum_{\ell_2=0}^\infty w_{\ell_1}w_{\ell_2}
     f(\zeta_{1,\ell_1},\zeta_{1,\ell_2})\tilde P_{i_1}(\zeta_{1,\ell_1})\tilde P_{i_2}(\zeta_{2,\ell_2}).
\end{align}
Using the first two terms from each sum, we obtain the coefficients
\begin{align}
f_{0,0}&=\frac{(2a+b)(2c+d)}{4},\\
f_{0,1}&= \frac{d\sqrt{3}}{12}(2a+b), \\
f_{1,0}&=\frac{b\sqrt{3}}{12}(2c+d) \\
f_{1,1}&=\frac{bd}{12},
\end{align}
\begin{align}
f(x,y)&=f_{0,0}\tilde P_0(\zeta_1)\tilde P_0(\zeta_2) +
f_{0,1}\tilde P_0(\zeta_1)\tilde P_1(\zeta_2) +
f_{1,0}\tilde P_1(\zeta_1)\tilde P_0(\zeta_2) +
f_{1,1}\tilde P_1(\zeta_1)\tilde P_1(\zeta_2),\\
&=(a+b\zeta_1)(c+d\zeta_2).
\end{align}

\subsubsection{General Multivariate Expansion}
From the two examples above, it is straightforward to extrapolate the general formulation for an expansion in an unknown number of dimensions.  We consider a function of $\zeta\equiv(\zeta_1,\zeta_2,\cdots,\zeta_n,\cdots, \zeta_N)$
\begin{equation}
f(\zeta)\equiv f(\zeta_1,\cdots,\zeta_n,\cdots, \zeta_N).
\end{equation}
We expand it in $N$ dimensions in normalized shifted Legendre polynomials,
\begin{align}
f(\zeta)&=\sum_{i_1}^\infty \sum_{i_2}^\infty \cdots\sum_{i_N}^\infty 
        f_{i_1,i_2,\cdots,i_N} \prod_{n=1}^N \tilde P_{i_n}(\zeta_n),\\
 &=\sum_{i_1}^\infty \cdots\sum_{i_N}^\infty
        f_{i} \prod_{n=1}^N \tilde P_{i_n}(\zeta_n),
\end{align}
where for simplicity we have defined $f_i$ as the coefficient for the full set of polynomials at a particular set in the sum $i=(i_1,\cdots,i_N)$.  As before, the coefficients $f_i$ are determined using orthogonality,
\begin{equation}
f_i=\into \cdots \into \left[f(\zeta)\prod_{n=1}^N \tilde P_{i_n}(\zeta_n)\right] d\zeta_1\cdots d\zeta_N,
\end{equation}
which is approximated with Legendre quadrature as
\begin{align}
f_i&=\sum_{\ell_1=0}^\infty\cdots\sum_{\ell_N=0}^\infty \left(\prod_{n=1}^N w_{\ell_n}\right) 
              f(\zeta_\ell)\prod_{n=1}^N \tilde P_{i_n}(\zeta_{n,\ell_n}),\\
  &=\sum_{\ell_1=0}^\infty\cdots\sum_{\ell_N=0}^\infty \left(\prod_{n=1}^N w_{\ell_n}\tilde P_{i_n}(\zeta_{n,\ell_n})\right) 
              f(\zeta_\ell), 
\end{align}
where for convenience we define
\begin{equation}
f(\zeta_\ell)\equiv f(\zeta_{1,\ell_1},\cdots,\zeta_{n,\ell_n},\cdots\zeta_{N,\ell_N}).
\end{equation}


%\subsection{Alternative Uncertainties}
%\subsubsection{Arbitrary Uncertainties}
%While the probability distributions and polynomial chaos above are useful in describing uncertainties, there exist many other possible uncertainty distributions.  However, it is possible to project arbitrary uncertainties into uniform [0,1] space and treat them with shifted Legendre polynomials.  In fact, we require only the percent point (or percentile) distribution of an uncertain variable to create a mapping between its natural domain and the [0,1] domain.  The drawback to this method is that shifted Legendre polynomials may not efficiently describe the distribution, and many terms may be necessary to develop an accurate representation.
%
%We follow here the pattern outlined by TODO CITE Xiu and Kerniadakis. Consider an uncertain parameter $\xi$ with arbitrary probability distribution function $f(\xi)$.  We can expand this parameter in basis polynomials that describe the desired [0,1] space; namely, (normalized) shifted Legendre polynomials $\tilde P_i$,
%\begin{equation}
%\xi=\sum_{i=0}^\infty \xi_i \tilde P_i.
%\end{equation}
%As before, we find the coefficients using the orthogonality of, and the inner product in the Hilbert space spanned by, the polynomial basis,
%\begin{equation}\label{eqn:nonsense}
%\xi_i=\int_S \xi P_i(\zeta)g(\zeta)d\zeta,
%\end{equation}
%where $g(\zeta)$ is the uniform probability distribution of $\zeta\in [0,1]$. 
%We note that Eq. \eqref{eqn:nonsense} is mathematically nonsensical, in that we assume $\zeta$ to be dependent on $\xi$ and their supports are not guaranteed to be the same; that is, they are likely to belong to different probability spaces - if not, then there is no need to perform the mapping.  To correlate the two, we introduce a new uncertain variable $u\in [0,1]$.  Recalling the probability distribution functions $f(\xi)$ and $g(\zeta)$, we transform probability space to show
%\begin{equation}
%du=f(\xi)d\xi=dF(\xi),\hspace{20pt}du=g(\zeta)=dG(\zeta),
%\end{equation}
%where $F,G$ are the cumulative distribution function (cdf)'s for $f,g$,
%\begin{equation}
%F(\xi)=\int_{-\infty}^{\xi}f(s)ds,\hspace{20pt}G(\zeta)=\int_{-\infty}^{\zeta}g(s)ds.
%\end{equation}
%We require both $\xi$ and $\zeta$ to be mapped to the domain of $u$, and show
%\begin{equation}
%\xi=F^{-1}(u),\hspace{20pt}\zeta=G^{-1}(u),
%\end{equation}
%where $F^{-1},G^{-1}$ are the inverse of the cdf, or percent point function (ppf).  Using these transformations, we return to the expansion of $\xi$ and write
%\begin{equation}
%\xi=\sum_{i=0}^\infty \xi_i P_i,
%\end{equation}
%\begin{align}
%\xi_i&=\intz F^{-1}(u)P_i\big(G^{-1}(u)\big)du,\\
%  &=\sum_{n=0}^\infty w_n F^{-1}(u_n)P_i\big(G^{-1}(u_n)\big),
%\end{align}
%where we have applied shifted Gauss-Legendre quadrature to evaluate the integral.  We note that the only requirement for mapping any arbitrary uncertainty onto a common space is the ability to evaluate the ppf of an uncertainty distribution at quadrature points ($u_n$).  Also, this procedure is general for any pdf $g(\zeta)$ to map $\xi$ onto the domain of $\zeta$; for our purposes, $\zeta\in[0,1]$ is the most beneficial.


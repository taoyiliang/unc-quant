\documentclass[11pt]{article} % use larger type; default would be 10pt
\usepackage[utf8]{inputenc} % set input encoding (not needed with XeLaTeX)
\usepackage{fullpage}
\usepackage{graphicx} % support the \includegraphics command and options
\usepackage{amsmath}
\newcommand{\drv}[2]{\ensuremath{\frac{\partial #1}{\partial #2}}}
\newcommand{\ddrv}[2]{\ensuremath{\frac{\partial^2 #1}{\partial #2^2}}}
\newcommand{\into}{\ensuremath{\int_{-1}^1}}
\newcommand{\order}[1]{\ensuremath{\mathcal{O}(#1)}}
\newcommand{\eqn}[1]{\begin{equation} #1 \end{equation}}

\title{Finite Difference Heat Equation}
\author{Paul Talbot}
%\date{}

\begin{document}
\maketitle

%%%%%%%%%%%%%%%%%%%%%%%%%%%%%%%%%%%%%
\section{Introduction}

\section{Numerical Scheme Derivation in 1D}
We begin with the one-dimensional heat equation
\eqn{\drv{u(x,t)}{t}=\alpha\ddrv{u}{x},}
where $u$ is temperature, $t$ is time, $\rho$ is the material density, $C_p$ is the specific heat capacity, $k$ is the conductivity, and $x$ is the one-dimensional axis in space.  We implicitly discretize in time steps with uniform spacing $\Delta_t$.  This leads to
\eqn{\drv{u(x,t)}{t}\approx\frac{u^{n+1}(x)-u^n(x)}{\Delta_t},}
\eqn{u(x,t)\approx u^{n+1}(x),\hspace{15pt} n=1,2,...}
where $n$ is used to denote successive time steps.  These approximations are exact in the limit as $\Delta_t$ reduces to zero, which restores the continuous time spectrum.  Applying this to the heat equation,
\eqn{\frac{u^{n+1}(x)-u^n(x)}{\Delta_t}=\alpha\ddrv{u^{n+1}(x)}{x}.}
We similarly apply a uniform spatial discretization in $x$, making use of the second-order central differencing scheme
\eqn{\ddrv{u(x,t)}{x}\approx\frac{u_{j+1}(t)-2u_{j}(t)+u_{j-1}(t)}{\Delta_x^2},}
\eqn{u(x,t)\approx u_j(t),\hspace{15pt} j=1,2,..,J ,}
where $j$ indexes the spatial dimension, with cell $j-1$ being to the left of cell $j$. The heat equation now takes the form
\eqn{\frac{u^{n+1}_j-u^n_j}{\Delta_t}=\alpha\frac{u_{j+1}^{n+1}-2u_{j}^{n+1}+u_{j-1}^{n+1}}{\Delta_x^2},}
\eqn{u^{n+1}_j-u^n_j=\Gamma\left(u_{j+1}^{n+1}-2u_{j}^{n+1}+u_{j-1}^{n+1}\right),}
\eqn{\Gamma\equiv\frac{\Delta_t\alpha}{\Delta_x^2}.}
Bringing all the implicit $n+1$ terms to the left and explicit $n$ terms to the right,
\eqn{-\Gamma u_{j+1}^{n+1}+ (1+2\Gamma)u_j^{n+1} -\Gamma u_{j-1}^{n+1}=u_j^n.\label{disc}}
For the boundary cases,
\eqn{j=1\to x_{j-1}=x_L,\label{xL}}
\eqn{j=J\to J_{j+1}=x_R,\label{xR}}
where $x_L$ and $x_R$ are the left and right faces of the problem, respectively.  Eqs. \eqref{disc} through \eqref{xR} describe a tridiagonal system that can be written as
\[
\begin{bmatrix}
1+2\Gamma & -\Gamma & 0 & 0 &...\\
-\Gamma & 1+2\Gamma & -\Gamma & 0 &...\\
0 & -\Gamma & 1+2\Gamma & -\Gamma & ...\\
 & & ... & & \\
...& -\Gamma & 1+2\Gamma & -\Gamma & 0\\
...& 0 & -\Gamma & 1+2\Gamma & -\Gamma\\
...& 0 & 0 & -\Gamma & 1+2\Gamma
\end{bmatrix}
\begin{bmatrix}
u^{n+1}_1 \\ u^{n+1}_2 \\ u^{n+1}_3 \\ ... \\ u^{n+1}_{J-2} \\ u^{n+1}_{J-1} \\ u^{n+1}_{J}
\end{bmatrix}
=
\begin{bmatrix}
u^n_1+\Gamma u_L \\ u^n_2 \\ u^n_3 \\ ... \\ u^n_{J-2} \\ u^n_{J-1} \\ u^n_J  +\Gamma u_R
\end{bmatrix},
\]
where $u_L=u(x_L,t)=u(0,t)$ and $u_R=u(x_R,t)=u(L,t)$.  This system can be solved using the Thomas algorithm (TDMA) for each time step, obtaining a forward-marching solution for $u(x,t)$.


\section{Test Cases}
\subsection{Fundamental Solution Initial Condition in 1D}
We begin with the one-dimensional heat equation
\eqn{\drv{u(x,t)}{t}=\alpha\ddrv{u}{x},}
\eqn{\alpha\equiv\frac{\rho C_p}{k},}
where $u$ is temperature, $t$ is time, $\rho$ is the material density, $C_p$ is the specific heat capacity, $k$ is the conductivity, and $x$ is the one-dimensional axis in space.  We apply vacuum boundary conditions and set the initial temperature as a fundamental solution of the time-independent heat equation,
\eqn{u(0,t)=u(L,t)=0,}
\eqn{u(x,0)=u_0\sin{\frac{\pi x}{L}},}
where $L$ is the problem length and $u_0$ is a constant maximum initial temperature.
This second-order homogeneous PDE is separable.
\eqn{u(x,t)\equiv X(x)T(t),}
\eqn{\frac{1}{\alpha}X(T')=T(X''),}
\eqn{\frac{T'}{\alpha T}=\frac{X''}{X}.}
Since the two terms are independent, the two must both be equal to the same constant. Arbitrarily,
\eqn{\frac{T'}{\alpha T}=\frac{X''}{X}=-\lambda^2.}
Considering the spatial dimension first,
\eqn{X''=-\lambda^2X,}
\eqn{X(x)=c_1\sin{\lambda x}+c_2\cos{\lambda x}.}
Applying the left boundary condition,
\eqn{X(0)=0=0+c_2,\hspace{30pt} c_2=0.}
Applying the left boundary condition,
\eqn{X(L)=0=c_1\sin{\lambda x},}
which requires either trivially $c_1=0$ or
\eqn{\lambda=\frac{n\pi}{L},\hspace{15pt}n=0,1,2,...}
\eqn{X(x)=c_1\sin{\frac{n\pi x}{L}}.}
Considering the temporal dimension,
\eqn{T'+\alpha\lambda^2 T=0.}
The root of the characteristic equation is
\eqn{R=\alpha\lambda^2 T,}
so \eqn{T(t)=c_3e^{-\alpha\lambda^2t}.}
Combing $X$ and $T$ to obtain the original variable $u$,
\eqn{u(x,t)=X(x)T(t)=c_4e^{-\alpha\lambda^2t}\sin{\frac{n\pi x}{L}}.}
The initial condition gives
\eqn{u(x,0)=u_0\sin{\frac{\pi x}{L}}=c_4\sin{\frac{n\pi x}{L}}.}
By inspection, the most straightforward solution is
\eqn{c_4=u_0,\hspace{15pt}n=1,}
\eqn{u(x,t)=u_0e^{-\alpha\lambda^2t}\sin{\frac{\pi x}{L}}.}

\section{Results}


\end{document}


\begin{figure}[h!]
\centering
\includegraphics[width=\linewidth]{graph_both_diff}
\label{diff_res}
\caption{Diffusive Results}
\end{figure}
\documentclass[11pt]{article} % use larger type; default would be 10pt
\usepackage[utf8]{inputenc} % set input encoding (not needed with XeLaTeX)
\usepackage{fullpage}
\usepackage{graphicx} % support the \includegraphics command and options
\usepackage{caption}
\usepackage{subcaption}
\usepackage{amsmath}
\usepackage{amssymb}
\usepackage{lscape}
\usepackage{pdflscape}
\usepackage{float}
\usepackage{titlesec}
\usepackage{physics} %TODO convert everything else across to this
\newcommand{\drv}[2]{\ensuremath{\frac{d #1}{d #2}}}
\newcommand{\ddrv}[2]{\ensuremath{\frac{d^2 #1}{d^2 #2}}}
\newcommand{\into}{\ensuremath{\int_{-1}^1}}
\newcommand{\intz}{\ensuremath{\int_0^1}}
\newcommand{\intf}{\ensuremath{\int_{-\infty}^\infty}}
\newcommand{\inti}{\ensuremath{\int_{x_{i-1/2}}^{x_{i+1/2}}}}
\newcommand{\intO}{\ensuremath{\int_{4\pi}}}
%\newcommand{\order}[1]{\ensuremath{\mathcal{O}(#1)}}
\newcommand{\He}{\ensuremath{\mbox{He}}}
\newcommand{\expv}[1]{\ensuremath{\mathbb{E}[ #1]}}
\newcommand{\xs}[2]{\ensuremath{\Sigma_{#1}^{(#2)}}}


\title{Tests for UQ Framework}
\author{Paul Talbot}
%\date{}

\begin{document}
\maketitle
\section{Introduction}
This work outlines the development of an uncertainty quantification (UQ) framework using generalized polynomial chaos expansions and stochastic collocation (PCESC), verified using Monte Carlo (MC) sampling.  The intended use is as a ``black-box wrapper,'' agnostic of the algorithm whose uncertaintly is quantified.  To verify the several stages this framework undergoes in development and its independence from any deterministic solver, we present here several test codes of increasing complexity that the UQ framework will act on.  The four test codes solve four problems: a polynomial expression; 1D mono-energetic neutron transport in a semi-infinite medium with uniform source and single material; 1D $k$-eigenvalue neutron diffusion transport with two energy groups and a single material; and a 2D, two energy group $k$-eigenvalue neutron diffusion transport quarter-core benchmark.

\subsection{Algorithm}
Each problem-solving code is treated as a black box that reads in an input file and produces a result readable from an output file.  The problem-solving code can be represented as a function $U$ of certain input parameters $\theta$ in deterministic parameter space $\Theta$ and uncertain parameters $Y(\omega)$ in uncertainty space $\Gamma$, where $Y$ could be a single parameter or a vector of uncertain parameters and $\omega$ is a single realization in the uncertainty space $\Gamma$.  We expand $U(\theta,Y)$ in basis polynomials characteristic of the uncertain parameters:
\begin{equation}
U(\theta;Y)\approx U_P(\theta;Y) \equiv \sum_{p=0}^P u_p(\theta) \psi_p(Y),
\end{equation}
Generally, we omit the dependency $\theta$ when considering stochastic space ($U(\theta;Y)=U(\theta))$.
$u_p(\theta)$ are polynomial expansion coefficients, $\psi_p(Y)$ are orthonormal basis polynomials, and the sum is necessarily truncated at finite order $P$.  In the limit as $P$ approaches infinity (or if $U(Y)$ can be expressed exactly as a polynomial of order $P$), there is no approximation.  Ideally the expansion converges after a reasonably small number of terms.

We make use of the orthonormal nature of the polynomial basis to calculate the coefficients $c_i$,
\begin{equation}
u_p(\theta) = \int_\Gamma U(Y)\psi_p(Y)dY.
\end{equation}
With the right choice of polynomials, we can apply quadrature to solve the integral,
\begin{equation}
u_p = \sum_{\ell=0}^{L} w_\ell U(Y_\ell) \psi_p(Y_\ell).
\end{equation}
In this case we are applying Gaussian quadrature, where an expansion of order $L$ can exactly integrate a polynomial of order $2L-1$.  While the order of the polynomial $\psi_p(Y_\ell)$ is $p$, the equivalent polynomial order of $U(Y_\ell)$ is unknown and must be determined or approximated.  If $U(Y)$ is scalar, $L$ need only be $(p+1)/2$; this is the low bound for quadrature order.  Coefficient convergence as a function of quadrature order is further explored for some of the cases in this report (see \S \ref{sec:quadconv}).

Once the coefficients are calculated, they in combination with the basis polynomials create a reduced-order model that can be sampled like the original function, but ideally at much less computational expense.  The measure of success for the PCESC algorithm is its ability to preserve the mean and variance of the original function, as well as produce a virtually identical probability density function (pdf) for the solution quantity of interest, $U(\theta;Y)$.  The mean, variance, and pdf are confirmed using brute-force Monte Carlo sampling of the original code.

\section{Deterministic Solvers}
In this section we identify the four deterministic solvers on which uncertainty propagation will be applied.

\subsection{Polynomial Solver}
We include this test case because of the analytic solution, mean, and variance.  Since we treat a solver as a black box accepting inputs and returning outputs, evaluating a polynomial is a very simple test case.
The solver performs one of the two function evaluations, depending on the number of uncertain variables:
\begin{align}
f(x) &= 1+2x,\\
f(x,y) &=x\qty(1+y).
\end{align}
The analytic first and second moments of the functions $f(x)$ and $f(x,y)$ are shown in Table \ref{tab:polymoments}.
\begin{table}[H]
\centering
\begin{tabular}{c|c|c}
Function & $\expv{f}$ & $\expv{f^2}$ \\ \hline
$f(x)$ & $1+2\expv{x}$ & $1+4\expv{x}+4\expv{x^2}$ \\
$f(x,y)$ & $\expv{x}\qty(1+\expv{x})$ & $\expv{x^2}\qty(1+2\expv{y}+\expv{y^2})$
\end{tabular}
\caption{First and Second Moments of Polynomial Solvers}
\label{tab:polymoments}
\end{table}

\subsection{Source Solver}
Like polynomial solvers, this solver evaluates a simple expression; however, there are several potential uncertain terms.  Because the solution $U$ varies nonlinearly with some of the terms, a polynomial expansion of finite degree cannot exactly represent the solution.  This solver is the analytic solution to an isotropic monoenergetic neutron source spread homogeneously throughout a purely absorbing one-dimensional semi-infinite medium.  The governing PDE is
\begin{equation}
-D\ddrv{\phi}{x}+\Sigma_a\phi = S.
\end{equation}
The unknown is the scalar flux $\phi$.  $D$ is the diffusion coefficient, $\Sigma_a$ is the macroscopic absorbing cross section, and the forcing term $S$ is the homogenous source strength.  Boundary conditions include no-traction on the left and TODO on the right.  The solution at any particular point $x$ is given by
\begin{equation}
\phi=\frac{S}{\Sigma_a}\left(1-e^{-x/L}\right),
\end{equation}
\begin{equation}
L^2= \frac{D}{\Sigma_a}.
\end{equation}
We can summarize this solver as
\begin{equation}
\phi=\phi\qty(S,D,x,\Sigma_a),\hspace{10pt}\phi\in\mathbb{R}.
\end{equation}

\subsection{1D Diffusion Solver}
This problem is a simple version of a $k$-eigenvalue criticality problem using 1D, two-energy diffusion for neutron transport.  While this problem is 1D, we use a 2D mesh to solve it by imposing reflecting boundary conditions on the top and bottom, with vacuum (no-traction) boundaries on the right and left.  We also consider only one homogeneous material for the entire mesh.  The governing PDE for this equation is
\begin{equation}
-\drv{}{x}D_g\drv{\phi_g}{x}+(\Sigma_{g,a}+\Sigma_{g,s})\phi_g = \sum_{g'}\sigma_{s}^{g'\to g}\phi_{g'} + \frac{\chi_{g}}{k(\phi)}\sum_{g'}\nu_{g'}\sigma_{f,g'}\phi_{g'},\hspace{15pt} g\in[1,2],
\end{equation}
\begin{equation}
\Sigma_{g,a}=\Sigma_{g,c}+\Sigma_{g,f},
\end{equation}
The unknown is the two-vector angular flux $\phi=(\phi_1,\phi_2)$ and the quantity of interest is criticality eigenvalue $k(\phi)$.  Group index $g$ denotes the energy group; $D$ is the group diffusion cross section; $x$ is the location within the problem; $\Sigma_a,\Sigma_s,\Sigma_f,\Sigma_c$ are the macroscopic absorption, scattering, fission, and capture cross sections respectively; and $\chi$ is the fraction of neutrons born into an energy group.  In this case, we consider only downscattering, and fission neutrons are only born into the high energy group ($\Sigma_s^{2\to1}=\chi_2=0$).

The input parameters for this solver include all the material properties ($D_g,\Sigma_{g,c},\Sigma_{g,s},\Sigma_{g,f},\nu_g$), $g=1,2$.  The output parameter is $k$.  We can summarize this solver as 
\begin{equation}
k=k\qty(D_g,\Sigma_{g,c},\Sigma_{g'\to g,s},\Sigma_{g,f},\nu_g),\hspace{10pt}g\in(1,2),\hspace{10pt}k\in\mathbb{R}.
\end{equation}

\subsection{Quarter Core Solver}
The last solver acts on the PDEs most often used in approximating nuclear reactor behavior.  In particular, we consider a two-dimensional $k$-eigenvalue steady state problem with two energy groups.  The domain is square as
\begin{equation}
D=[0,200\text{ cm}]^2,
\end{equation}
and is shown in Fig. \ref{core}.  We assign the labels top, bottom, left, and right to corresponding locations in Fig. \ref{core}.
\begin{figure}[H]
\centering
   \includegraphics[width=0.3\textwidth]{../graphics/core}
   \caption{Problem Domain}
   \label{core}
\end{figure}

We consider all fission neutrons to be initialized with fast energies, and disregard up-scattering in energy.  Our coupled PDEs are
\begin{equation}
-\grad D_1^{(R)}\grad\phi_1(\bar x)+(\xs{1,a}{R}+\xs{1\to2,s}{R})\phi_1(\bar x) = \frac{1}{k(\phi)}\sum_{g'=1}^2\nu_{g'}^{(R)}\xs{g',f}{R}\phi_{g'}(\bar x),
\end{equation}
\begin{equation}
-\grad D_2^{(R)}\grad \phi_2(\bar x)+\xs{2,a}{R}\phi_2(\bar x) = \xs{1\to 2,s}{R}\phi_1(\bar x).
\end{equation}
We impose reflecting boundary conditions on the left and bottom, and vacuum (or no-traction) boundaries on the top and right, defined by
\begin{align}
\text{Vacuum: }&-D\eval{\pdv{\phi}{x_1}}_{x_1=0}=0,\\
&-D\eval{\pdv{\phi}{x_2}}_{x_2=0}=0,
\end{align}
\begin{align}
\text{Reflective: }&\eval{\pdv{\phi}{x_1}}_{x_1=0}=0,\\
&\eval{\pdv{\phi}{x_2}}_{x_2=0}=0.
\end{align}

The unknown is the two-vector $\phi=(\phi_1,\phi_2)^T$ and the quantity of interest $k$ is the criticality eigenvalue, given by
\begin{equation}
k(\phi)=\sum_{g=1}^2\iint\limits_D\nu\xs{f}{g}\phi(x_1,x_2)dxdy,
\end{equation}
Group index $g\in(1,2)$ denotes the energy group of the property; material index $R(\bar x)\in(1,2,3,4,5)$ is the material of the property; $D$ is the neutron diffusion cross section; $\phi(\bar x)$ is the scalar neutron flux measured in neutrons per area per time;  $\nu$ is the average number of neutrons produced per fission; $\xs{g'\to g,s}{R}$ is the macroscopic neutron interaction cross section for neutrons with initial energy in group $g'$ transitioning to energy group $g$ in material $R$; $\xs{g,f}{R}$ is the macroscopic neutron fission cross section for neutrons in energy group $g$ for material $R$; and $\xs{g,a}{R}$ is the total neutron absorption interaction cross section for group $g$ in material $R$, which is further given by the macroscopic capture and fission cross sections as
\begin{equation}
\xs{g,a}{R}\equiv\xs{g,c}{R}+\xs{g,f}{R},\hspace{10pt} g=1,2.
\end{equation}
We can summarize this solver as
\begin{equation}
k=k\qty(D_g^{(R)},\xs{g,c}{R},\xs{g'\to g,s}{R},\xs{g,f}{R},\nu_g^{(R)}),\hspace{10pt} g\in(1,2),R\in(1,2,3,4,5),\hspace{10pt}k\in\mathbb{R}.
\end{equation}

\section{Univariate UQ}
In this section we limit ourselves to the analysis of the UQ algorithm as it applies to univariate deterministic solvers.

\section{Multivariate UQ}
We now extend the univariate uncertainty quantification to multivariate solvers.  For now, we limit ourselves to the tensor product collocation space, which we acknowledge as sufficient but quite inefficient.  Improving on this approach is the significant focus for future work.  We consider uncertain vector $Y=[Y_1,Y_2,\ldots,Y_N]$ as uncertain input parameters for the solution $U(Y)$.  We expand as before, but considering $N$ uncorrelated input parameters:
\begin{equation}
U(\theta;Y)\approx U_P(\theta;Y) = \sum_{p_1=0}^{P_1}\sum_{p_2=0}^{P_2}\cdots\sum_{p_N=0}^{P_N} u_{\bar p}(\theta) \prod_{n=1}^N\psi_{p_n}(Y),
\end{equation}
where now multi-index $\bar p=[p_1,p_2,\ldots,p_N]$ denotes a single set of expansion orders, one for each variable.  The multi-index is taken from the multi-index set $\Lambda(L)$, which includes all desired multi-indices.  In the case of tensor product space,
\begin{equation}
\Lambda_\text{TP}(L)=\qty{\bar p=(p_1,\ldots,p_N):\max_{1\leq n\leq N} P_n\leq L}.
\end{equation}
This tensor product space quickly becomes unmanageably large, as the size of the index set scales as
\begin{equation}
|\Lambda_\text{TP}(L)|=(L+1)^N.
\end{equation}
However, this index set will serve to demonstrate the UQ algorithm and its capacity to treat multivariate UQ for deterministic solvers.

\subsection{Polynomial Solver: $f(x,y)$}
We once again include this case for its analytic mean and variance.  We introduce uniform uncertainty to both parameters as $x\sim\mathcal{U}(3,7),y\sim\mathcal{U}(1,6)$.  The analytic expected statistics along with the Monte Carlo and PCESC statistics are shown in Table \ref{tab: poly milt res}, and the PDFs in Fig. \ref{fig: poly milt res}
\begin{table}[H]
\begin{center}
\begin{tabular}{c c|l l}
type & runs/order & mean & variance \\ \hline
Analytic & - & $45/2$ & $81.8611111111$ \\
MC & $1\times10^7$ & 22.4962950638 & 81.8754664265 \\
SC & (1,1) & 22.5 & 81.8611111111 \\
\end{tabular}
\end{center}
\caption{Polynomial Solver, Bivariate Uniform Distribution Statistics}
\label{tab: poly milt res}
\end{table}
\begin{figure}[H]
\centering
%  \begin{subfigure}[b]{0.45 \textwidth}
   \includegraphics[width=0.5\textwidth]{../graphics/poly_2v_uniform_pdfs}
   \caption{Polynomial Solver, Bivariate Uniform PDFs}
      \label{fig: poly milt res}
%  \end{subfigure}
\end{figure}

\subsection{Source Solver: $\phi=\phi\qty(S,D,x,\Sigma_a)$}
In addition to the absorption cross section, we introduce uncertainty in the location at which the flux is measured.  This occasion might arise when the exact absorption properties of a medium are unknown and a point detector is placed with some uncertainty.  We allow the input parameters to be uncertain as $\Sigma_a\sim\mathcal{U}(0.5,1),x\sim\mathcal{U}(1.5,2.5)$.  The other two parameters remain constant as
\begin{align}
S &= 1.0 \text{ n/cm}^2\text{/s},\\
D &= 0.5 \text{ /cm}.
\end{align}
The statistical results are in Table \ref{tab: source mult res} and the PDFs in Fig. \ref{fig: source mult res}.
\begin{table}[H]
\begin{center}
\begin{tabular}{c c|l l}
type & order($\Sigma_a,x$) & mean & variance \\ \hline
MC & $1\times10^6$ & 1.24791828682 & 0.0508287413676\\
SC & (2,2) & 1.24804231569 & 0.0506451101763 \\
SC & (2,4) & 1.24804212351 & 0.0506466208388\\
SC & (4,2) & 1.24806746049 & 0.0507934845282\\
SC & (4,4) & 1.24806726831 & 0.0507949951904 \\
\end{tabular}
\end{center}
\caption{Statistics for Source Solver with Bivariate Uniform Uncertainty}
\label{tab: source mult res}
\end{table}
\begin{figure}[h]
\centering
   \includegraphics[width=0.5\textwidth]{../graphics/source_2v_uniform_pdfs}
\caption{Bivariate Source Solver Solution Distributions}
\label{fig: source mult res}
\end{figure}

\newpage
\subsection{Quarter Core Solver: $k=k\qty(D_g^{(R)},\xs{g,c}{R},\xs{g'\to g,s}{R},\xs{g,f}{R},\nu_g^{(R)})$}
In this case we consider five uncertain parameters simultaneously, as in Table \ref{tab:2d2g5param}.  Each was given approximately 10\% uncertainty from its mean in the benchmark problem.
\begin{table}[H]
\begin{center}
\begin{tabular}{c c c c}
Region & Energy Group & Parameter & Uncertainty \\ \hline
1 & 2 & $\Sigma_c$ & $\mathcal{U}(0.050,0.061) $\\
1 & 2 & $\Sigma_f$ & $\mathcal{U}(0.098,0.120) $\\
4 & 2 & $\Sigma_c$ & $\mathcal{U}(0.037,0.046) $\\
4 & 2 & $\Sigma_f$ & $\mathcal{U}(0.092,0.112) $\\
5 & 2 & $D$            & $\mathcal{U}(0.143,0.175) $\\
\end{tabular}
\end{center}
\caption{Quarter Core Multivariate Uncertainty Space}
\label{tab:2d2g5param}
\end{table}

We also can attempt to make informed guesses as to the appropriate expansion order of each input parameter by performing independent convergence studies for each.  Using the results from \S \ref{sec:spaceconv}, we use expansion orders based on convergence tolerance.  The results are in Table \ref{tab:2dcrit5v}.

\begin{table}[H]
\begin{center}
\begin{tabular}{c c c|l l}
type & tol & $\mathcal{P}(\Sigma_{2,c}^1,\Sigma_{2,f}^1,\Sigma_{2,c}^4,\Sigma_{2,f}^4,D^5_2)$ & mean & variance \\ \hline
MC & - & - & 0.999064586714 & 0.0262019588485 \\
SC & 1e-4 & (5, 4, 3, 3, 1) & 1.00191085676 & 0.000202816510815 \\
SC & 1e-5 & (6, 5, 4, 3, 1) & 1.00112487809 & 0.000554497293737\\
SC & 1e-6 & (8, 7, 5, 4, 2) & 1.00018130781 & 0.00245280240592 \\
SC & 1e-8 & (10, 9, 6, 4, 3) & 1.00014901416 & 0.00207327774315
\end{tabular}
\end{center}
\caption{Statistics for Multivariate Quarter Core Solver}
\label{tab:2dcrit5v}
\end{table}
The mean is converged quite swiftly.  The variance improves quickly with decreased tolerance, but there is still an order of magnitude difference between the Monte Carlo-calculated variance and the PCESC-calculated variance.  This suggests that considering only independent convergence of parameters is insufficient, and future studies will include consideration for convergence by pairwise input parameters.


%\section{Polynomial}
We include this test case because of the analytic solution, mean, and variance.  The test code simple solves the function evaluation
\begin{equation}
U(\theta) = 1+2\theta.
\end{equation}

We consider the cases when $\theta$ has a uniform distribution as well as a normal distribution.

%\begin{table}
%\begin{center}
%\begin{tabular}{c c|l l| r}
%type & runs/order & mean & variance & run time (sec) \\ \hline
%MC & 1\times10^6 &  &  & \\
%SC & 2 & & & \\
%SC & 4 & & & \\
%SC & 8 & & & \\
%SC & 16 & & &
%\end{tabular}
%\end{center}
%\caption{}
%\label{}
%\end{table}
%
%\begin{figure}[h!]
%\centering
%   \includegraphics[width=\textwidth]{../graphics/}
%   \label{}
%   \caption{}
%\end{figure}

%\section{Semi-Infinite Uniform Source}
This case is also an evaluation of an analytic function, but can't be exactly represented by a finite polynomial expansion.  The solution models the mono-energetic neutron flux at a point inside a 1D semi-infinite homogenous absorbing medium with a uniform source.  The governing PDE for this equation is
\begin{equation}
-D\ddrv{\phi}{x}+\Sigma_a\phi = S,
\end{equation}
and its solution is
\begin{equation}
\phi(S,D,x,\Sigma_a)=\frac{S}{\Sigma_a}\left(1-e^{-x/L}\right),
\end{equation}
\begin{equation}
L^2\equiv \frac{D}{\Sigma_a}.
\end{equation}
where $S$ is the uniform source, $\Sigma_a$ is the material's macroscopic absorption cross section, $D$ is the material's diffusion coefficient, $x$ is a distance into the medium from the boundary, and $\phi$ is the neutron flux.  Restated in the form used by PCESC,
\begin{equation}
U(p;\theta) = \frac{S}{\theta}\left(1-e^{-\sqrt{\theta} x/\sqrt{D}}\right),
\end{equation}
where $p=(S,D,x)$.  
We consider the cases when the absorption cross section $\theta$ has a uniform distribution as well as a normal distribution.   For both cases, parameters $p$ are as follows.
\begin{align}
S &= 1.0 \text{ n/cm}^2\text{/s},\\
D &= 0.5 \text{ /cm},\\
x &= 2.0 \text{ cm}.
\end{align}

We allow $\Sigma_a$ to vary uniformly as $\Sigma_a\in[0.5,1]$ or normally as $\Sigma_a\in\mathcal{N}(0.75,0.15)$ and quantify the uncertainty using PCESC as well as Monte Carlo sampling.
For increasing orders of expansion, the mean and variance obtained are shown along with the run time and are shown in Tables \ref{tab:source uni} and \ref{tab:source norm}.
\begin{table}
\begin{center}
\begin{tabular}{c c|l l}
type & runs/order & mean & variance \\ \hline
MC & $1\times10^6$ & 1.26069628111 & 0.0632432419713\\
SC & 2 & 1.25774207229 & 0.0495341371244 \\
SC & 4 & 1.26064320417 & 0.0604388749588 \\
SC & 8 & 1.26108375978 & 0.0637370898233\\
SC & 16 & 1.26112339681 & 0.0639754882641
\end{tabular}
\end{center}
\caption{Statistics for Source Problem with Uniform Uncertainty}
\label{tab:source uni}
\end{table}

\begin{table}[h!]
\begin{center}
\begin{tabular}{c c|l l| r}
type & runs/order & mean & variance & run time (sec) \\ \hline
MC & 23400 & 1.24922240195 & 0.0488719424418 & 366.31\\
SC & 2 & 1.2547221522 & 0 & 2.08 \\
SC & 4 & 1.25569029702 & 0.049198975952 & 3.11 \\
SC & 8 & 1.25569096924 & 0.0492316191443 & 4.74\\
SC & 16 & 1.25569096924 & 0.0492316191611 & 6.88
\end{tabular}
\end{center}
\caption{Statistics for Source Problem with Normal Uncertainty}
\label{tab:source norm}
\end{table}

\begin{figure}[h]
\centering
  \begin{subfigure}[b]{0.45 \textwidth}
   \includegraphics[width=\textwidth]{../graphics/source_uniform_pdfs}
   \caption{Uniform PDFs}
      \label{uni}
  \end{subfigure}
  \begin{subfigure}[b]{0.45\textwidth}
   \includegraphics[width=\textwidth]{../graphics/source_normal_pdfs}
   \caption{Normal PDFs}
      \label{norm}
  \end{subfigure}
\caption{Source Problem Solution Distributions}
\label{fig:sourcepdfs}
\end{figure}

The PDFs were obtained by Monte Carlo sampling of the ROM for the PCESC cases, and obtained directly for the Monte Carlo case, shown in Fig. \ref{fig:sourcepdfs}.  The x-axis is the value of the scalar flux, and the y-axis is the probability of obtaining a particular flux.


%\begin{table}
%\begin{center}
%\begin{tabular}{c c|l l| r}
%type & runs/order & mean & variance & run time (sec) \\ \hline
%MC & 1\times10^6 &  &  & \\
%SC & 2 & & & \\
%SC & 4 & & & \\
%SC & 8 & & & \\
%SC & 16 & & &
%\end{tabular}
%\end{center}
%\caption{}
%\label{}
%\end{table}
%
%\begin{figure}[h!]
%\centering
%   \includegraphics[width=\textwidth]{../graphics/}
%   \label{}
%   \caption{}
%\end{figure}

%\section{1D 2G Homogeneous}
This problem is a simple version of a $k$-eigenvalue criticality problem using neutron diffusion.  While this problem is 1D, we use a 2D mesh to solve it by imposing reflecting boundary conditions on the top and bottom.  The governing PDE for this equation is
\begin{equation}
-\drv{}{x}D_g\drv{\phi_g}{x}+(\Sigma_{g,a}+\Sigma_{g,s})\phi_g = \sum_{g'}\sigma_{s}^{g'\to g}\phi_{g'} + \frac{\chi_{g}}{k}\sum_{g'}\nu_{g'}\sigma_{f,g'}\phi_{g'},\hspace{15pt} g\in[1,2],
\end{equation}
where $g$ denotes the energy group, $D$ is the group diffusion cross section; $\phi$ is the group flux, $x$ is the location within the problem; $\Sigma_a,\Sigma_s,\Sigma_f$ are the macroscopic absorption, scattering, and fission cross sections respectively; $k$ is the criticality factor eigenvalue and quantity of interest; and $\chi$ is the fraction of neutrons born into an energy group.  In this case, we consider only downscattering, and fission neutrons are only born into the high energy group ($\Sigma_s^{2\to1}=\chi_2=0$).

This problem does not have a convenient general analytic solution.  We can express the solver as
\begin{equation}
U(p;\theta) = k(p;\Sigma_{2,a}),
\end{equation}
where
\begin{equation}
p=(D_g,\Sigma_{1,a},\Sigma_{g,s},\nu_g,\Sigma_{g,f},\chi_g),\hspace{20pt}g\in[1,2].
\end{equation}
While $\phi_g(x)$ might also be considered a parameter, it is an output value solved simultaneously with $k$.

For this test code we consider $\theta=\Sigma_{2,a}$ in three possible normal distributions.  Evaluated at the distribution mean of $\theta$, we consider one each case where $k=(0.9,1.0,1.1)$, given by the distributions $\theta\in\mathcal{N}(0.09434,0.1), \theta\in\mathcal{N}(0.106695,0.1), \theta\in\mathcal{N}(0.08455,0.1)$ respectively.  A summary of all three cases is shown in Fig. \ref{1d_all}.  Tabular data for mean and variance convergence is in Tables \ref{tab:1dcrit} to \ref{tab:1dsub}, and the pdfs for each case are in Figs. \ref{fig:1dcrit} to \ref{fig:1dsub}.  It is important to note that the Monte Carlo sampling was restricted to values within 3 standard deviations of the mean; as such, the means and variances obtained directly through Monte Carlo sampling are not representative of the full uncertainty space.  This truncation of the distribution is enforced because without such a restriction, it is possible to sample physically untenable values for $\Sigma_{2,a}$, including negative values.

\begin{figure}[h!]
\centering
   \includegraphics[width=.75\textwidth]{../graphics/1dall_normal_pdfs}
   \caption{Summary, 1D Criticality}
   \label{1d_all}
\end{figure}

\begin{table}
\begin{center}
\begin{tabular}{c c|l l}
type & runs/order & mean & variance \\ \hline
MC & $1\times10^6$ & 1.01023757498 & 0.0092547217648 \\
SC & 2 & 1.00999398244 & 0.00857615041851 \\
SC & 4 & 1.01022188926 & 0.00918078062636\\
SC & 8 & 1.010230418 & 0.0092238915176 \\
SC & 16 & 1.01023044508 & 0.00922009179288
\end{tabular}
\end{center}
\caption{Convergence of Mean, Variance for Critical Case}
\label{tab:1dcrit}
\end{table}

\begin{table}
\begin{center}
\begin{tabular}{c c|l l}
type & runs/order & mean & variance \\ \hline
MC & $1\times10^6$ & 1.11402940816 & 0.014621003 \\
SC & 2 & 1.11386614613 &0.0133637900516 \\
SC & 4 & 1.11426467694 &0.0145502163614 \\
SC & 8 & 1.114283758 &0.0146596758645 \\
SC & 16 & 1.11428385746 &0.0146501502189
\end{tabular}
\end{center}
\caption{Convergence of Mean, Variance for Supercritical Case}
\label{tab:1dsup}
\end{table}

\begin{table}
\begin{center}
\begin{tabular}{c c|l l}
type & runs/order & mean & variance \\ \hline
MC & $1\times10^6$ & 0.90705858894 &0.0055124462906   \\
SC & 2 & 0.906911426435 & 0.00521748368937  \\
SC & 4 & 0.907033407105 & 0.00550219402953 \\
SC & 8 & 0.907036892243 & 0.00551754177997  \\
SC & 16 & 0.907036898624 & 0.00551618720453
\end{tabular}
\end{center}
\caption{Convergence of Mean, Variance for Supercritical Case}
\label{tab:1dsub}
\end{table}

\begin{figure}[h!]
\centering
   \includegraphics[width=.5\textwidth]{../graphics/1d_normal_pdfs}
   \caption{Solution PDF Convergence, 1D Critical Case}
   \label{fig:1dcrit}
\end{figure}

\begin{figure}[h!]
\centering
   \includegraphics[width=.5\textwidth]{../graphics/1dsup_normal_pdfs}
   \caption{Solution PDF Convergence, 1D Supercritical Case}
      \label{fig:1dsup}
\end{figure}

\begin{figure}[h!]
\centering
   \includegraphics[width=0.5\textwidth]{../graphics/1dsub_normal_pdfs}
   \caption{Solution PDF Convergence, 1D Supercritical Case}
      \label{fig:1dsub}
\end{figure}


%
%\begin{figure}[h!]
%\centering
%   \includegraphics[width=\textwidth]{../graphics/}
%   \label{}
%   \caption{}
%\end{figure}
%\begin{table}
%\begin{center}
%\begin{tabular}{c c|l l| r}
%type & runs/order & mean & variance & run time (sec) \\ \hline
%MC & 1\times10^6 &  &  & \\
%SC & 2 & & & \\
%SC & 4 & & & \\
%SC & 8 & & & \\
%SC & 16 & & &
%\end{tabular}
%\end{center}
%\caption{}
%\label{}
%\end{table}
%
%\begin{figure}[h!]
%\centering
%   \includegraphics[width=\textwidth]{../graphics/}
%   \label{}
%   \caption{}
%\end{figure}

%\setcounter{secnumdepth}{4}
\titleformat{\paragraph}
{\normalfont\normalsize\bfseries}{\theparagraph}{1em}{}
\titlespacing*{\paragraph}
{0pt}{3.25ex plus 1ex minus .2ex}{1.5ex plus .2ex}

\section{2D 2G Quarter Core}
\subsection{Equations}
This problem is a more traditional $k$-eigenvalue criticality problem using neutron diffusion.  We simulate a benchmark reactor core by imposing reflecting conditions on the left and bottom boundaries of a quarter-core geometry.  The governing PDE for this equation is still
\begin{equation}
-\drv{}{x}D_g\drv{\phi_g}{x}+(\Sigma_{g,a}+\Sigma_{g,s})\phi_g = \sum_{g'}\sigma_{s}^{g'\to g}\phi_{g'} + \frac{\chi_{g}}{k}\sum_{g'}\nu_{g'}\sigma_{f,g'}\phi_{g'},\hspace{15pt} g\in[1,2],
\end{equation}
\begin{equation}
\Sigma_{g,a}=\Sigma_{g,c}+\Sigma_{g,f},
\end{equation}
where $g$ denotes the energy group, $D$ is the group diffusion cross section; $\phi$ is the group flux, $x$ is the location within the problem; $\Sigma_a,\Sigma_s,\Sigma_f$ are the macroscopic absorption, scattering, and fission cross sections respectively; $k$ is the criticality factor eigenvalue and quantity of interest; and $\chi$ is the fraction of neutrons born into an energy group.  In this case, we consider only downscattering, and fission neutrons are only born into the high energy group ($\Sigma_s^{2\to1}=\chi_2=0$).  Our coupled equations are
\begin{equation}
-\drv{}{x}D_1\drv{\phi_1}{x}+(\Sigma_{1,a}+\Sigma_s^{1\to2})\phi_1 = \frac{1}{k}\sum_{g'=1}^2\nu_{g'}\sigma_{f,g'}\phi_{g'},
\end{equation}
\begin{equation}
-\drv{}{x}D_2\drv{\phi_2}{x}+\Sigma_{2,a}\phi_2 = \sigma_{s}^{1'\to 2}\phi_1,
\end{equation}
\begin{equation}
\Sigma_{g,a}=\Sigma_{g,c}+\Sigma_{g,f}.
\end{equation}

\subsection{Materials and Geometry}
The two-dimensional core is shown in Fig. \ref{coremap} and the material properties are listed in Table \ref{tab:coremats}.
\begin{figure}[H]
\centering
   \includegraphics[width=0.4\textwidth]{../graphics/core}
   \caption{Core Map}
   \label{coremap}
\end{figure}
\begin{table}[H]
\centering
\begin{tabular}{c c | c c c c}
Region & Group & $D_g$ & $\Sigma_{c,g}$ & $\nu\Sigma_{f,g}$ & $\Sigma_s^{1,2}$ \\ \hline
1 & 1 & 1.255    & 4.602e-3 & 4.602e-3 & 2.533e-2 \\
  & 2 & 2.11e-1  & 5.540e-2 & 1.091e-1 & \\ \hline
2 & 1 & 1.268    & 4.609e-3 & 4.609e-3 & 2.767e-2 \\
  & 2 & 1.902e-1 & 8.675e-2 & 8.675e-2 & \\ \hline
3 & 1 & 1.259    & 6.083e-3 & 4.663e-3 & 2.617e-2 \\
  & 2 & 2.091e-1 & 4.142e-2 & 1.021e-1 & \\ \hline
4 & 1 & 1.259    & 4.663e-3 & 4.663e-3 & 2.617e-2 \\
  & 2 & 2.091e-1 & 3.131e-2 & 1.021e-1 & \\ \hline
5 & 1 & 1.257    & 6.034e-4  & 0 & 4.754e-2 \\
  & 2 & 1.592e-1 & 1.911e-2  & 0 & 
\end{tabular}
\caption{Basic Material Properties for Core}
\label{tab:coremats}
\end{table}

\subsection{Uncertainty Quantification}
\subsubsection{Univariate}
This problem also does not have a convenient general analytic solution.  We can express the solver as
\begin{equation}
U(p;\theta) = k(p;\Sigma_{2,c}),
\end{equation}
where
\begin{equation}
p=(D_g,\Sigma_{1,c},\Sigma_{g,s},\nu_g,\Sigma_{g,f},\chi_g),\hspace{20pt}g\in[1,2].
\end{equation}
While $\phi_g(x)$ might also be considered a parameter, it is an output value solved simultaneously with $k$.

\paragraph{Uniform Uncertainty}
For this test we consider $\theta=\Sigma_{2,c}$ uniformly distributed as $\theta\in\mathcal{N}(0.0454,0.0654s)$. Tabular data for mean and variance convergence is in Table \ref{tab:2dcrit uni}, and the pdfs obtained are in Fig. \ref{fig:2dcrit uni}.

The PCESC runs all made use of order 32 quadrature to integrate chaos moments.

\begin{table}[H]
\begin{center}
\begin{tabular}{c c|l l}
type & runs/order & mean & variance \\ \hline
MC & $1\times10^6$ & 1.00406413634 & 0.000446173081079 \\
SC & 2  & 1.00416405471 & 0.000375112851817 \\
SC & 4  & 1.00416405471 & 0.000390962150246 \\
SC & 8  & 1.00416405471 & 0.000406864600682 \\
SC & 16 & 1.00416405471 & 0.000421349517322 \\
SC & 32 & 1.00416405471 & 0.000425027572716
\end{tabular}
\end{center}
\caption{Convergence of Mean, Variance for 2D2G Case}
\label{tab:2dcrit uni}
\end{table}

\begin{figure}[H]
\centering
   \includegraphics[width=\textwidth]{../graphics/2d_uniform_pdfs}
   \caption{Solution PDF Convergence, 2D2G Case}
   \label{fig:2dcrit uni}
\end{figure}

\paragraph{Normal Uncertainty}
For this test we consider $\theta=\Sigma_{2,c}$ normally distributed as $\theta\in\mathcal{N}(0.0554,0.01^2)$. Tabular data for mean and variance convergence is in Table \ref{tab:2dcrit}, and the pdfs obtained are in Fig. \ref{fig:2dcrit}.  Once again, it is important to note that the Monte Carlo sampling was restricted to values within 3 standard deviations of the mean; as such, the means and variances obtained directly through Monte Carlo sampling are not representative of the full uncertainty space.  This truncation of the distribution is enforced because without such a restriction, it is possible to sample physically untenable values for $\Sigma_{2,c}$, including negative values.

The PCESC runs all made use of order 32 quadrature to integrate chaos moments.

\begin{table}[H]
\begin{center}
\begin{tabular}{c c|l l}
type & runs/order & mean & variance \\ \hline
MC & $6\times10^5$ & 1.01333702129 & 0.00160652595587 \\
SC & 2  & 1.01643813464 & 0.00138703446968 \\
SC & 4  & 1.01643813464 & 0.00184314998697 \\
SC & 8  & 1.01643813464 & 0.00184690058216 \\
SC & 16 & 1.01643813464 & 0.00184724103523 \\
SC & 32 & 1.01643813464 & 0.00184726152781
\end{tabular}
\end{center}
\caption{Convergence of Mean, Variance for 2D2G Case}
\label{tab:2dcrit}
\end{table}

\begin{figure}[H]
\centering
   \includegraphics[width=\textwidth]{../graphics/2d_normal_pdfs}
   \caption{Solution PDF Convergence, 2D2G Case}
   \label{fig:2dcrit}
\end{figure}

\subsubsection{Multivariate}
In this case we consider five uncertain parameters simultaneously, as in Table \ref{tab:2d2g5param}.  Each was given approximately 10\% uncertainty from its mean in the benchmark problem.
\begin{table}
\begin{center}
\begin{tabular}{c c c c}
Region & Energy Group & Parameter & Uncertainty \\ \hline
1 & 2 & $\Sigma_c$ & $\mathcal{U}(0.050,0.061) $\\
1 & 2 & $\Sigma_f$ & $\mathcal{U}(0.098,0.120) $\\
4 & 2 & $\Sigma_c$ & $\mathcal{U}(0.037,0.046) $\\
4 & 2 & $\Sigma_f$ & $\mathcal{U}(0.092,0.112) $\\
5 & 2 & $D$            & $\mathcal{U}(0.143,0.175) $\\
\end{tabular}
\end{center}
\caption{Multivariate Uncertainty Space}
\label{tab:2d2g5param}
\end{table}

To explore the input space, we consider a variety of low-order expansions and one of higher order.  The results are in Table \ref{tab:2dcrit5v}.

\begin{table}[H]
\begin{center}
\begin{tabular}{c c|l l}
type & $\mathcal{O}(\Sigma_{2,c}^1,\Sigma_{2,f}^1,\Sigma_{2,c}^4,\Sigma_{2,f}^4,D^5_2)$ & mean & variance \\ \hline
MC & $1\times10^6$ & 0.995950138052 & 1.34575404516 \\
SC & (2,2,2,2,2)  & 1.00261445437 & 0.000173828352441 \\
SC & (4,2,2,2,2)  & 1.00125722573 & 0.000245435396045 \\
SC & (2,4,2,2,2)  & 1.00125489157 & 0.000244969265818 \\
SC & (2,2,4,2,2)  & 1.00261445458 & 0.000173828361529 \\
SC & (2,2,2,4,2)  & 1.00261445431 & 0.000173828341974 \\
SC & (2,2,2,2,4)  & 1.00261445434 & 0.000173828350938 \\
SC & (4,4,2,2,2)  & 1.00261444028 & 0.00017376644541\\
SC & (6,6,2,2,2)  & 1.00192854415 & 0.000216430946449 \\
SC & (8,8,8,8,8) & 0.988408029125 & 1.17178207098
\end{tabular}
\end{center}
\caption{Convergence of Mean, Variance for 2D2G Case}
\label{tab:2dcrit5v}
\end{table}

\paragraph{Convergence Study}
There are several factors to consider in the selection of expansion orders for each input parameter.  To inform us on appropriate expansion orders, we first consider each input parameter separately.  Because they are independent, holding the other parameters at their means while exploring convergence in expansion order of a single parameter gives an appropriate estimate for each expansion order in a multivariate run.  To increase run speed, we reduce the mesh refinement to its coarsest level (one grid cell per region).  We use 256-order Gauss Legendre quadrature to construct 256 expansion moments for each input parameter separately.  The results are shown in Fig. \ref{fig:2g2d5v coarse cof}.  The y-axis is the log of the magnitude of the expansion coefficient, and the x-axis is the expansion moments sorted by order.  The coefficient magnitude falls of quickly over the first several terms and plateaus for each case; however, the high value plateau for the Region 1 material cross sections suggests a lack of convergence.
\begin{figure}[H]
\centering
   \includegraphics[width=0.7\textwidth]{../graphics/coefficient_decay}
   \caption{2G2D: Coarse Mesh Coefficient Decay}
   \label{fig:2g2d5v coarse cof}
\end{figure}
We hypothesize that this lack of convergence is because of the discretization error in the solver with such a coarse mesh.   To demonstrate the effect of increasing mesh refinement, we consider only the low-energy capture cross section for the first region (1xc2, $\xs{c}{1}$).  The results are shown in Fig. \ref{fig2g2d 1xc2 cof decay}.
\begin{figure}[H]
\centering
   \includegraphics[width=0.7\textwidth]{../graphics/cof_decay_1xc2_meshes}
   \caption{2G2D: Coefficients over Mesh Refinement}
   \label{fig:2g2d 1xc2 cof decay}
\end{figure}
It is worth noting that any integer increase in the mesh refinement per region results in a large increase in mesh refinement overall.  The coarsest mesh, using a factor of 1 (1x1 grid cell per region) results in a mesh that is 11x11.  A mesh factor of 2 (2x2 grid cells per region) increases the refinement to 22x22.  This is equivalent to moving from $h=\Delta_x=\Delta_y=15$ cm to $h=7.5$ cm.  For this particular set of parameters, increase the mesh factor from 1 to 2 is plenty to assure good convergence for this cross section.  It can be the shown that the other Region 1 input parameter in this study behaves similarly and converges much further using a mesh factor of 2.

We perform the same convergence study as before, but using the more refined mesh and only continuing to order 32 expansions.  The results are shown in Fig. \ref{fig:2g2d5v fine cof}.  While the first region material cross sections don't converge as far as the fourth region and reflector (fifth region), they do converge much further.  Looking at this plot, we can choose a multivariate expansion order that accurately represents the system.  Table \ref{tab:informed 5v} shows the results of keeping different expansion orders chosen by imposing a coefficient convergence tolerance.  In each case, the quadrature order used is double the expansion order.
\begin{figure}[H]
\centering
   \includegraphics[width=0.7\textwidth]{../graphics/coefficient_decay2}
   \caption{2G2D: Coarse Mesh Coefficient Decay}
   \label{fig:2g2d5v fine cof}
\end{figure}
\begin{table}[H]
\begin{center}
\begin{tabular}{c c c|l l}
type & tol & $\mathcal{P}(\Sigma_{2,c}^1,\Sigma_{2,f}^1,\Sigma_{2,c}^4,\Sigma_{2,f}^4,D^5_2)$ & mean & variance \\ \hline
MC & - & - & 0.999064586714 & 0.0262019588485 \\
SC & 1e-4 & (5, 4, 3, 3, 1) & 1.00191085676 & 0.000202816510815 \\
SC & 1e-5 & (6, 5, 4, 3, 1) & 1.00112487809 & 0.000554497293737\\
SC & 1e-6 & (8, 7, 5, 4, 2) & 1.00018130781 & 0.00245280240592 \\
SC & 1e-8 & (10, 9, 6, 4, 3) & 1.00014901416 & 0.00207327774315
\end{tabular}
\end{center}
\caption{Convergence of Mean, Variance for 2D2G Case, Expansion Tolerances}
\label{tab:informed 5v}
\end{table}
%
%\begin{figure}[h!]
%\centering
%   \includegraphics[width=\textwidth]{../graphics/}
%   \label{}
%   \caption{}
%\end{figure}
%\begin{table}
%\begin{center}
%\begin{tabular}{c c|l l| r}
%type & runs/order & mean & variance & run time (sec) \\ \hline
%MC & 1\times10^6 &  &  & \\
%SC & 2 & & & \\
%SC & 4 & & & \\
%SC & 8 & & & \\
%SC & 16 & & &
%\end{tabular}
%\end{center}
%\caption{}
%\label{}
%\end{table}
%
%\begin{figure}[h!]
%\centering
%   \includegraphics[width=\textwidth]{../graphics/}
%   \label{}
%   \caption{}
%\end{figure}






\end{document}




\begin{center}
\begin{tabular}{c c|c c| c}
\end{tabular}
\end{center}

\begin{figure}[h]
\centering
  \begin{subfigure}[b]{0.45 \textwidth}
   \includegraphics[width=\textwidth]{}
   \caption{}
   \label{}
  \end{subfigure}
  \begin{subfigure}[b]{0.45\textwidth}
   \includegraphics[width=\textwidth]{}
   \caption{}
   \label{}
  \end{subfigure}
\caption{}
\end{figure}
\section{Index Sets}
In the previous section, we use coefficients for the multivariate polynomial expansion obtained from a tensor product space; that is, we consider all possible combinations of expansion orders for all the input parameters.  In this section we compare and contrast this \textit{tensor product} (TP) index set with the \textit{total degree} (TD) and \textit{hyperbolic cross} (HC).  We use the following notation:
\begin{align}
\bar p=(p_1,p_2,\ldots,p_N):&\text{ multi-index for expansion order of each uncertain input parameter},\\
\psi_{\bar p}=\prod_{n=1}^N\psi_{p_n}(Y_n):&\text{ product of expansion polynomials for each uncertain input parameter},\\
\Lambda_\text{T}(L):&\text{ full index set of type T with maximum index $L$ in each dimension},\\
\mathbb{P}_{\Lambda_T(L)}:&\text{ polynomial space spanned by polynomial products } \qty{\psi_{\bar p}},\bar p\in\Lambda_T(L).
\end{align}
For clarity, we consider a particular example where $\bar Y=(Y_1,Y_2)\in[-1,1]^2$ with polynomial expansion truncated at order $L=6$ in both dimensions.

\subsection{Tensor Product TP}
The tensor product space considers all possible combinations of expansion order in all dimensions.  
The index set is described as
\begin{equation}
\Lambda_{\text{TP}}(L)=\qty{\bar p=(p_1,p_2,\ldots,p_N):\max_{1\leq n\leq N}P_n\leq L},
\end{equation}
and the polynomial space is
\begin{equation}
\mathbb{P}_{\Lambda_{\text{TP}}(L)}(\Gamma)=\mathbb{P}_{\Lambda(L)}(\Gamma_1)\otimes\cdots\otimes\mathbb{P}_{\Lambda(L)}(\Gamma_N).
\end{equation}
In our example, this includes all $\bar p=(p_1,p_2)$,
\begin{table}[H]
\centering
\begin{tabular}{c c c c c c c}
(0,0)&(0,1)&(0,2)&(0,3)&(0,4)&(0,5)&(0,6)\\
(1,0)&(1,1)&(1,2)&(1,3)&(1,4)&(1,5)&(0,6)\\
(2,0)&(2,1)&(2,2)&(2,3)&(2,4)&(2,5)&(0,6)\\
(3,0)&(3,1)&(3,2)&(3,3)&(3,4)&(3,5)&(0,6)\\
(4,0)&(4,1)&(4,2)&(4,3)&(4,4)&(4,5)&(0,6)\\
(5,0)&(5,1)&(5,2)&(5,3)&(5,4)&(5,5)&(0,6)\\
(6,0)&(6,1)&(6,2)&(6,3)&(6,4)&(6,5)&(6,6)
\end{tabular},
\end{table}\noindent
for a total of 49 combined expansion coefficients.  The size of this index set scales exponentially with dimension as
\begin{equation}
|\Lambda_{\text{TP}}(L)|=(L+1)^N,
\end{equation}
and becomes very large as order and dimension are increased.  While the effectiveness of the method depends on the regularity of $U(\bar Y)$, for $U(\bar Y)$ with $k$ derivatives TODO(have to explain this hilbert space), the error goes as
\begin{equation}
||U(\bar Y)-U_{\text{TP}(L)}||\sim M^{-k/N},
\end{equation}
where the $M$ is the number of coefficients or the size of the input space,
\begin{equation}
M=|\Lambda_\text{TP}|=(L+1)^N.
\end{equation}
The tensor product index set suffers from the curse of dimensionality, where convergence will suffer with dimensionality.

\subsection{Total Degree}

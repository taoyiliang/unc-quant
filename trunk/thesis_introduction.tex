\section{Introduction}
RAVEN (Reactor Analysis and Virtual control ENvironment) is a software framework that acts as the control logic driver for thermal hydraulic code RELAP-7.  Central to the purpose of RELAP-7 is determining safety margins in accident-type scenarios for light water nuclear reactors.  

Because the inputs to RELAP-7 are likely to have some level of uncertainty in them, RAVEN allows for the capability to use brute-force Monte Carlo to quantify output uncertainty in terms of input uncertainty.  This makes it valuable as a PRA (probability risk assessment) code, and allows users to more clearly understand margins calculated with RELAP.  

We propose a method of stochastic collocation methods along with generalized polynomial chaos to sample from the uncertainty space of input variables RELAP in an intelligent way and propagate those uncertainties through the code, leveraging RAVEN's interface.  This avoids the need to introduce stochastic noise from Monte Carlo calculations and, for a limited number of uncertain inputs, offers significant speedup over brute force Monte Carlo for the same degree of precision.  Stochastic collocation may be slower than Monte Carlo as the number of uncertain variables grows, but much of this loss can be gained by employing sparse grid methods to reduce the number of necessary samples.  The accuracy cost in stochastic collocation and generalized polynomial chaos originates in truncating infinite sums to a small number of terms; the accuracy of the method generally increases with increasing terms.

We intend eventually to extend the uncertainty quantification tools in RAVEN to propagate uncertainty from inputs of one code through other coupled mutliphysics models in the MOOSE (multiphysics object-oriented simulation environment) system.  Of particular interest is BISON, a fuels performance code, which could in turn provide inputs for RAVEN.
\documentclass{beamer}

\mode<presentation>
{
  \usetheme{Antibes}
  \usecolortheme{beaver}
  % or ...

  \setbeamercovered{transparent}
  % or whatever (possibly just delete it)
}


\usepackage[english]{babel}
\usepackage[utf8]{inputenc}
\usepackage{times}
\usepackage[T1]{fontenc}
\usepackage{graphicx}
\usepackage[compatibility=false]{caption}
\usepackage{subcaption}
\usepackage{physics}
\usepackage{amsmath}
\usepackage{amssymb}

%\usepackage{multimedia}
%\usepackage{movie9}


\newcommand{\expv}[1]{\ensuremath{\mathbb{E}[ #1]}}
\newcommand{\xs}[2]{\ensuremath{\Sigma_{#1}^{(#2)}}}
\newcommand{\intO}{\ensuremath{\int\limits_{4\pi}}}
\newcommand{\intz}{\ensuremath{\int\limits_0^1}}
\newcommand{\intf}{\ensuremath{\int\limits_{-\infty}^\infty}}
\newcommand{\intzf}{\ensuremath{\int\limits_{0}^\infty}}

\title[Numerical UQ Methods] % (optional, use only with long paper titles)
{Numerical Methods for Uncertainty Quantification}

%\subtitle
%{A Term Project}

\author[Talbot] % (optional, use only with lots of authors)
{Paul W. Talbot\inst{1}}


\institute[University of New Mexico] % (optional, but mostly needed)
{
  \inst{1}%
  University of New Mexico\\
   \vspace{10pt}
\footnotesize{Supported by Idaho National Laboratory}
}

\date[BYU-I, 2014] % (optional, should be abbreviation of conference name)
{BYU-Idaho Physics Dept. Colloquium, December 2014}


\subject{Uncertainty Quantification}

\pgfdeclareimage[height=0.5cm]{university-logo}{../graphics/unmlogo}
%\logo{\pgfuseimage{university-logo}}
\logo{\makebox[0.95\paperwidth]{
  \includegraphics[height=1cm]{../graphics/INL}\hfill
    \includegraphics[height=0.5cm]{../graphics/unmlogo}}

}

\begin{document}

\begin{frame}
  \titlepage
\end{frame}

\begin{frame}{Outline}{Discussion Points}
  \tableofcontents[pausesections]
  % You might wish to add the option [pausesections]
\end{frame}

\section{Uncertainty}
\begin{frame}{Uncertainty}
\begin{itemize}
\item Aleatory (physical)
\item Epistemic (measured)
\end{itemize}
\end{frame}

\begin{frame}{Example Stochastic Problem}\vspace{-30pt}
\begin{equation}
y_f=y_i + v\sin(\theta)t - \frac{1}{2}gt^2,
\end{equation}
\begin{equation}
x_f=v\cos(\theta)t.
\end{equation}\vspace{-10pt}
Solution: $x_f=\frac{v\cos{\theta}}{g}\left(v\sin\theta+\sqrt{v^2\sin^2\theta + 2gy_i}\right)$
\end{frame}

\begin{frame}{Example Stochastic Problem}\vspace{-50pt}
\begin{equation}
x_f=\frac{v\cos{\theta}}{g}\left(v\sin\theta+\sqrt{v^2\sin^2\theta + 2gy_i}\right)
\end{equation}
\begin{itemize}
\item initial height $y_i = 2$ m
\item initial velocity $v = 50$ m/s
\item initial trajectory $\theta = 35^o$
\item accel. gravity $g = -9.81$ m/s/s
\end{itemize}
\end{frame}

\begin{frame}{Example Stochastic Problem}\vspace{-50pt}
\begin{equation}
x_f=\frac{v\cos{\theta}}{g}\left(v\sin\theta+\sqrt{v^2\sin^2\theta + 2gy_i}\right)
\end{equation}
\begin{itemize}
\item initial height $y_i = 2\pm0.1$ m
\item initial velocity $v = 50\pm5$ m/s
\item initial trajectory $\theta = 35\pm5^o$
\item accel. gravity $g = 9.81 \pm0.01$ m/s/s
\end{itemize}
\end{frame}

\begin{frame}{Uncertainty Quantification}\vspace{-20pt}
$x_f=\frac{v\cos{\theta}}{g}\left(v\sin\theta+\sqrt{v^2\sin^2\theta + 2gy_i}\right)$\\
Min-Max
\begin{itemize}
\item $x_{f,\text{min}}=\frac{}{}\left(()()+\sqrt{()^2()^2+2()()}\right)=$ m
\item $x_{f,\text{max}}=\frac{}{}\left(()()+\sqrt{()^2()^2+2()()}\right)=$ m
\end{itemize} \vspace{15pt}
Result: $y_f\approx7.16\pm1.12$m \vspace{15pt}\\ \pause
Flawed Reasoning
\begin{itemize}
\item Nonlinear Flight Path
\item Does increasing $\theta$ make a longer or shorter range?
\end{itemize}
\end{frame}

\begin{frame}{Uncertainty Quantification}\vspace{-30pt}
$x_f=\frac{v\cos{\theta}}{g}\left(v\sin\theta+\sqrt{v^2\sin^2\theta + 2gy_i}\right)$\\\vspace{10pt}
Analytic Uncertainty\\ \vspace{10pt}
$\sigma_{x_f} = \sqrt{\left(\pdv{x_f}{y_i}\right)^2\sigma_{y_i}^2 + \left(\pdv{x_f}{v}\right)^2\sigma_{v}^2 + \left(\pdv{x_f}{g}\right)^2\sigma_{g}^2 + \left(\pdv{x_f}{\theta}\right)^2\sigma_{\theta}^2}$ \vspace{20pt}\\
Result: $y_f=0\pm0$m \vspace{15pt} \\
Works well for simple functions
\begin{itemize}
\item Simple derivatives
\item Analytic solution
\end{itemize}
\end{frame}

\begin{frame}{Uncertainty Quantification}
No Air Resistance:
\begin{equation}
y_f=v\sin(\theta)t - \frac{1}{2}gt^2,
\end{equation}
\begin{equation}
x_f=v\cos(\theta)t.
\end{equation}
With Air Resistance:
\begin{equation}
y_f=\frac{v_t}{g}(v\sin\theta+v_t)\left(1-e^{-gt/v_t}\right)-v_tt,
\end{equation}
\begin{equation}
x_f=\frac{vv_t\cos\theta}{g}\qty(1-e^{-gt/v_t}).
\end{equation}
\hspace{30pt}Solve numerically to get $x_f$ (Forward Euler).
\end{frame}

\begin{frame}{Uncertainty Quantification: Complicated Problems}\vspace{-20pt}
How do we quantify uncertainty for problems without simple analytic solutions?\vspace{15pt}
\begin{itemize}
\item Monte Carlo sampling
\item Stochastic Collocation
\item High Density Model Reduction (low-order)
\end{itemize}
\end{frame}

\begin{frame}{Uncertainty Quantification}{Monte Carlo}\vspace{-30pt}
\begin{itemize}
\item Let $u(Y)$ be any system, like $x_f(v,\theta,g,y_i$)
\item Randomly sample input parameters, record outputs
\item Calculate moments (mean, variance, skew, kurtosis)
\end{itemize}
\[\expv{u^r}\approx\frac{1}{M}\sum_{m=1}^M u\left(Y^{(m)}\right)^r\]%TODO pictures
\centerline{Mean: $\bar u\approx\frac{1}{M}\sum u\left(Y^{(m)}\right)$}
\end{frame}
\end{document}



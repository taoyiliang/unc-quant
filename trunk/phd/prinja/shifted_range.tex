\documentclass[11pt]{article} % use larger type; default would be 10pt
\usepackage[utf8]{inputenc} % set input encoding (not needed with XeLaTeX)
\usepackage{fullpage}
\usepackage{graphicx} % support the \includegraphics command and options
\usepackage{amsmath}
\usepackage{amssymb}
\newcommand{\drv}[2]{\ensuremath{\frac{d #1}{d #2}}}
\newcommand{\into}{\ensuremath{\int_{-1}^1}}
\newcommand{\intz}{\ensuremath{\int_0^1}}
\newcommand{\intf}{\ensuremath{\int_{-\infty}^\infty}}
\newcommand{\inti}{\ensuremath{\int_{x_{i-1/2}}^{x_{i+1/2}}}}
\newcommand{\intO}{\ensuremath{\int_{4\pi}}}
\newcommand{\order}[1]{\ensuremath{\mathcal{O}(#1)}}
\newcommand{\He}{\ensuremath{\mbox{He}}}

\title{UQ}
\author{Paul Talbot}
%\date{}

\begin{document}
\maketitle



We use standard orthonormal Legendre polynomials and quadrature to expand $f(\zeta)$.  Since these polynomials are orthogonormal over [-1,1], I have some uncertainty on how this should work, and I can see a couple cases.  First, a review of the expansion.

\section{Standard Case}
\begin{equation}
f(\xi) \approx \sum_{i=0}^I f_i P_i(\xi).
\end{equation}
The coefficients are given because of the orthonormal Legendre polynomials,
\begin{equation}
 \into\sum_{i=0}^I f_i P_i(\xi) d\xi = f_i = \into f(\xi)P_i(\xi) d\xi,
\end{equation}
applying Gauss-Legendre quadrature,
\begin{equation}
f_i = \sum_{\ell=1}^L w_\ell f(\xi_\ell)P_i(\xi_\ell).
\end{equation}
To demonstrate a simple case, we consider the simple linear function
\begin{align}
f(\xi)&=a+b\xi,\hspace{10pt}\xi\in(-1,1),\\
  &=\sum_{i=0}^{I=1} f_iP_i(\xi).
\end{align}
\begin{align}
f_i&=\sum_{\ell=1}^{L=2} w_\ell f(\xi_\ell)P_i(\xi_\ell),\\
  &= w_1 f(\xi_1)P_i(\xi_1) + w_2 f(\xi_2)P_i(\xi_2).
\end{align}
Using the weights $w_\ell=(1,1)$ and Gauss points $\xi_\ell = \pm 1/\sqrt{3}$, as well as the orthonormal Legendre polynomials
\begin{align}
P_0(x)&=\frac{1}{\sqrt{2}},\\
P_1(x)&=\sqrt{\frac{3}{2}}x,
\end{align}
we can find our coefficients,
\begin{align}
f_0 &= (1)\left(a+b\left(\frac{-1}{\sqrt{3}}\right)\right)\frac{1}{\sqrt{2}} + (1)\left(a+b\left(\frac{1}{\sqrt{3}}\right)\right)\frac{1}{\sqrt{2}},\\
  &=\frac{1}{\sqrt{2}}\left[a-\frac{b}{\sqrt{3}}+a+\frac{b}{\sqrt{3}}\right],\\
  &= a\sqrt{2}.
\end{align}
\begin{align}
f_1 &=(1)\left(a+b\left(\frac{-1}{\sqrt{3}}\right)\right)\sqrt{\frac{3}{2}}\frac{(-1)}{\sqrt{3}} + (1)\left(a+b\left(\frac{1}{\sqrt{3}}\right)\right)\sqrt{\frac{3}{2}}\frac{1}{\sqrt{3}},\\
  &=b\sqrt{\frac{2}{3}}.
\end{align}
Reconstructing the original equation,
\begin{align}
f(\xi)&=\sum_{i=0}^{I}f_iP_i(\xi)=f_0P_0(\xi) + f_1P_1(x),\\
  &=\frac{a\sqrt{2}}{\sqrt{2}}+b\sqrt{\frac{2}{3}}\sqrt{\frac{3}{2}}\xi,\\
    &=a+b\xi.
\end{align}

\section{Adjusted Range}
The problem at hand is when $f$ is a function of a variable that isn't distributed in the standard way.  For instance, we take the same function, but of $\zeta\in(3,5)$,
\begin{equation}
f(\zeta) = a+b\zeta, \hspace{10pt}\zeta\in(3,5).
\end{equation}
We still expand $f(\zeta)$ as before,
\begin{equation}
f(\zeta) = \sum_{i=0}^{I=1}f_iP_i(\zeta).
\end{equation}
The problem, however, is in how to calculate the coefficients.  $P_i(x)$ are only orthogonal over (-1,1), but the integration range is $(a,b)$:
\begin{equation}
f_i=\int_a^b f(\zeta)P_i(\zeta) d\zeta.
\end{equation}
An integral over arbitrary range $(a,b)$ can be shited to $(-1,1)$ as
\begin{equation}
\int_a^b g(x)dx = \frac{b-a}{2}\int_{-1}^1 f\left(\frac{b-a}{2}\xi+\frac{a+b}{2}\right)dz, \hspace{10pt}x\in(a,b),\xi\in(-1,1),
\end{equation}
or, defining
\begin{equation}
\sigma \equiv \frac{b-a}{2},\hspace{20pt}\mu\equiv\frac{b+a}{2},
\end{equation}
\begin{equation}
\int_a^b g(x)dx = \sigma\int_{-1}^1 f\left(\sigma z+\mu\right)dz, \hspace{10pt}x\in(a,b),z\in(-1,1).
\end{equation}
Applying this to finding the coefficients,
\begin{equation}
f_i=\int_a^b f(\zeta)P_i(\zeta)d\zeta.
\end{equation}
The question lies how to handle $P_i(\zeta)$, covered in the sections below as ``Brute Force'' and ``Hold Polys''.

\section{Brute Force}
The first approach is simply to apply the shifted integral rule,
\begin{equation}
f_i=\int_a^b f(\zeta)P_i(\zeta)d\zeta=\sigma\into f(\sigma\xi+\mu)P(\sigma\xi+\mu)d\xi.
\end{equation}
Applying quadrature,
\begin{equation}
f_i=\sigma\sum_\ell w_\ell f(\sigma\xi_\ell+\mu)P_i(\sigma\xi_\ell+\mu).
\end{equation}
Expanding the first coefficient,
\begin{align}
f_0&=\sigma w_1f(\sigma\xi_1+\mu)P_0(\sigma\xi_1+\mu) + \sigma w_2f(\sigma\xi_2+\mu)P_0(\sigma\xi_2+\mu),\\
  &= \sigma\left(\left[\left(a+b\left[\sigma\xi_1+\mu\right]\right)\frac{1}{\sqrt{2}}\right] + 
        \left[\left(a+b\left[\sigma\xi_2+\mu\right]\right)\frac{1}{\sqrt{2}}\right]\right),\\
  &=\frac{\sigma}{\sqrt{2}}\left[\left(a+b\left[\frac{-\sigma}{\sqrt{3}}+\mu\right]\right) + 
        \left(a+b\left[\frac{\sigma}{\sqrt{3}}+\mu\right]\right)\right],\\
  &=\sigma\sqrt{2}\left(a+\mu\right).
\end{align}
However, on expanding this expression, we do not receive the expected result $f_0P_0(\zeta)$=$a$.

\section{Hold Polys}

Since the Legendre polynomials aren't orthogonal over $(a,b)$, I thought perhaps we invent alternative functions $P^*_i(\zeta)$ such that
\begin{equation}
\int_a^b P^*_i(\zeta) d\zeta = \sigma \int_{-1}^1 P_i(x)dx,
\end{equation}
\begin{equation}
\sigma \equiv \frac{b-a}{2} \text{ range}, \hspace{30pt} \mu\equiv \frac{a+b}{2} \text{ (mean)}.
\end{equation}
\begin{align}
f(\xi)&=a+b\xi,\hspace{10pt}\xi\in[a,b],\\
  &=\sum_i f_i P_i(x),
\end{align}
\begin{align}
f_i &= \int_a^b f(\xi)P_i^*(\xi) d\xi,\\
    &=\sigma\int_{-1}^1 f(\sigma x+\mu) P_i(x)dx,\\
    &\approx \sigma\sum_\ell w_\ell f(\sigma x_\ell+\mu)P_i(x_\ell),\\
    &=\sigma w_1 f(\sigma x_1+\mu)P_i(x_1) + w_2 f(\sigma x_2+\mu)P_i(x_2).
\end{align}
The first coefficient is
\begin{align}
f_0 &= \sigma\left[(1)\left(1+2\left(\frac{-\sigma}{\sqrt{3}}+\mu\right)\right)\frac{1}{\sqrt{2}} + (1)\left(1+2\left(\frac{\sigma}{\sqrt{3}}+\mu\right)\right) \frac{1}{\sqrt{2}}\right],\\
  &= \frac{b-a}{2\sqrt{2}}\left[1+2\left(\frac{-(b-a)}{2\sqrt{3}}+\frac{b+a}{2}\right) + 1+2\left(\frac{(b-a)}{2\sqrt{3}}+\frac{b+a}{2}\right) \right],\\
  &=\frac{b-a}{2\sqrt{2}}\left[2-\frac{(b-a)}{\sqrt{3}}+2b+2a + \frac{(b-a)}{\sqrt{3}}  \right],\\
  &=\frac{b-a}{\sqrt{2}}(1+b+a).
\end{align}
Unfortunately, this also does not give the expected $f_0P_0(\zeta)$=$a$.

\end{document}
\documentclass[11pt]{article}
\usepackage{amsmath}
\usepackage{amssymb}
\usepackage{amsthm}
\usepackage{amscd}
\usepackage{amsfonts}
\usepackage{graphicx}%
\usepackage{fancyhdr}
\usepackage{color}
\usepackage{cite}

%\usepackage[T1]{fontenc}
\usepackage[utf8]{inputenc}
\usepackage{authblk}
\usepackage{physics}
\usepackage{float}
\usepackage{caption}
\usepackage{subcaption}
\newcommand{\expv}[1]{\ensuremath{\mathbb{E}[ #1]}}
\newcommand{\xs}[2]{\ensuremath{\Sigma_{#1}^{(#2)}}}
\newcommand{\intO}{\ensuremath{\int\limits_{4\pi}}}
\newcommand{\intnp}{\ensuremath{\int\limits_{-1}^1}}
\newcommand{\intab}[1]{\ensuremath{\int\limits_{a_{#1}}^{b_#1}}}
\newcommand{\intz}{\ensuremath{\int\limits_0^1}}
\newcommand{\intf}{\ensuremath{\int\limits_{-\infty}^\infty}}
\newcommand{\intzf}{\ensuremath{\int\limits_{0}^\infty}}
\newcommand{\LargerCdot}{\raisebox{-0.25ex}{\scalebox{1.2}{$\cdot$}}}

\textwidth6.6in
\textheight9in


\setlength{\topmargin}{0.3in} \addtolength{\topmargin}{-\headheight}
\addtolength{\topmargin}{-\headsep}

\setlength{\oddsidemargin}{0in}

\oddsidemargin  0.0in \evensidemargin 0.0in \parindent0em

%\pagestyle{fancy}\lhead{MATH 579 (UQ for PDEs)} \rhead{02/24/2014}
%\chead{Project Proposal} \lfoot{} \rfoot{\bf \thepage} \cfoot{}


\begin{document}

\title{Adaptive gPC Convergence Criteria Norms}

\author[]{Paul Talbot\thanks{talbotp@unm.edu}}
\date{}
\renewcommand\Authands{ and }
\maketitle
\section{Notation}
\begin{itemize}
\item $\xi$ - The vector of input variables $\xi=(\xi_1,\xi_2,\cdots,\xi_N)$
\item $\Omega$ - The input space spanned by the input variable distributions
\item $\epsilon$ - The error between an iteration $\Lambda_2$ and $\Lambda_1$
\item $k$ - A combination of polynomial orders, given as a tuple
\item $\Lambda_1$ - The set of all desired polynomial combinations $k$ for the current iteration
\item $\Lambda_2$ - The set of all desired polynomial combinations $k$ for the next iteration
\item $c_{i,k}$ - The scalar coefficient for polynomial combination with orders $k$ in index set $\Lambda_i$
\item $\Phi_k$ - The multidimensional orthonormal polynomial basis with degrees $k$.
\end{itemize}
\section{L2 of Difference Space}
\begin{equation}
\epsilon = \sqrt{\int_\Omega d\xi \left[ \sum_{k\in\Lambda_2} c_{2,k}\Phi_k(\xi) - \sum_{k\in\Lambda_1} c_{1,k}\Phi_k(\xi) \right]^2}.
\end{equation}

Let $c_{1,k}=0\forall k\notin\Lambda_1$, and noting
\begin{equation}
\int_\Omega \Phi_j(\xi)\Phi_k(\xi)P(\xi) d\xi = \delta_{jk},
\end{equation}

\begin{align}
\epsilon &= \sqrt{\int_\Omega d\xi \left[ 
    %\sum_{k\in\Lambda_2} c^2_{2,k}\Phi_k^2(\xi)
   %-2 \sum_{k\in\Lambda_2}c_{2,k}c_{1,k}\Phi_k^2(\xi)
    %+ \sum_{k\in\Lambda_2}c_{1,k}^2\Phi_k^2(\xi) 
     \sum_{m=1}^2\sum_{n=1}^2\sum_{j\in\Lambda_2}\sum_{k\in\Lambda_2}c_{m,k}c_{n,j}\Phi_k(\xi)\Phi_j(\xi)
    \right]},\\
  &= \sqrt{\sum_{k\in\Lambda_2} c^2_{2,k} -2 \sum_{k\in\Lambda_2}c_{2,k}c_{1,k} + \sum_{k\in\Lambda_2}c_{1,k}^2}, \\
  &= \sqrt{\sum_{k\in\Lambda_2}(c_{2,k}-c_{1,k})^2}.
\end{align}


\end{document}
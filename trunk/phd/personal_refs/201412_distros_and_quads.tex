\documentclass[11pt]{article}
\usepackage{amsmath}
\usepackage{amssymb}
\usepackage{amsthm}
\usepackage{amscd}
\usepackage{amsfonts}
\usepackage{graphicx}%
\usepackage{fancyhdr}
\usepackage{color}
\usepackage{cite}

%\usepackage[T1]{fontenc}
\usepackage[utf8]{inputenc}
\usepackage{authblk}
\usepackage{physics}
\usepackage{float}
\usepackage{caption}
\usepackage{subcaption}
\newcommand{\expv}[1]{\ensuremath{\mathbb{E}[ #1]}}
\newcommand{\xs}[2]{\ensuremath{\Sigma_{#1}^{(#2)}}}
\newcommand{\intO}{\ensuremath{\int\limits_{4\pi}}}
\newcommand{\intz}{\ensuremath{\int\limits_0^1}}
\newcommand{\intf}{\ensuremath{\int\limits_{-\infty}^\infty}}
\newcommand{\intzf}{\ensuremath{\int\limits_{0}^\infty}}
\newcommand{\LargerCdot}{\raisebox{-0.25ex}{\scalebox{1.2}{$\cdot$}}}

\textwidth6.6in
\textheight9in


\setlength{\topmargin}{0.3in} \addtolength{\topmargin}{-\headheight}
\addtolength{\topmargin}{-\headsep}

\setlength{\oddsidemargin}{0in}

\oddsidemargin  0.0in \evensidemargin 0.0in \parindent0em

%\pagestyle{fancy}\lhead{MATH 579 (UQ for PDEs)} \rhead{02/24/2014}
%\chead{Project Proposal} \lfoot{} \rfoot{\bf \thepage} \cfoot{}


\begin{document}

\title{Distributions and Quadratures}

\author[]{Paul Talbot\thanks{talbotp@unm.edu}}
%\date{}
\renewcommand\Authands{ and }
\maketitle

\section{General Syntax}
\begin{itemize}
\item Probability Measure: $f(y)$
\item Normalization Factor: $A$
\item Probability Distribution: $Af(y)$
\item Generic function: $h(y)$
\item Lower bound $L$
\item Upper bound $R$
\end{itemize}
\newpage

\section{Distributions}
\subsection{Uniform}
Continuous values $y\in[L,R]$.
\begin{equation}
\mu = \frac{R+L}{2},
\end{equation}
\begin{equation}
\sigma = \frac{R-L}{2}
\end{equation}
\begin{equation}
f(y) = \frac{1}{2\sigma},
\end{equation}
\begin{equation}
A=1,
\end{equation}
\begin{equation}
  1 = \frac{1}{2\sigma}\int_L^R dy,
\end{equation}

\subsection{Normal}
Continuous values $y\in(-\infty,\infty)$.
\begin{equation}
f(y) = \exp\qty(-\frac{(y-\mu)^2}{2\sigma^2}),
\end{equation}
\begin{equation}
A = \frac{1}{\sigma\sqrt{2\pi}}
\end{equation}
\begin{equation}
  1 =\frac{1}{\sigma\sqrt{2\pi}} \int_{-\infty}^\infty \exp\qty( -\frac{(y-\mu)^2}{2\sigma^2} )dy,
\end{equation}

\subsection{Gamma}
Continuous values $y\in[L,\infty)$.
\begin{equation}
f(y) = y^{\alpha-1}e^{-\beta y},
\end{equation}
\begin{equation}
A = \frac{\beta^\alpha}{\Gamma(\alpha)},
\end{equation}
\begin{equation}
1 = \frac{\beta^\alpha}{\Gamma(\alpha)}\int_L^\infty (y-L)^{\alpha-1}e^{-\beta (y-L)} dy,
\end{equation}

\subsection{Beta}
Continuous values $y\in[L,R]$.
\begin{equation}
f(y) = y^{\alpha-1}(1-y)^{\beta-1},
\end{equation}
\begin{equation}
A =\frac{ \Gamma(\alpha+\beta)}{\Gamma(\alpha)\Gamma(\beta)},
\end{equation}
\begin{equation}
1 = \frac{ \Gamma(\alpha+\beta)}{\Gamma(\alpha)\Gamma(\beta)}\int_L^R y^{\alpha-1}(1-y)^{\beta-1} dy.
\end{equation}

\subsection{Triangular}
Continuous values $y\in[L,R]$.
$c$ is the $y$-value where the apex sits, and has a value of $\frac{2}{R-L}$.
\begin{equation}
f(y) = \left\{
 \begin{array}{lr}
  \frac{y-L}{c-L} & L\leq y < c,\vspace{5pt} \\ 
  1 & y=c, \vspace{5pt} \\
  \frac{R-y}{R-c} & c<y\leq R, \vspace{5pt} \\
  0 & \text{else}.
 \end{array}
\right.
\end{equation}
\begin{equation}
A = \frac{2}{R-L},
\end{equation}
\begin{equation}
1 = \int_L^c \frac{2}{(R-L)}\frac{y-L}{(c-L)} dy + \int_c^R \frac{2}{(R-L)}\frac{R-y}{(R-c)} dy
\end{equation}

\subsection{Poisson}
Discrete values $y\in\mathbb{N}_0$.
\begin{equation}
f(y) = \frac{\lambda^y}{y!},
\end{equation}
\begin{equation}
A = e^{-\lambda},
\end{equation}
\begin{equation}
1 = e^{-\lambda}\sum_{y=0}^\infty  \frac{\lambda^y}{y!},\hspace{10pt}y\in\mathbb{N}_0.
\end{equation}

\subsection{Binomial}
Discrete values $y\in\mathbb{N_0},y<n$,
\begin{equation}
f(y)=p^y(1-p)^{n-y},
\end{equation}
\begin{equation}
A=\frac{n!}{y!(n-y)!},
\end{equation}
\begin{equation}
1=\frac{n!}{y!(n-y)!}\sum_{y=0}^n p^y(1-p)^{n-y}.
\end{equation}

\subsection{Bernoulli}
Boolean values $y\in[0,1]$.
\begin{equation}
f(y)=\left\{
  \begin{array}{lr}
    1-p & y=0,\\
    p & y=1,
  \end{array}
\right.
\end{equation}
\begin{equation}
A=1,
\end{equation}
\begin{equation}
1 = \sum_{y=1}^2 f(y) = p + (1-p).
\end{equation}

\subsection{Logistic}
Continuous values $y\in(-\infty,\infty)$.
\begin{equation} % \frac{\exp\qty(-\frac{y-\mu}{s})}{s\qty(1+\exp\qty(-\frac{y-\mu}{s}))}
f(y) =\sech^2\qty(\frac{y-\mu}{2s}),
\end{equation}
\begin{equation}
A=\frac{1}{4s},
\end{equation}
\begin{equation}
1=\frac{1}{4s}\int_{-\infty}^\infty \sech^2\qty(\frac{y-\mu}{2s}) dy.
\end{equation}

\subsection{Exponential}
Continuous values $y\in[0,\infty)$ TODO make L to infinity
\begin{equation}
f(y)=e^{-\lambda x},
\end{equation}
\begin{equation}
A=\lambda,
\end{equation}
\begin{equation}
1=\int_0^\infty \lambda e^{-\lambda x}.
\end{equation}

\subsection{Arbitrary}
Continuous values $-\infty < L \leq y \leq R < \infty$
\begin{equation}
f(y) = f(y),
\end{equation}
\begin{equation}
F(y) = \int_{L}^y f(y')dy',
\end{equation}
\begin{equation}
1 = \int_{L}^R f(y')dy'. 
\end{equation}


\section{Quadrature}
\subsection{Legendre}
\begin{equation}
 \int_{-1}^1 h(x)d(x) = \sum_{\ell=1}^\infty w_\ell h(x_\ell)
\end{equation}

\subsection{Hermite}
\begin{equation}
\int_{-\infty}^\infty h(x)\exp(\frac{-x^2}{2})dy = \sum_{h=1}^\infty w_h h(x_h)
\end{equation}

\subsection{Laguerre}
\begin{equation}
\int_0^\infty h(x) x^\alpha e^{-x} dx = \sum_{\mathcal{L}=1}^\infty w_\mathcal{L} h(x_\mathcal{L})
\end{equation}

\subsection{Jacobi}
\begin{equation}
\int_{-1}^1 h(x)(1-x)^\alpha (1+x)^\beta dx = \sum_{j=1}^\infty w_j h(x_j)
\end{equation}

\subsection{Clenshaw-Curtis}
\begin{equation}
\int_{-1}^1 h(x)dx = \sum w_{cc} h(x_{cc})
\end{equation}

\section{Conversions}
\subsection{Uniform and Legendre}
\begin{equation}
y = \sigma x+\mu,
\end{equation}
\begin{equation}
x = \frac{y-\mu}{\sigma},
\end{equation}
\begin{equation}
\int_a^b h(y)f_\ell(y)dy =\frac{1}{2} \sum_{\ell=1}^\infty w_\ell h(\sigma x_\ell+\mu)
\end{equation}

\subsection{Normal and Hermite}
\begin{equation}
y = \sigma x+\mu,
\end{equation}
\begin{equation}
x = \frac{y-\mu}{\sigma},
\end{equation}
\begin{equation}
\int_{-\infty}^\infty h(y)f_h(y)dy =\frac{1}{\sqrt{2\pi}} \sum_{h=1}^\infty w_h h(\sigma x_h+\mu)
\end{equation}

\subsection{Gamma and Laguerre}
\begin{equation}
y = \frac{x}{\beta}+L,
\end{equation}
\begin{equation}
x = (y-L)\beta,
\end{equation}
\begin{equation}
\int_{L}^\infty h(y)f_g(y)dy =\frac{1}{(\alpha-1)!} \sum_{g=1}^\infty w_g h\qty(\frac{x_g}{\beta}+L)
\end{equation}
Points $x_g$ and weights $w_g$ must be obtained from Laguerre quadrature as
\begin{verbatim}
pts,wts = laguerre_generator(alpha-1).
\end{verbatim}

\subsection{Beta and Jacobi}
General Beta:
\begin{equation}
1=\frac{1}{\text{B}(\alpha,\beta)(R-L)}\int_L^R \qty(\frac{y-L}{R-L})^{\alpha-1}\qty(1-\frac{y-L}{R-L})^{\beta-1}dy,
\end{equation}
To convert to standard Beta:
\begin{equation}
z=\frac{y-L}{R-L},\hspace{10pt}y=(R-L)z+L,\hspace{10pt} dy=(R-L)dz,
\end{equation}
\begin{equation}
1=\frac{1}{\text{B}(\alpha,\beta)}\int_0^1 z^{\alpha-1}(1-z)^{\beta-1}dz,
\end{equation}
To convert to same form as Jacobi:
\begin{equation}
z=\frac{1+x}{2},\hspace{10pt}x=2z-1,\hspace{10pt}dz=\frac{1}{2}dx,
\end{equation}
\begin{equation}
1=\frac{1}{2^{\alpha+\beta-1}\text{B}(\alpha,\beta)}\int_{-1}^1 (1+x)^{\alpha-1}(1-x)^{\beta-1} dx,
\end{equation}
Combined:
\begin{equation}
y=\frac{R-L}{2}x+\frac{R+L}{2},\hspace{10pt}x=\qty(y-\frac{R+L}{2})\qty(\frac{2}{R-L})
\end{equation}
Especially note the naming convention
\begin{equation}
\alpha_\text{Jacobi} = \beta_\text{Beta}-1,\hspace{15pt}\beta_\text{Jacobi} = \alpha_\text{Beta}-1,
\end{equation}
So,
\begin{equation}
\int_L^R h(y)f_B(y)dy = \frac{1}{2^{\alpha_B+\beta_B-1}\text{B}(\alpha_B,\beta_B)}\sum_{b=1}^\infty w_b h\qty(\frac{R-L}{2}x_b+\frac{R+L}{2}).
\end{equation}
Points $x_j$ and weights $w_j$ must be obtained from Jacobi quadrature as
\begin{verbatim}
pts,wts = jacobi_generator(beta-1, alpha-1).
\end{verbatim}


\subsection{Arbitrary and Legendre}
Let $u\in[0,1]$, and note $F(y)\in[0,1]$.
\begin{equation}
du = dF(y) = f(y)dy,
\end{equation}
\begin{equation}
F(y)=u\hspace{10pt}\therefore\hspace{10pt}y=F^{-1}(u),
\end{equation}
\begin{equation}
dy = \frac{1}{f(y)}du,
\end{equation}
\begin{align}
\int_L^R h(y)f(y)dy &= \int_0^1 h\qty(F^{-1}(u))f\qty(F^{-1}(u))\frac{1}{f\qty(F^{-1}(u))}du,\\
  &=\int_0^1 h\qty(F^{-1}(u))du.
\end{align}
\begin{equation}
x = \frac{u-\mu}{\sigma}\therefore u=\sigma x+\mu,
\end{equation}
\begin{equation}
u = \frac{R-L}{2}x+\frac{R+L}{2},\hspace{10pt}R=1,L=0,
\end{equation}
\begin{equation}
u = \frac{1}{2}(x+1),
\end{equation}
\begin{align}
\int_L^R h(y)f(y)dy&= \int_0^1 h\qty(F^{-1}(u))du,\\
  &=\frac{1}{2}\sum_{\ell=1}^\infty w_\ell h\qty(F^{-1}\qty(\frac{1}{2}(x+1)))du.
\end{align}

\end{document}
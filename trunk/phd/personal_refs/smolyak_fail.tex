\documentclass[11pt]{article}
\usepackage{amsmath}
\usepackage{amssymb}
\usepackage{amsthm}
\usepackage{amscd}
\usepackage{amsfonts}
\usepackage{graphicx}%
\usepackage{fancyhdr}
\usepackage{color}
\usepackage{cite}

%\usepackage[T1]{fontenc}
\usepackage[utf8]{inputenc}
\usepackage{authblk}
\usepackage{physics}
\usepackage{float}
\usepackage{caption}
\usepackage{subcaption}
\newcommand{\expv}[1]{\ensuremath{\mathbb{E}[ #1]}}
\newcommand{\xs}[2]{\ensuremath{\Sigma_{#1}^{(#2)}}}
\newcommand{\intO}{\ensuremath{\int\limits_{4\pi}}}
\newcommand{\intnp}{\ensuremath{\int\limits_{-1}^1}}
\newcommand{\intab}[1]{\ensuremath{\int\limits_{a_{#1}}^{b_#1}}}
\newcommand{\intz}{\ensuremath{\int\limits_0^1}}
\newcommand{\intf}{\ensuremath{\int\limits_{-\infty}^\infty}}
\newcommand{\intzf}{\ensuremath{\int\limits_{0}^\infty}}
\newcommand{\LargerCdot}{\raisebox{-0.25ex}{\scalebox{1.2}{$\cdot$}}}

\textwidth6.6in
\textheight9in


\setlength{\topmargin}{0.3in} \addtolength{\topmargin}{-\headheight}
\addtolength{\topmargin}{-\headsep}

\setlength{\oddsidemargin}{0in}

\oddsidemargin  0.0in \evensidemargin 0.0in \parindent0em

%\pagestyle{fancy}\lhead{MATH 579 (UQ for PDEs)} \rhead{02/24/2014}
%\chead{Project Proposal} \lfoot{} \rfoot{\bf \thepage} \cfoot{}


\begin{document}

\title{Failure Case for the Smolyak Sparse Quadrature for gPC Expansion}

\author[]{Paul Talbot\thanks{talbotp@unm.edu}}
\date{}
\renewcommand\Authands{ and }
\maketitle
\section{Introduction}
Let $(\Omega,\mathcal{F},rho)$ be a complete $N$-variate probability space.
We consider the algorithms for expanding a quantity of interest $u(Y)$ as a function of uncertain independent
input parameters $Y = (y_1,\cdots,y_n,\cdots,y_N)$ in a generalized polynomial chaos expansion using
orthonormal Gaussian polynomials $\phi_i^{(n)}(y_n)$.  These Gaussian polynomials are orthonormal with respect
to their corresponding individual monovariate probability space $(\Omega_n,\mathcal{F}_n,\rho_n)$.  The
expansion is given by
\begin{equation}\label{eq:gpc}
  u(Y)\approx \tilde u(Y) = \sum_{k\in\Lambda} c_k \Phi_k(Y),
\end{equation}
where $k$ is a multivariate index, $\Lambda$ is a set of $N$-variate indices corresponding to polynomial
orders, and $\Phi_k$ are a set of orthonormal multidimensional polynomials given by
\begin{equation}
  \Phi_k(Y) = \prod_{n=1}^N \phi_{k_n}(y_n).
\end{equation}
We assume $\Lambda$ can be constructed adaptively.  The admissability condition for new indices $k$ into
$\Lambda$ is
\begin{equation}
  k-e_j \in \Lambda \forall 1\leq j\leq N\,
\end{equation}
where $e_j$ is a unit vector in the direction of $j$.

The scalar coefficients $c_k$ in Eq. \ref{eq:gpc} can be obtained via the orthonormality of $\Phi_k$ as
\begin{equation}\label{eq:coeffs}
  c_k = \int_\Omega \rho(Y) u(Y) \Phi_k(Y) dY \equiv \mathcal{I}\big(u\cdot\Phi_k\big).
\end{equation}
We approximate the integral using Smolyak-like sparse quadrature. Using the notation for a single-dimension
quadrature operation
\begin{equation}
  \int \rho(x)f(x)dx = \mathcal{I}(f) \approx \sum_{\ell=1}^L w_\ell f(x_\ell) \equiv q^{L}(f),
\end{equation}
the sparse quadrature is given by
\begin{equation}
  \mathcal{I}\big(u\cdot\Phi\big)\approx\mathcal{S}[u\cdot\Phi]\equiv \sum_{\hat k\in\Lambda} s_{\hat k} \bigotimes_{n=1}^N
  q^{L_n}\big(u(Y)\Phi_k(Y)\big).
\end{equation}
The quadrature coefficient $s_{\hat k}$ is given by
\begin{equation}
  s_{\hat k} = \sum_{j\in\{0,1\}^N,i+j\in\Lambda} (-1)^{|j|_1}, \hspace{10pt} |j|_1 = \sum_{n=1}^N j_n. 
\end{equation}

We demonstrate here that for a particular index set $\Lambda$, the Smolyak algorithm does not accurately
integrate the quadrature coefficients.

\section{Case}
For demonstration, we consider the quantity of interest
\begin{equation}
  u(x,y) = x^2 y^2,
\end{equation}
with $x$ and $y$ uniformly distributed from -1 to 1.  In this case, we use orthonormalized Legendre
polynomials for the expansions polynoials $\phi$.

For expansion polynomials, we consider as an example the following polynomial set,
\begin{table}[H]
  \centering
  \begin{tabular}{c c c c c c c c c}
    (8,0) &       &       &      &       &       &       &       &      \\
    (7,0) &       &       &      &       &       &       &       &      \\
    (6,0) &       &       &      &       &       &       &       &      \\
    (5,0) &       &       &      &       &       &       &       &      \\
    (4,0) &       &       &      &       &       &       &       &      \\
    (3,0) & (3,1) &       &      &       &       &       &       &      \\
    (2,0) & (2,1) & (2,2) &      &       &       &       &       &      \\
    (1,0) & (1,1) & (1,2) &(1,3) &       &       &       &       &      \\
    (0,0) & (0,1) & (0,2) &(0,3) & (0,4) & (0,5) & (0,6) & (0,7) & (0,8) 
  \end{tabular}
\end{table}
Because it includes the index set point (2,2), we expect this expansion to exactly represent the original
quantity of interest.

\section{Analytic}
First, we demonstrate the correct, analytic performance of the gPC expansion.  The polynomial coefficients
$c_k$ are given by Eq. \ref{eq:coeffs}.  Each coefficient integrates to zero with the exception of the
following:
\begin{align}
  c_{(0,0)} &= \frac{1}{9},\\
  c_{(0,2)} = c_{(2,0)} &= \frac{2}{9\sqrt{5}},\\
  c_{(2,2)} &= \frac{4}{45}.
\end{align}
Reconstructing the original model from the expansion, as expected we recover the original model exactly.

\section{Smolyak}
In order to be sufficient in a general sense, we desire the Smolyak sparse quadrature algorithm to perform
as accurately as the analytic case for this polynomial quantity of interest case.  We begin by evaluating the
values of the quadrature coefficients $s_{\hat k}$.  These are all zero with the exception of the following:
\begin{align}
  s_{(0,3)} = s_{(3,0)} &= -1, \\
  s_{(0,8)} = s_{(8,0)} &=  1, \\
  s_{(1,2)} = s_{(2,1)} &= -1, \\
  s_{(1,3)} = s_{(3,1)} &=  1, \\
  s_{(2,2)} = s_{(2,2)} &=  1.
\end{align}
Using the quadrature order rule $L=k_n+1$, we will need points and weights for Legendre quadrature orders 1,
2, 3, 4 and 9.  The points and weights are listed here for convenience.
\begin{table}[H]
  \centering
  \begin{tabular}{c c c}
    Quadrature Order & Points & Weights \\ \hline
    1 & 0 & 2 \\ \hline
    2 & $\pm$ 0.5773502691896257 & 1\\ \hline
    3 & $\pm$ 0.7745966692414834 & 0.5555555555555556 \\
      & 0                        & 0.8888888888888888 \\ \hline
    4 & $\pm$ 0.8611363115940526 & 0.3478548451374538 \\
      & $\pm$ 0.3399810435848563 & 0.6521451548625461 \\ \hline
    9 & $\pm$ 0.9681602395076261 & 0.0812743883615744 \\
      & $\pm$ 0.8360311073266358 & 0.1806481606948574 \\
      & $\pm$ 0.6133714327005904 & 0.2606106964029354 \\
      & $\pm$ 0.3242534234038089 & 0.3123470770400029 \\
      & 0                        & 0.3302393550012598
  \end{tabular}
\end{table}
There are nine distinct tensor quadratures necessary to construct the Smolyak-like quadrature set, four of
which are duplicated because of symmetry.  This results in the following Smolyak-like quadrature set:
\begin{table}[H]
  \centering
  \begin{tabular}{c c c}
    Tensor & Points & Weights \\

\end{document}

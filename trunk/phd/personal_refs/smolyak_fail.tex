\documentclass[11pt]{article}
\usepackage{amsmath}
\usepackage{amssymb}
\usepackage{amsthm}
\usepackage{amscd}
\usepackage{amsfonts}
\usepackage{graphicx}%
\usepackage{fancyhdr}
\usepackage{color}
\usepackage{cite}

\usepackage{longtable}

%\usepackage[T1]{fontenc}
\usepackage[utf8]{inputenc}
\usepackage{authblk}
\usepackage{physics}
\usepackage{float}
\usepackage{caption}
\usepackage{subcaption}
\newcommand{\expv}[1]{\ensuremath{\mathbb{E}[ #1]}}
\newcommand{\xs}[2]{\ensuremath{\Sigma_{#1}^{(#2)}}}
\newcommand{\intO}{\ensuremath{\int\limits_{4\pi}}}
\newcommand{\intnp}{\ensuremath{\int\limits_{-1}^1}}
\newcommand{\intab}[1]{\ensuremath{\int\limits_{a_{#1}}^{b_#1}}}
\newcommand{\intz}{\ensuremath{\int\limits_0^1}}
\newcommand{\intf}{\ensuremath{\int\limits_{-\infty}^\infty}}
\newcommand{\intzf}{\ensuremath{\int\limits_{0}^\infty}}
\newcommand{\LargerCdot}{\raisebox{-0.25ex}{\scalebox{1.2}{$\cdot$}}}

\textwidth6.6in
\textheight9in


\setlength{\topmargin}{0.3in} \addtolength{\topmargin}{-\headheight}
\addtolength{\topmargin}{-\headsep}

\setlength{\oddsidemargin}{0in}

\oddsidemargin  0.0in \evensidemargin 0.0in \parindent0em

%\pagestyle{fancy}\lhead{MATH 579 (UQ for PDEs)} \rhead{02/24/2014}
%\chead{Project Proposal} \lfoot{} \rfoot{\bf \thepage} \cfoot{}


\begin{document}

\title{Failure Case for the Smolyak Sparse Quadrature for gPC Expansion}

\author[]{Paul Talbot\thanks{talbotp@unm.edu}}
\date{}
\renewcommand\Authands{ and }
\maketitle
\section{Introduction}
Let $(\Omega,\mathcal{F},\rho)$ be a complete $N$-variate probability space.
We consider the algorithms for expanding a quantity of interest $u(Y)$ as a function of uncertain independent
input parameters $Y = (y_1,\cdots,y_n,\cdots,y_N)$ in a generalized polynomial chaos expansion using
orthonormal Gaussian polynomials $\phi_i^{(n)}(y_n)$.  These Gaussian polynomials are orthonormal with respect
to their corresponding individual monovariate probability space $(\Omega_n,\mathcal{F}_n,\rho_n)$.  The
expansion is given by
\begin{equation}\label{eq:gpc}
  u(Y)\approx \tilde u(Y) = \sum_{k\in\Lambda} c_k \Phi_k(Y),
\end{equation}
where $k$ is a multivariate index, $\Lambda$ is a set of $N$-variate indices corresponding to polynomial
orders, and $\Phi_k$ are a set of orthonormal multidimensional polynomials given by
\begin{equation}
  \Phi_k(Y) = \prod_{n=1}^N \phi_{k_n}(y_n).
\end{equation}
We assume $\Lambda$ can be constructed adaptively.  The admissability condition for new indices $k$ into
$\Lambda$ is
\begin{equation}
  k-e_j \in \Lambda \forall 1\leq j\leq N\,
\end{equation}
where $e_j$ is a unit vector in the direction of $j$.

The scalar coefficients $c_k$ in Eq. \ref{eq:gpc} can be obtained via the orthonormality of $\Phi_k$ as
\begin{equation}\label{eq:coeffs}
  c_k = \int_\Omega \rho(Y) u(Y) \Phi_k(Y) dY \equiv \mathcal{I}\big(u\cdot\Phi_k\big).
\end{equation}
We approximate the integral using Smolyak-like sparse quadrature. Using the notation for a single-dimension
quadrature operation
\begin{equation}
  \int \rho(x)f(x)dx = \mathcal{I}(f) \approx \sum_{\ell=1}^L w_\ell f(x_\ell) \equiv q^{L}(f),
\end{equation}
the sparse quadrature is given by
\begin{equation}
  \mathcal{I}\big(u\cdot\Phi\big)\approx\mathcal{S}[u\cdot\Phi]\equiv \sum_{\hat k\in\Lambda} s_{\hat k} \bigotimes_{n=1}^N
  q^{L_n}\big(u(Y)\Phi_k(Y)\big).
\end{equation}
The quadrature coefficient $s_{\hat k}$ is given by
\begin{equation}
  s_{\hat k} = \sum_{j\in\{0,1\}^N,i+j\in\Lambda} (-1)^{|j|_1}, \hspace{10pt} |j|_1 = \sum_{n=1}^N j_n. 
\end{equation}

We demonstrate here that for a particular simple response $u(Y)$ and index set $\Lambda$, the Smolyak algorithm does not accurately
integrate all the quadrature coefficients.

\section{Case}
For demonstration, we consider the quantity of interest
\begin{equation}
  u(x,y) = x^2 y^2,
\end{equation}
with $x$ and $y$ uniformly distributed from -1 to 1.  In this case, we use orthonormalized Legendre
polynomials for the expansions polynoials $\phi$.

For expansion polynomials, we consider as an example the following polynomial set,
\begin{table}[H]
  \centering
  \begin{tabular}{c c c c c c c c c}
    (8,0) &       &       &      &       &       &       &       &      \\
    (7,0) &       &       &      &       &       &       &       &      \\
    (6,0) &       &       &      &       &       &       &       &      \\
    (5,0) &       &       &      &       &       &       &       &      \\
    (4,0) &       &       &      &       &       &       &       &      \\
    (3,0) & (3,1) &       &      &       &       &       &       &      \\
    (2,0) & (2,1) & (2,2) &      &       &       &       &       &      \\
    (1,0) & (1,1) & (1,2) &(1,3) &       &       &       &       &      \\
    (0,0) & (0,1) & (0,2) &(0,3) & (0,4) & (0,5) & (0,6) & (0,7) & (0,8) 
  \end{tabular}
\end{table}
Because it includes the index set point (2,2), we expect this expansion to exactly represent the original
quantity of interest when coefficients $c_k$ are calculated correctly.

\section{Analytic}
First, we demonstrate the correct, analytic performance of the gPC expansion.  The polynomial coefficients
$c_k$ are given by Eq. \ref{eq:coeffs}.  Each coefficient integrates to zero with the exception of the
following:
\begin{align}\label{eq:analytic coeffs}
  c_{(0,0)} &= \frac{1}{9},\\
  c_{(0,2)} = c_{(2,0)} &= \frac{2}{9\sqrt{5}},\\
  c_{(2,2)} &= \frac{4}{45}.
\end{align}
Reconstructing the original model from the expansion, as expected we recover the original model exactly.

\section{Smolyak}
In order to be sufficient in a general sense, we desire the Smolyak sparse quadrature algorithm to perform
as accurately as the analytic case for this polynomial quantity of interest case.  We begin by evaluating the
values of the quadrature coefficients $s_{\hat k}$.  These are all zero with the exception of the following:
\begin{align}\label{eq:smolyak coeffs}
  s_{(0,3)} = s_{(3,0)} &= -1, \\
  s_{(0,8)} = s_{(8,0)} &=  1, \\
  s_{(1,2)} = s_{(2,1)} &= -1, \\
  s_{(1,3)} = s_{(3,1)} &=  1, \\
  s_{(2,2)} = s_{(2,2)} &=  1.
\end{align}
Using the quadrature order rule $L=k_n+1$, we will need points and weights for Legendre quadrature orders 1,
2, 3, 4 and 9.  The points and weights are listed here for convenience.
\begin{table}[H]
  \centering
  \begin{tabular}{c c c}
    Quadrature Order & Points & Weights \\ \hline
    1 & 0 & 2 \\ \hline
    2 & $\pm$ 0.5773502691896257 & 1\\ \hline
    3 & $\pm$ 0.7745966692414834 & 0.5555555555555556 \\
      & 0                        & 0.8888888888888888 \\ \hline
    4 & $\pm$ 0.8611363115940526 & 0.3478548451374538 \\
      & $\pm$ 0.3399810435848563 & 0.6521451548625461 \\ \hline
    9 & $\pm$ 0.9681602395076261 & 0.0812743883615744 \\
      & $\pm$ 0.8360311073266358 & 0.1806481606948574 \\
      & $\pm$ 0.6133714327005904 & 0.2606106964029354 \\
      & $\pm$ 0.3242534234038089 & 0.3123470770400029 \\
      & 0                        & 0.3302393550012598
  \end{tabular}
\end{table}
There are nine distinct tensor quadratures necessary to construct the Smolyak-like quadrature set, four of
which are duplicated because of symmetry.  This results in the following Smolyak-like quadrature set:
\begin{table}[H]
  \centering
  \begin{tabular}{c c}
    Tensor & Points \\ \hline
    (1)$\cross$(4) = (4)$\cross$(1)  & (0, $\pm$ 0.8611363115940526)   \\
                                     & (0, $\pm$ 0.3399810435848563)  \\ \hline
    (1)$\cross$(9) = (9)$\cross$(1)  & (0, $\pm$  0.9681602395076261)  \\
                                     & (0, $\pm$ 0.8360311073266358)  \\
                                     & (0, $\pm$ 0.6133714327005904)  \\
                                     & (0, $\pm$ 0.3242534234038089)  \\
                                     & (0, 0)  \\ \hline
    (2)$\cross$(3) = (3)$\cross$(2)  & ($\pm$ 0.5773502691896257, $\pm$ 0.7745966692414834)  \\
                                     & ($\pm$ 0.5773502691896257,0)  \\
    (2)$\cross$(4) = (4)$\cross$(2)  & ($\pm$ 0.5773502691896257, $\pm$ 0.8611363115940526)  \\
                                     & ($\pm$ 0.5773502691896257, $\pm$ 0.3399810435848563)  \\
    (3)$\cross$(3) = (3)$\cross$(3)  & ($\pm$ 0.7745966692414834, $\pm$ 0.7745966692414834)  \\
                                     & ($\pm$ 0, $\pm$ 0.7745966692414834)  \\
                                     & ($\pm$ $\pm$ 0.7745966692414834, 0)  \\
                                     & (0, 0)  \\
  \end{tabular}
\end{table}
The weights for each ordered set are the product of the weights for each individual point within the set.

We use this quadrature to evaluate Eq. \ref{eq:coeffs}.
We truncate the values to four digits in tables below, but retain machine precision throughout the calculations.  
In the calculation tables below, the first two columns are the
ordered set $(x,y)$ that make up a realization of the input space.  The third and fourth columns are the quadrature
weights corresponding to each individual point in the ordered set.  The fifth column is the Smolyak quadrature
coefficient shown in Eq. \ref{eq:smolyak coeffs}.  The sixth column is the evaluation of the quantity of
interest for the provided realization of $x$ and $y$.  The seventh and eighth columns are the orthonormal Legendre
polynomials evaluated at their respective realization (quadrature points).  Finally, the last column is a product comprising
a single term in the quadrature summation.
\newpage
\subsection{$c_k$, $k=(2,2)$}
We first consider the coefficient for $k=(2,2)$ in Table \ref{tab:2 2}.  
\begin{longtable}{c c|c c|c|c|c c|c}
Points ($x$) & ($y$) & Weights $w_x$ & $w_y$ & $s_k$ & $u(x,y)$ & $\phi_{k_1}(x)$ & $\phi_{k_2}(y)$ &
     $u(x,y)\cdot\Phi_k(x,y)\cdot w_xw_y\cdot s_k$\\ \hline
0.0000 & -0.8611 & 2.0000 & 0.3479 & -1 & 0.0000 & -1.1180 & 1.3692 & 0.0000 \\
0.0000 & -0.3400 & 2.0000 & 0.6521 & -1 & 0.0000 & -1.1180 & -0.7303 & 0.0000 \\
0.0000 & 0.3400 & 2.0000 & 0.6521 & -1 & 0.0000 & -1.1180 & -0.7303 & 0.0000 \\
0.0000 & 0.8611 & 2.0000 & 0.3479 & -1 & 0.0000 & -1.1180 & 1.3692 & 0.0000 \\
-0.8611 & 0.0000 & 0.3479 & 2.0000 & -1 & 0.0000 & 1.3692 & -1.1180 & 0.0000 \\
-0.3400 & 0.0000 & 0.6521 & 2.0000 & -1 & 0.0000 & -0.7303 & -1.1180 & 0.0000 \\
0.3400 & 0.0000 & 0.6521 & 2.0000 & -1 & 0.0000 & -0.7303 & -1.1180 & 0.0000 \\
0.8611 & 0.0000 & 0.3479 & 2.0000 & -1 & 0.0000 & 1.3692 & -1.1180 & 0.0000 \\
0.0000 & -0.9682 & 2.0000 & 0.0813 & 1 & 0.0000 & -1.1180 & 2.0259 & 0.0000 \\
0.0000 & -0.8360 & 2.0000 & 0.1806 & 1 & 0.0000 & -1.1180 & 1.2263 & 0.0000 \\
0.0000 & -0.6134 & 2.0000 & 0.2606 & 1 & 0.0000 & -1.1180 & 0.1439 & 0.0000 \\
0.0000 & -0.3243 & 2.0000 & 0.3123 & 1 & 0.0000 & -1.1180 & -0.7654 & 0.0000 \\
0.0000 & 0.0000 & 2.0000 & 0.3302 & 1 & 0.0000 & -1.1180 & -1.1180 & 0.0000 \\
0.0000 & 0.3243 & 2.0000 & 0.3123 & 1 & 0.0000 & -1.1180 & -0.7654 & 0.0000 \\
0.0000 & 0.6134 & 2.0000 & 0.2606 & 1 & 0.0000 & -1.1180 & 0.1439 & 0.0000 \\
0.0000 & 0.8360 & 2.0000 & 0.1806 & 1 & 0.0000 & -1.1180 & 1.2263 & 0.0000 \\
0.0000 & 0.9682 & 2.0000 & 0.0813 & 1 & 0.0000 & -1.1180 & 2.0259 & 0.0000 \\
-0.9682 & 0.0000 & 0.0813 & 2.0000 & 1 & 0.0000 & 2.0259 & -1.1180 & 0.0000 \\
-0.8360 & 0.0000 & 0.1806 & 2.0000 & 1 & 0.0000 & 1.2263 & -1.1180 & 0.0000 \\
-0.6134 & 0.0000 & 0.2606 & 2.0000 & 1 & 0.0000 & 0.1439 & -1.1180 & 0.0000 \\
-0.3243 & 0.0000 & 0.3123 & 2.0000 & 1 & 0.0000 & -0.7654 & -1.1180 & 0.0000 \\
0.0000 & 0.0000 & 0.3302 & 2.0000 & 1 & 0.0000 & -1.1180 & -1.1180 & 0.0000 \\
0.3243 & 0.0000 & 0.3123 & 2.0000 & 1 & 0.0000 & -0.7654 & -1.1180 & 0.0000 \\
0.6134 & 0.0000 & 0.2606 & 2.0000 & 1 & 0.0000 & 0.1439 & -1.1180 & 0.0000 \\
0.8360 & 0.0000 & 0.1806 & 2.0000 & 1 & 0.0000 & 1.2263 & -1.1180 & 0.0000 \\
0.9682 & 0.0000 & 0.0813 & 2.0000 & 1 & 0.0000 & 2.0259 & -1.1180 & 0.0000 \\
-0.5774 & 0.7746 & 1.0000 & 0.5556 & -1 & 0.2000 & 0.0000 & 0.8944 & 0.0000 \\
-0.5774 & 0.0000 & 1.0000 & 0.8889 & -1 & 0.0000 & 0.0000 & -1.1180 & 0.0000 \\
-0.5774 & -0.7746 & 1.0000 & 0.5556 & -1 & 0.2000 & 0.0000 & 0.8944 & 0.0000 \\
0.5774 & 0.7746 & 1.0000 & 0.5556 & -1 & 0.2000 & 0.0000 & 0.8944 & 0.0000 \\
0.5774 & 0.0000 & 1.0000 & 0.8889 & -1 & 0.0000 & 0.0000 & -1.1180 & 0.0000 \\
0.5774 & -0.7746 & 1.0000 & 0.5556 & -1 & 0.2000 & 0.0000 & 0.8944 & 0.0000 \\
0.7746 & -0.5774 & 0.5556 & 1.0000 & -1 & 0.2000 & 0.8944 & 0.0000 & 0.0000 \\
0.0000 & -0.5774 & 0.8889 & 1.0000 & -1 & 0.0000 & -1.1180 & 0.0000 & 0.0000 \\
-0.7746 & -0.5774 & 0.5556 & 1.0000 & -1 & 0.2000 & 0.8944 & 0.0000 & 0.0000 \\
0.7746 & 0.5774 & 0.5556 & 1.0000 & -1 & 0.2000 & 0.8944 & 0.0000 & 0.0000 \\
0.0000 & 0.5774 & 0.8889 & 1.0000 & -1 & 0.0000 & -1.1180 & 0.0000 & 0.0000 \\
-0.7746 & 0.5774 & 0.5556 & 1.0000 & -1 & 0.2000 & 0.8944 & 0.0000 & 0.0000 \\
-0.5774 & -0.8611 & 1.0000 & 0.3479 & 1 & 0.2472 & 0.0000 & 1.3692 & 0.0000 \\
-0.5774 & -0.3400 & 1.0000 & 0.6521 & 1 & 0.0385 & 0.0000 & -0.7303 & 0.0000 \\
-0.5774 & 0.3400 & 1.0000 & 0.6521 & 1 & 0.0385 & 0.0000 & -0.7303 & 0.0000 \\
-0.5774 & 0.8611 & 1.0000 & 0.3479 & 1 & 0.2472 & 0.0000 & 1.3692 & 0.0000 \\
0.5774 & -0.8611 & 1.0000 & 0.3479 & 1 & 0.2472 & 0.0000 & 1.3692 & 0.0000 \\
0.5774 & -0.3400 & 1.0000 & 0.6521 & 1 & 0.0385 & 0.0000 & -0.7303 & 0.0000 \\
0.5774 & 0.3400 & 1.0000 & 0.6521 & 1 & 0.0385 & 0.0000 & -0.7303 & 0.0000 \\
0.5774 & 0.8611 & 1.0000 & 0.3479 & 1 & 0.2472 & 0.0000 & 1.3692 & 0.0000 \\
-0.8611 & -0.5774 & 0.3479 & 1.0000 & 1 & 0.2472 & 1.3692 & 0.0000 & 0.0000 \\
-0.3400 & -0.5774 & 0.6521 & 1.0000 & 1 & 0.0385 & -0.7303 & 0.0000 & 0.0000 \\
0.3400 & -0.5774 & 0.6521 & 1.0000 & 1 & 0.0385 & -0.7303 & 0.0000 & 0.0000 \\
0.8611 & -0.5774 & 0.3479 & 1.0000 & 1 & 0.2472 & 1.3692 & 0.0000 & 0.0000 \\
-0.8611 & 0.5774 & 0.3479 & 1.0000 & 1 & 0.2472 & 1.3692 & 0.0000 & 0.0000 \\
-0.3400 & 0.5774 & 0.6521 & 1.0000 & 1 & 0.0385 & -0.7303 & 0.0000 & 0.0000 \\
0.3400 & 0.5774 & 0.6521 & 1.0000 & 1 & 0.0385 & -0.7303 & 0.0000 & 0.0000 \\
0.8611 & 0.5774 & 0.3479 & 1.0000 & 1 & 0.2472 & 1.3692 & 0.0000 & 0.0000 \\
-0.7746 & -0.7746 & 0.5556 & 0.5556 & 1 & 0.3600 & 0.8944 & 0.8944 & 0.0889 \\
-0.7746 & 0.0000 & 0.8889 & 0.8889 & 1 & 0.0000 & 0.8944 & -1.1180 & 0.0000 \\
-0.7746 & 0.7746 & 0.5556 & 0.5556 & 1 & 0.3600 & 0.8944 & 0.8944 & 0.0889 \\
0.0000 & -0.7746 & 0.5556 & 0.5556 & 1 & 0.0000 & -1.1180 & 0.8944 & 0.0000 \\
0.0000 & 0.0000 & 0.8889 & 0.8889 & 1 & 0.0000 & -1.1180 & -1.1180 & 0.0000 \\
0.0000 & 0.7746 & 0.5556 & 0.5556 & 1 & 0.0000 & -1.1180 & 0.8944 & 0.0000 \\
0.7746 & -0.7746 & 0.5556 & 0.5556 & 1 & 0.3600 & 0.8944 & 0.8944 & 0.0889 \\
0.7746 & 0.0000 & 0.8889 & 0.8889 & 1 & 0.0000 & 0.8944 & -1.1180 & 0.0000 \\
0.7746 & 0.7746 & 0.5556 & 0.5556 & 1 & 0.3600 & 0.8944 & 0.8944 & 0.0889 \\
\caption{Numeric Integration of $c_k$, $k=(2,2)$}
\label{tab:2 2}
\end{longtable}
Dividing the sum of the last column by $2^N=4$ yields a machine-precision accurate result for the analytic value
of $c_{(2,2)}$ shown in Eq. \ref{eq:analytic coeffs}, 0.0888888888888883.  This demonstrates that the Smolyak
quadrature is suitable for performing the numerical integral.


\newpage
\subsection{$c_k$, $k=(6,0)$}
We now consider the coefficient for $k=(6,0)$ in Table \ref{tab:6 0}.  
\begin{longtable}{c c|c c|c|c|c c|c}
Points ($x$) & ($y$) & Weights $w_x$ & $w_y$ & $s_k$ & $u(x,y)$ & $\phi_{k_1}(x)$ & $\phi_{k_2}(y)$ &
     $u(x,y)\cdot\Phi_k(x,y)\cdot w_xw_y\cdot s_k$\\ \hline
-0.8611 & 0.0000 & 0.3479 & 2.0000 & -1.0000 & 0.0000 & -1.3878 & 1.0000 & 0.0000 \\
-0.3400 & 0.0000 & 0.6521 & 2.0000 & -1.0000 & 0.0000 & 0.7402 & 1.0000 & 0.0000 \\
0.3400 & 0.0000 & 0.6521 & 2.0000 & -1.0000 & 0.0000 & 0.7402 & 1.0000 & 0.0000 \\
0.8611 & 0.0000 & 0.3479 & 2.0000 & -1.0000 & 0.0000 & -1.3878 & 1.0000 & 0.0000 \\
0.0000 & -0.9682 & 2.0000 & 0.0813 & 1.0000 & 0.0000 & -1.1267 & 1.0000 & 0.0000 \\
0.0000 & -0.8360 & 2.0000 & 0.1806 & 1.0000 & 0.0000 & -1.1267 & 1.0000 & 0.0000 \\
0.0000 & -0.6134 & 2.0000 & 0.2606 & 1.0000 & 0.0000 & -1.1267 & 1.0000 & 0.0000 \\
0.0000 & -0.3243 & 2.0000 & 0.3123 & 1.0000 & 0.0000 & -1.1267 & 1.0000 & 0.0000 \\
0.0000 & 0.0000 & 2.0000 & 0.3302 & 1.0000 & 0.0000 & -1.1267 & 1.0000 & 0.0000 \\
0.0000 & 0.3243 & 2.0000 & 0.3123 & 1.0000 & 0.0000 & -1.1267 & 1.0000 & 0.0000 \\
0.0000 & 0.6134 & 2.0000 & 0.2606 & 1.0000 & 0.0000 & -1.1267 & 1.0000 & 0.0000 \\
0.0000 & 0.8360 & 2.0000 & 0.1806 & 1.0000 & 0.0000 & -1.1267 & 1.0000 & 0.0000 \\
0.0000 & 0.9682 & 2.0000 & 0.0813 & 1.0000 & 0.0000 & -1.1267 & 1.0000 & 0.0000 \\
-0.9682 & 0.0000 & 0.0813 & 2.0000 & 1.0000 & 0.0000 & 1.5548 & 1.0000 & 0.0000 \\
-0.8360 & 0.0000 & 0.1806 & 2.0000 & 1.0000 & 0.0000 & -1.4919 & 1.0000 & 0.0000 \\
-0.6134 & 0.0000 & 0.2606 & 2.0000 & 1.0000 & 0.0000 & 0.4999 & 1.0000 & 0.0000 \\
-0.3243 & 0.0000 & 0.3123 & 2.0000 & 1.0000 & 0.0000 & 0.6368 & 1.0000 & 0.0000 \\
0.0000 & 0.0000 & 0.3302 & 2.0000 & 1.0000 & 0.0000 & -1.1267 & 1.0000 & 0.0000 \\
0.3243 & 0.0000 & 0.3123 & 2.0000 & 1.0000 & 0.0000 & 0.6368 & 1.0000 & 0.0000 \\
0.6134 & 0.0000 & 0.2606 & 2.0000 & 1.0000 & 0.0000 & 0.4999 & 1.0000 & 0.0000 \\
0.8360 & 0.0000 & 0.1806 & 2.0000 & 1.0000 & 0.0000 & -1.4919 & 1.0000 & 0.0000 \\
0.9682 & 0.0000 & 0.0813 & 2.0000 & 1.0000 & 0.0000 & 1.5548 & 1.0000 & 0.0000 \\
-0.5774 & 0.7746 & 1.0000 & 0.5556 & -1.0000 & 0.2000 & 0.8012 & 1.0000 & -0.0890 \\
-0.5774 & 0.0000 & 1.0000 & 0.8889 & -1.0000 & 0.0000 & 0.8012 & 1.0000 & 0.0000 \\
-0.5774 & -0.7746 & 1.0000 & 0.5556 & -1.0000 & 0.2000 & 0.8012 & 1.0000 & -0.0890 \\
0.5774 & 0.7746 & 1.0000 & 0.5556 & -1.0000 & 0.2000 & 0.8012 & 1.0000 & -0.0890 \\
0.5774 & 0.0000 & 1.0000 & 0.8889 & -1.0000 & 0.0000 & 0.8012 & 1.0000 & 0.0000 \\
0.5774 & -0.7746 & 1.0000 & 0.5556 & -1.0000 & 0.2000 & 0.8012 & 1.0000 & -0.0890 \\
0.7746 & -0.5774 & 0.5556 & 1.0000 & -1.0000 & 0.2000 & -1.2403 & 1.0000 & 0.1378 \\
0.0000 & -0.5774 & 0.8889 & 1.0000 & -1.0000 & 0.0000 & -1.1267 & 1.0000 & 0.0000 \\
-0.7746 & -0.5774 & 0.5556 & 1.0000 & -1.0000 & 0.2000 & -1.2403 & 1.0000 & 0.1378 \\
0.7746 & 0.5774 & 0.5556 & 1.0000 & -1.0000 & 0.2000 & -1.2403 & 1.0000 & 0.1378 \\
0.0000 & 0.5774 & 0.8889 & 1.0000 & -1.0000 & 0.0000 & -1.1267 & 1.0000 & 0.0000 \\
-0.7746 & 0.5774 & 0.5556 & 1.0000 & -1.0000 & 0.2000 & -1.2403 & 1.0000 & 0.1378 \\
-0.5774 & -0.8611 & 1.0000 & 0.3479 & 1.0000 & 0.2472 & 0.8012 & 1.0000 & 0.0689 \\
-0.5774 & -0.3400 & 1.0000 & 0.6521 & 1.0000 & 0.0385 & 0.8012 & 1.0000 & 0.0201 \\
-0.5774 & 0.3400 & 1.0000 & 0.6521 & 1.0000 & 0.0385 & 0.8012 & 1.0000 & 0.0201 \\
-0.5774 & 0.8611 & 1.0000 & 0.3479 & 1.0000 & 0.2472 & 0.8012 & 1.0000 & 0.0689 \\
0.5774 & -0.8611 & 1.0000 & 0.3479 & 1.0000 & 0.2472 & 0.8012 & 1.0000 & 0.0689 \\
0.5774 & -0.3400 & 1.0000 & 0.6521 & 1.0000 & 0.0385 & 0.8012 & 1.0000 & 0.0201 \\
0.5774 & 0.3400 & 1.0000 & 0.6521 & 1.0000 & 0.0385 & 0.8012 & 1.0000 & 0.0201 \\
0.5774 & 0.8611 & 1.0000 & 0.3479 & 1.0000 & 0.2472 & 0.8012 & 1.0000 & 0.0689 \\
-0.8611 & -0.5774 & 0.3479 & 1.0000 & 1.0000 & 0.2472 & -1.3878 & 1.0000 & -0.1193 \\
-0.3400 & -0.5774 & 0.6521 & 1.0000 & 1.0000 & 0.0385 & 0.7402 & 1.0000 & 0.0186 \\
0.3400 & -0.5774 & 0.6521 & 1.0000 & 1.0000 & 0.0385 & 0.7402 & 1.0000 & 0.0186 \\
0.8611 & -0.5774 & 0.3479 & 1.0000 & 1.0000 & 0.2472 & -1.3878 & 1.0000 & -0.1193 \\
-0.8611 & 0.5774 & 0.3479 & 1.0000 & 1.0000 & 0.2472 & -1.3878 & 1.0000 & -0.1193 \\
-0.3400 & 0.5774 & 0.6521 & 1.0000 & 1.0000 & 0.0385 & 0.7402 & 1.0000 & 0.0186 \\
0.3400 & 0.5774 & 0.6521 & 1.0000 & 1.0000 & 0.0385 & 0.7402 & 1.0000 & 0.0186 \\
0.8611 & 0.5774 & 0.3479 & 1.0000 & 1.0000 & 0.2472 & -1.3878 & 1.0000 & -0.1193 \\
-0.7746 & -0.7746 & 0.5556 & 0.5556 & 1.0000 & 0.3600 & -1.2403 & 1.0000 & -0.1378 \\
-0.7746 & 0.0000 & 0.8889 & 0.8889 & 1.0000 & 0.0000 & -1.2403 & 1.0000 & 0.0000 \\
-0.7746 & 0.7746 & 0.5556 & 0.5556 & 1.0000 & 0.3600 & -1.2403 & 1.0000 & -0.1378 \\
0.0000 & -0.7746 & 0.5556 & 0.5556 & 1.0000 & 0.0000 & -1.1267 & 1.0000 & 0.0000 \\
0.0000 & 0.0000 & 0.8889 & 0.8889 & 1.0000 & 0.0000 & -1.1267 & 1.0000 & 0.0000 \\
0.0000 & 0.7746 & 0.5556 & 0.5556 & 1.0000 & 0.0000 & -1.1267 & 1.0000 & 0.0000 \\
0.7746 & -0.7746 & 0.5556 & 0.5556 & 1.0000 & 0.3600 & -1.2403 & 1.0000 & -0.1378 \\
0.7746 & 0.0000 & 0.8889 & 0.8889 & 1.0000 & 0.0000 & -1.2403 & 1.0000 & 0.0000 \\
0.7746 & 0.7746 & 0.5556 & 0.5556 & 1.0000 & 0.3600 & -1.2403 & 1.0000 & -0.1378 \\
\caption{Numeric Integration of $c_k$, $k=(6,0)$}
\label{tab:6 0}
\end{longtable}
Dividing the sum of the last column by $2^N=4$ yields the value -0.1007265118224860, but the analytic value
for this polynomial coefficient is zero.  This demonstrates that Smoyak sparse grid quadrature as outlined
above is not suitable for performing the numeric integral.

\section{Conclusion}
Having outlined an implementation of gPC expansion and Smolyak sparse quadrature algorithms, along with a
particular response function, we have demonstrated cases where the Smolyak quadrature both does and does not
work well.  Because there are numerical integrals the Smolyak algorithm cannot perform well, it seems
unsuitable for general integration of polynomial expansion coefficients in arbitrary polynomial sets for gPC
expansions.\\

It is my hope that I have erred at some point, and that the Smolyak algorithm is more suitable than shown
here.
\end{document}

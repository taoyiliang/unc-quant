%%%%%%%%%%%%%%%%%%%%%%%%%%%%%%%%%%%%%%%%%
% Masters/Doctoral Thesis 
% LaTeX Template
% Version 1.43 (17/5/14)
%
% This template has been downloaded from:
% http://www.LaTeXTemplates.com
%
% Original authors:
% Steven Gunn 
% http://users.ecs.soton.ac.uk/srg/softwaretools/document/templates/
% and
% Sunil Patel
% http://www.sunilpatel.co.uk/thesis-template/
%
% License:
% CC BY-NC-SA 3.0 (http://creativecommons.org/licenses/by-nc-sa/3.0/)
%
% Note:
% Make sure to edit document variables in the Thesis.cls file
%
%%%%%%%%%%%%%%%%%%%%%%%%%%%%%%%%%%%%%%%%%

%----------------------------------------------------------------------------------------
%	PACKAGES AND OTHER DOCUMENT CONFIGURATIONS
%----------------------------------------------------------------------------------------

\documentclass[11pt, oneside]{Thesis} % The default font size and one-sided printing (no margin offsets)

\graphicspath{{../graphics/}} % Specifies the directory where pictures are stored

\usepackage[square, numbers, comma, sort&compress]{natbib} % Use the natbib reference package - read up on this to edit the reference style; if you want text (e.g. Smith et al., 2012) for the in-text references (instead of numbers), remove 'numbers' 

%from my usual papers
\usepackage{amsmath}
\usepackage{amssymb}
\usepackage{amsthm}
\usepackage{amscd}
\usepackage{amsfonts}
\usepackage{graphicx}%
%\usepackage{fancyhdr}
%\usepackage{color}
%\usepackage{cite}
\usepackage{physics}
\usepackage{float}
\usepackage{caption}
\usepackage{subcaption}

\hypersetup{urlcolor=blue, colorlinks=true} % Colors hyperlinks in blue - change to black if annoying
\title{\ttitle} % Defines the thesis title - don't touch this

\begin{document}

\frontmatter % Use roman page numbering style (i, ii, iii, iv...) for the pre-content pages

\setstretch{1.3} % Line spacing of 1.3

% Define the page headers using the FancyHdr package and set up for one-sided printing
\fancyhead{} % Clears all page headers and footers
\rhead{\thepage} % Sets the right side header to show the page number
\lhead{} % Clears the left side page header

\pagestyle{fancy} % Finally, use the "fancy" page style to implement the FancyHdr headers

\newcommand{\HRule}{\rule{\linewidth}{0.5mm}} % New command to make the lines in the title page

%code handling
\newcommand{\raven}{\texttt{raven}}
\newcommand{\bison}{\texttt{bison}}
\newcommand{\moose}{\texttt{moose}}
\newcommand{\rattlesnake}{\texttt{rattleSNake}}

% some handy math stuff
\newcommand{\expv}[1]{\ensuremath{\mathbb{E}[ #1]}}
\newcommand{\xs}[2]{\ensuremath{\Sigma_{#1}^{(#2)}}}
\newcommand{\intz}{\ensuremath{\int\limits_0^1}}
\newcommand{\intf}{\ensuremath{\int\limits_{-\infty}^\infty}}
\newcommand{\intzf}{\ensuremath{\int\limits_{0}^\infty}}
\newcommand{\LargerCdot}{\raisebox{-0.25ex}{\scalebox{1.2}{$\cdot$}}}
\newcommand{\hold}[1]{\ensuremath{\Big|_{#1}}}

\renewcommand{\vec}[1]{\mathbf{#1}}

% PDF meta-data
\hypersetup{pdftitle={\ttitle}}
\hypersetup{pdfsubject=\subjectname}
\hypersetup{pdfauthor=\authornames}
\hypersetup{pdfkeywords=\keywordnames}

%----------------------------------------------------------------------------------------
%	TITLE PAGE
%----------------------------------------------------------------------------------------

\begin{titlepage}
\begin{center}

\textsc{\LARGE \univname}\\[1.5cm] % University name
\textsc{\Large Doctoral Thesis}\\[0.5cm] % Thesis type

\HRule \\[0.4cm] % Horizontal line
{\huge \bfseries \ttitle}\\[0.4cm] % Thesis title
\HRule \\[1.5cm] % Horizontal line
 
\begin{minipage}{0.4\textwidth}
\begin{flushleft} \large
\emph{Author:}\\
%\href{http://www.johnsmith.com}
{\authornames} % Author name - remove the \href bracket to remove the link
\end{flushleft}
\end{minipage}
\begin{minipage}{0.4\textwidth}
\begin{flushright} \large
\emph{Supervisor:} \\
%\href{http://www.jamessmith.com}
{\supname} % Supervisor name - remove the \href bracket to remove the link  
\end{flushright}
\end{minipage}\\[3cm]
 
\large \textit{Submitted in partial fulfillment of the requirements\\ for the degree of \degreename}\\[0.3cm] % University requirement text
\textit{in the}\\[0.4cm]
%\groupname\\
\deptname\\[2cm] % Research group name and department name
 
{\large \today}\\[4cm] % Date
%\includegraphics{Logo} % University/department logo - uncomment to place it
 
\vfill
\end{center}

\end{titlepage}

%----------------------------------------------------------------------------------------
%	DECLARATION PAGE
%	Your institution may give you a different text to place here
%----------------------------------------------------------------------------------------

%\Declaration{
%
%\addtocontents{toc}{\vspace{1em}} % Add a gap in the Contents, for aesthetics
%
%I, \authornames, declare that this thesis titled, '\ttitle' and the work presented in it are my own. I confirm that:
%
%\begin{itemize} 
%\item[\tiny{$\blacksquare$}] This work was done wholly or mainly while in candidature for a research degree at this University.
%\item[\tiny{$\blacksquare$}] Where any part of this thesis has previously been submitted for a degree or any other qualification at this University or any other institution, this has been clearly stated.
%\item[\tiny{$\blacksquare$}] Where I have consulted the published work of others, this is always clearly attributed.
%\item[\tiny{$\blacksquare$}] Where I have quoted from the work of others, the source is always given. With the exception of such  quotations, this thesis is entirely my own work.
%\item[\tiny{$\blacksquare$}] I have acknowledged all main sources of help.
%\item[\tiny{$\blacksquare$}] Where the thesis is based on work done by myself jointly with others, I have made clear exactly what was done by others and what I have contributed myself.\\
%\end{itemize}
% 
%Signed:\\
%\rule[1em]{25em}{0.5pt} % This prints a line for the signature
% 
%Date:\\
%\rule[1em]{25em}{0.5pt} % This prints a line to write the date
%}

\clearpage % Start a new page

%----------------------------------------------------------------------------------------
%	QUOTATION PAGE
%----------------------------------------------------------------------------------------

\pagestyle{empty} % No headers or footers for the following pages

%\null\vfill % Add some space to move the quote down the page a bit
%
%\textit{``Thanks to my solid academic training, today I can write hundreds of words on virtually any topic without possessing a shred of information, which is how I got a good job in journalism."}
%
%\begin{flushright}
%Dave Barry
%\end{flushright}
%
%\vfill\vfill\vfill\vfill\vfill\vfill\null % Add some space at the bottom to position the quote just right
%
%\clearpage % Start a new page

%----------------------------------------------------------------------------------------
%	ABSTRACT PAGE
%----------------------------------------------------------------------------------------

\addtotoc{Abstract} % Add the "Abstract" page entry to the Contents

\abstract{\addtocontents{toc}{\vspace{1em}} % Add a gap in the Contents, for aesthetics

As experiment complexity in fields such as nuclear engineering continues to increase, so does the
demand for robust computational methods to simulate them.  In many 
simulations, input design parameters as well as intrinsic experiment properties are sources
of input uncertainty.  Often, small perturbations in uncertain parameters have significant impact on the
experiment outcome.  For instance, when considering nuclear fuel performance, small changes
in the fuel thermal conductivity can greatly affect the maximum stress on the surrounding cladding.
The difficulty of quantifying input uncertainty impact in such systems has grown with the
complexity of the numerical models.  Traditionally, uncertainty quantification has been approached using
random sampling methods like Monte Carlo.
For some models, the input parametric space and corresponding
quantity-of-interest output space is sufficiently explored with a few low-cost computational calculations.
For other models, it is computationally costly to obtain a
good understanding of the output space.  

To combat the costliness of random sampling, 
this research explores the possibilities of advanced methods in
stochastic collocation for generalized polynomial chaos (SCgPC) as an alternative to traditional uncertainty
quantification techniques such as Monte Carlo (MC) and Latin Hypercube sampling (LHS) methods.  In this
proposal we explore the behavior of traditional SCgPC construction strategies, as well as
truncated polynomial spaces using total degree (TD) and hyperbolic cross (HC) construction strategies.  We
also consider applying anisotropy to the polynomial space, and analyze methods whereby the level of
anisotropy can be approximated.  We review and develop potential adaptive polynomial construction
strategies.  Finally, we add high-dimension model reduction (HDMR) expansions, using SCgPC representations for
the constituent terms, and consider adaptive methods to construct them.
We analyze these methods on a series of models of increasing complexity.  We primarily use analytic methods of
various means, and finally demonstrate on an engineering-scale neutron transport problem.
For this analysis, we demonstrate the application of the
algorithms discussed above in \raven, a production-level uncertainty quantification framework.

Finally, we propose additional work in enhancing the current implementations of SCgPC and HDMR.
}

\clearpage % Start a new page

%----------------------------------------------------------------------------------------
%	ACKNOWLEDGEMENTS
%----------------------------------------------------------------------------------------

%\setstretch{1.3} % Reset the line-spacing to 1.3 for body text (if it has changed)
%
%\acknowledgements{\addtocontents{toc}{\vspace{1em}} % Add a gap in the Contents, for aesthetics
%
%The acknowledgements and the people to thank go here, don't forget to include your project advisor\ldots
%}
%\clearpage % Start a new page

%----------------------------------------------------------------------------------------
%	LIST OF CONTENTS/FIGURES/TABLES PAGES
%----------------------------------------------------------------------------------------

\pagestyle{fancy} % The page style headers have been "empty" all this time, now use the "fancy" headers as defined before to bring them back

\lhead{\emph{Contents}} % Set the left side page header to "Contents"
\tableofcontents % Write out the Table of Contents

\lhead{\emph{List of Figures}} % Set the left side page header to "List of Figures"
\listoffigures % Write out the List of Figures

\lhead{\emph{List of Tables}} % Set the left side page header to "List of Tables"
\listoftables % Write out the List of Tables

%----------------------------------------------------------------------------------------
%	ABBREVIATIONS
%----------------------------------------------------------------------------------------

%\clearpage % Start a new page
%
%\setstretch{1.5} % Set the line spacing to 1.5, this makes the following tables easier to read
%
%\lhead{\emph{Abbreviations}} % Set the left side page header to "Abbreviations"
%\listofsymbols{ll} % Include a list of Abbreviations (a table of two columns)
%{
%\textbf{LAH} & \textbf{L}ist \textbf{A}bbreviations \textbf{H}ere \\
%%\textbf{Acronym} & \textbf{W}hat (it) \textbf{S}tands \textbf{F}or \\
%}
%
%%----------------------------------------------------------------------------------------
%%	PHYSICAL CONSTANTS/OTHER DEFINITIONS
%%----------------------------------------------------------------------------------------
%
%\clearpage % Start a new page
%
%\lhead{\emph{Physical Constants}} % Set the left side page header to "Physical Constants"
%
%\listofconstants{lrcl} % Include a list of Physical Constants (a four column table)
%{
%Speed of Light & $c$ & $=$ & $2.997\ 924\ 58\times10^{8}\ \mbox{ms}^{-\mbox{s}}$ (exact)\\
%% Constant Name & Symbol & = & Constant Value (with units) \\
%}
%
%%----------------------------------------------------------------------------------------
%%	SYMBOLS
%%----------------------------------------------------------------------------------------
%
%\clearpage % Start a new page
%
%\lhead{\emph{Symbols}} % Set the left side page header to "Symbols"
%
%\listofnomenclature{lll} % Include a list of Symbols (a three column table)
%{
%$a$ & distance & m \\
%$P$ & power & W (Js$^{-1}$) \\
%% Symbol & Name & Unit \\
%
%& & \\ % Gap to separate the Roman symbols from the Greek
%
%$\omega$ & angular frequency & rads$^{-1}$ \\
%% Symbol & Name & Unit \\
%}
%
%%----------------------------------------------------------------------------------------
%%	DEDICATION
%%----------------------------------------------------------------------------------------
%
%\setstretch{1.3} % Return the line spacing back to 1.3
%
%\pagestyle{empty} % Page style needs to be empty for this page
%
%\dedicatory{For/Dedicated to/To my\ldots} % Dedication text
%
%\addtocontents{toc}{\vspace{2em}} % Add a gap in the Contents, for aesthetics

%----------------------------------------------------------------------------------------
%	THESIS CONTENT - CHAPTERS
%----------------------------------------------------------------------------------------

\mainmatter % Begin numeric (1,2,3...) page numbering

\pagestyle{fancy} % Return the page headers back to the "fancy" style

% Include the chapters of the thesis as separate files from the Chapters folder
% Uncomment the lines as you write the chapters
\nocite{*}
% Chapter 1

\chapter{Introduction} % Main chapter title

\label{ch:intro} % For referencing the chapter elsewhere, use \ref{Chapter1} 

\lhead{1. \emph{Introduction}} % This is for the header on each page - perhaps a shortened title

%----------------------------------------------------------------------------------------

%\section{Welcome and Thank You}

%problem description
Fuels performance codes are numerical simulations intended to characterize the performance of a set of
materials in a particular geometry under a certain environment, over time.  The environmental considerations
might include temperature, neutron flux, external pressure, and similar factors.  In many cases, the
performance is quantified by considering the maximum stress undergone by cladding around the fuel as it
expands and makes contact.  By varying the construction materials and geometry of the fuel, its cladding, and
the gap between them, fuel can be designed for optimal performance without experiencing a rupture or similar
break.

%introduce uq for problem
There are a plethora of parameters that go into simulating fuel performance.  The fuel itself is made up of
many constituent materials with a variety of densities and structures, as well as behavior under irradiation.
The contents of the fuel-cladding gap determine how effectively heat can conduct out of the fuel and to the
cladding, then out to a moderator, and the thickness of this gap determines the amount of fuel expansion
allowed before contact is made and outward pressure begins increasing.  The material and geometry of the
cladding determine limits on stress and efficiency of heat transfer.  Any of the material properties in the
fuel, gap, or cladding, along with the environmental conditions, can be a source of uncertainty in determining
the maximum stress applied to the cladding.

%explain nature of uncertainty
There are two categories into which sources of uncertainty fall: aleatoric, or the statistical uncertainty inherent in a
system; and epistemic, or the systematic uncertainty due to imprecision in measurement or existence of
measurable unknowns.  While there are aleatoric uncertainties in fuel performance (such as the neutronics of
irradiated fuel), in this work we consider mostly epistemic uncertainties surrounding the material properties
and geometries of the problems.  For an example case, we can consider the overall reactor power, fuel mesoscale
grain growth, and fuel thermal expansion coefficient as uncertain input parameters, with maximum Von Mises stress in the 
axial center of a fuel rod as a quantity of interest in the output space.

%explain scope of proposal
In this work, we consider several methodologies for quantifying the uncertainty in fuel performance
calculations.  In order to demonstrate clearly the function of these methods, we demonstrate them first on
several simpler problems, such as polynomial evaluations or projectile motion.  The first method we consider
is traditional, analog Monte Carlo (MC) analysis, wherein random sampling of the input space generates a view of
the output space.  MC is used as a benchmark methodology; if other methods converge on quantities of interest
more quickly and accurately than MC, we consider them ``better'' for our purposes.

The second method we consider is isotropic, tensor-product (TP) stochastic collocation for generalized polynomial
chaos (SCgPC)\cite{sparseSC}\cite{sparse1}\cite{sparse2}\cite{xiu}, whereby deterministic collocation points are used to develop a polynomial reduced-order model
of the output quantities of interest as a function of the inputs.  The other methods we consider expand on
this method.  First, we introduce non-tensor-product methods for determining polynomial bases, using the 
total degree (TD) and hyperbolic cross (HC) polynomial set construction methods\cite{hctd}.
These bases will then be used to construct Smolyak-like sparse grids for collocation\cite{smolyak}.  Second, we consider
anisotropic sparse grids,
allowing additional collocation points for preferential input parameters.  We also consider methods for
determining weights that determine the level of preference to give parameters, and explore the effects of a
variety of anisotropic choices.

The third method we consider is high-dimension model representation (HDMR), which correlates with Sobol
decomposition \cite{hdmr}.  This method is useful both for developing sensitivities of the quantity of interest to subsets
of the input space, as well as constructing a reduced-order model itself.  We demonstrate the strength of HDMR
as a method to inform anisotropic sensitivity weights for SCgPC.

Additionally, we propose continued work on developing adaptive algorithms for both SCgPC and HDMR\cite{Ayres}.  In adaptive
SCgPC, the polynomial basis is constructed level-by-level based on the highest-impact subset polynomials.  In
adaptive HDMR, the constituent subset input spaces are developed similarly, based on the highest-impact input
subset.  The crowning achievement we propose is combining HDMR and SCgPC to develop both the subset input
space as well as the overall reduced-order model adaptively in an attempt to construct a
competitively-efficient method for uncertainty quantification.

Finally, we propose all these methods be developed within Idaho National Laboratory's \raven{}\cite{raven}
uncertainty quantification framework. \raven{} is a Python-written framework that non-intrusively provides
tools for analysts to quantify the uncertainty in their simulations with minimal impact. \raven{}  has already
been shown to work seamlessly with \moose{}-based fuel performance code \bison{}\cite{moose}\cite{bison}, on which we propose to demonstrate the various
methods described above.

%outline chapters
The remainder of this work will proceed as follows:
\begin{itemize}
  \item Chapter 2: We mathematically describe the problems solved by the simulations we will be running,
    including polynomial evaluations, attenuation, projectile, diffusion, and fuel performance.  We discuss
    their potential applications and approach using random sampling.
  \item Chapter 3: We describe several methods for uncertainty quantification, including Monte Carlo, Latin
    Hypercube sampling, generalized Polynomial Chaos, and high-dimension model representation.  We also
    discuss methods to accelerate the convergence of SCgPC and HDMR models and.
  \item Chapter 4: We discuss proposed work extending both SCgPC and HDMR to be constructed adaptively.  We
    also discuss the predicted shortfalls in the adaptive algorithms and some potential methods to address
    them.
\end{itemize}
%----------------------------------------------------------------------------------------
      % 1
% Chapter Template

\chapter{Models} % Main chapter title

\label{Chapter2} % Change X to a consecutive number; for referencing this chapter elsewhere, use \ref{ChapterX}

\lhead{Chapter 2. \emph{Models}} % Change X to a consecutive number; this is for the header on each page - perhaps a shortened title

%----------------------------------------------------------------------------------------
%	SECTION: INTRO
%----------------------------------------------------------------------------------------

\section{Introduction to Models}
In this section we present the models used in demonstrating the uncertainty quantification (UQ) methods in this
work.  We use the term \emph{model} to describe any code with uncertain inputs and a set of at least one
response quantity of interest.
We include a variety of models with the intent to demonstrate the strengths and weaknesses of each UQ
method.

Firstly, we include several analytic models.  These are models who have an exact, derivable value for
statistical moments or sensitivities.  They are simpler in mechanics than full engineering-scale problems, and
offer a way to benchmark the performance of the UQ methods.

Secondly, we include an engineering-scale multiphysics application.  There are no analytic response values in
this model, only a nominal case with no uncertainty included in the input space.  Demonstration of the
performance of UQ methods on this model will highlight the practical application of each method.

We describe each model in turn.  Throughout the models, we will describe them using the syntax
\begin{equation}
  u(Y) = Q,
\end{equation}
where $u(Y)$ is the model as a function of $N$ uncertain inputs $Y=(y_1,\ldots,y_N)$ and $Q$ is a
single-valued response quantity of interest.


%----------------------------------------------------------------------------------------
%	SECTION: TENSOR POLY
%----------------------------------------------------------------------------------------

\section{Tensor Monomials}\label{mod:first tensor poly}
The simplest model we make use of is a first-order tensor polynomial (tensor monomial) combination \ref{Ayres}.
The mathematical expression is
\begin{equation}
  u(Y) = \prod_{n=1}^N (y_n+1).
\end{equation}
For example, for $N=3$ we have
\begin{equation}
  u(Y) = y_1y_2y_3 + y_1y_2 + y_1y_3 + y_2y_3 + y_1 + y_2 + y_3 + 1.
\end{equation}
For this model we distribute the uncertain inputs in several ways: uniformly on [-1,1], uniformly on
[0,1], and normally on [$\mu,\sigma$]. A summary of analytic statistics is given in Table \ref{tab:tensormono moments}.
%TODO derivations in appendix?

\begin{table}[H]
  \centering
  \begin{tabular}{c|c|c}
    Distribution & Mean & Variance \\\hline
    $\mathcal{U}[-1,1]$ & 1 & $\qty(\frac{4}{3})^N - 1$ \\
    $\mathcal{U}[0,1]$ & $\qty(\frac{3}{4})^N$ & $\qty(\frac{7}{3})^N - \qty(\frac{3}{4})^{2N}$ \\
    $\mathcal{N}[\mu,\sigma]$ & $\prod_{n=1}^N (\mu_{y_n}+1)$ & $\prod_{n=1}^N[(\mu_{y_n}+1)^2+\sigma_{y_n}^2]
    - \prod_{n=1}^N (\mu_{y_n}+1)^2$
  \end{tabular}
  \caption{Analytic Expressions for Tensor Monomial Case}
  \label{tab:tensormono moments}
\end{table}


%----------------------------------------------------------------------------------------
%	SECTION: Sudret
%----------------------------------------------------------------------------------------
\section{Sudret Polynomial}\label{mod:sudret}
The polynomial used by Sudret in his work \cite{sudret} is another tensor-like polynomial, and is a test case traditionally used to
identify convergence on sensitivity parameters.  The mathematical expression is
\begin{equation}
  u(Y) = \frac{1}{2^N}\prod_{n=1}^N (3y_n^2+1).
\end{equation}
The variables are distributed uniformly on [0,1].  The statistical moments and sensitivities are given in
Table \ref{tab:sudret}, where $\mathcal{S}_n$ is the global Sobol sensitivity of $u(Y)$ to perturbations in
$y_n$.

\begin{table}[H]
  \centering
  \begin{tabular}{c c}
    Statistic & Expression \\\hline
    Mean & 1 \\
    Variance & $\qty(\frac{6}{5})^N - 1$ \\
    $\mathcal{S}_n$ & $\frac{5^{-n}}{(6/5)^N-1}$
  \end{tabular}
  \caption{Analytic Expressions for Sudret Case}
  \label{tab:sudret}
\end{table}


%----------------------------------------------------------------------------------------
%	SECTION: ATTENUATION
%----------------------------------------------------------------------------------------
\section{Attenuation}\label{mod:attenuation}
This model represents an idealized single-dimension system where an beam of particles impinges on a
purely-absorbing material with total scaled length of 1.  The response of interest is the fraction of
particles exiting the opposite side of the material.  The material is divided into $N$ segments, each of which
has a distinct uncertain absorption cross section $y_n$.  The solution takes the form
\begin{equation}
  u(Y) = \prod_{n=1}^N \exp(-y_n/N).
\end{equation}
Because negative cross sections have dubious physical meaning, we restrict the distribution cases to uniform
on [0,1] as well as normally-distributed on [$\mu,\sigma$].  A summary of analytic statistics is given in
Table \ref{tab:attenuation moments}.

\begin{table}[H]
  \centering
  \begin{tabular}{c|c|c}
    Distribution & Mean & Variance \\\hline
    $\mathcal{U}[0,1]$ & $\qty[N\qty(1-e^{-1/N})]^N$ & $\qty[\frac{N}{2}\qty(1-e^{-2/N})]^N -
                       \qty[N\qty(1-e^{-1/N})]^{2N}$ \\
    $\mathcal{N}[\mu,\sigma]$ & $\prod_{n=1}^N \exp\qty[\frac{\sigma_{y_n}^2}{2N^2}-\frac{\mu_{y_n}}{N}]$
    & $\prod_{n=1}^N \exp\qty[\frac{2\sigma_{y_n}^2}{N^2} - \frac{2\mu_{y_n}}{N}]$
  \end{tabular}
  \caption{Analytic Expressions for Attenuation Case}
  \label{tab:attenuation moments}
\end{table}

This model has some interesting properties to demonstrate performance of polynomial-based UQ methods.  First,
because the solution is a product of exponential functions, it cannot be exactly represented by a finite
number of polynomials.  Second, the Taylor development of the exponential function includes all increasing
polynomial orders.  The product of several exponential functions is effectively a tensor combination of
polynomials for each dimension.

%----------------------------------------------------------------------------------------
%	SECTION: Gaussian Peak
%----------------------------------------------------------------------------------------
\section{Gaussian Peak}\label{mod:gausspeak}
Similar to the attenuation model, the Gaussian peak \cite{sfugenz} instead uses square arguments to the
exponential function.  A tuning parameter $a$ can be used to change the peakedness of the
function.  Increased peakedness leads to more difficult polynomial representation.  
A location parameter $\mu$ can be used to change the location of the peak.
The mathematical expression is
\begin{equation}
  u(Y) = \exp\qty(-\sum_{n=1}^N a^2\qty(y_n-\mu)^2).
\end{equation}
We allow each $y_n$ to vary uniformly on [0,1].
A summary of analytic statistics is given in Table \ref{tab:gausspeak moments}.

\begin{table}[H]
  \centering
  \begin{tabular}{c|c}
    Statistic & Expression \\ \hline
    Mean & $\qty(\frac{\sqrt{\pi}}{2a}\qty(\erf(a\mu)+\erf(a-a\mu)))^N$ \\
    Variance & $\qty(\frac{\sqrt{\pi/2}}{2a}\qty(\erf(a\mu\sqrt{2})-\erf(a\sqrt{2}(1-\mu))))^N - \qty(\frac{\sqrt{\pi}}{2a}\qty(\erf(a\mu)+\erf(a-a\mu)))^{2N}$
  \end{tabular}
  \caption{Analytic Expressions for Gaussian Peak Case}
  \label{tab:gausspeak moments}
\end{table}
This case offers particular challenge because of its Taylor development, which only includes even powers of
the uncertain parameters.  This suggests added difficulty in successive representation, especially for an
adaptive algorithm.


%----------------------------------------------------------------------------------------
%	SECTION: Ishigami
%----------------------------------------------------------------------------------------
\section{Ishigami Function}\label{mod:ishigami}
The Ishigami function \cite{ishigami} is a commonly-used function in performing sensitivity analysis.  It is
given by
\begin{equation}
  u(Y) = \sin{y_1} + a\sin^2{y_2} + b y_3^4\sin(y_1).
\end{equation}
In our case, we will use $a=7$ and $b=0.1$ as in \cite{ishigami2}.  In particular interest for this model are
its strong nonlinearity and lack of independence for $y_3$, as it only appears in conjunction with $y_1$.  The
analytic statistics of interest for this model are in Table \ref{tab:ishigami moments}, where $D_n$ is the
partial variance contributed by $y_n$ and Sobol sensitivities $\mathcal{S}_n$ are obtained by dividing $D_n$
by the total variance.

\begin{table}[H]
  \centering
  \begin{tabular}{c|c|c}
  Statistic & Expression & Approx. Value \\\hline
  Mean & $\frac{7}{2}$ & 3.5 \\
  Variance & $\frac{a^2}{8} + \frac{b\pi^4}{5} + \frac{b^2\pi^8}{18} + \frac{1}{2}$ & 13.84459 \\
  $D_1$ & $\frac{b\pi^4}{5} + \frac{b^2\pi^8}{50} + \frac{1}{2} $ &  4.34589 \\
  $D_2$ & $\frac{a^2}{8}$ & 6.125 \\
  $D_{1,3}$ & $\frac{8b^2\pi^8}{225}$ & 3.3737 \\
  $D_3,D_{1,2},D_{2,3},D_{1,2,3}$ & 0 & 0
  \end{tabular}
  \caption{Analytic Expressions for Ishigami Case}
  \label{tab:ishigami moments}
\end{table}


%----------------------------------------------------------------------------------------
%	SECTION: Sobol G-Function
%----------------------------------------------------------------------------------------
\section{Sobol G-Function}\label{mod:gfunc}

%----------------------------------------------------------------------------------------
%	SECTION: Anisotropic
%----------------------------------------------------------------------------------------
\section{Anisotropic Polynomial}\label{mod:aniso}

%----------------------------------------------------------------------------------------
%	SECTION: Pin Cell
%----------------------------------------------------------------------------------------
\section{Pin Cell}\label{mod:pincell}
This model is a coupled multiphysics engineering-scale model.  It simulates fuel behavior through the depletion of fissile material
in a single two-dimensional slice of a fuel rod.  The problem domain contains the fuel, gap, clad, and
moderator, and represents a symmetric quarter pin.  The depletion steps are carried out through a year-long
burn cycle.
The coupled multiphysics are neutronics, handled by \rattlesnake{}, and fuel performance, handled by
\bison{}.  

The mesh is shown in Fig. \ref{fig:pincell mesh}.  TODO BETTER FIGURE.  The mesh contains 20 bands of fuel
blocks, the gap, the clad, and the moderator.  TODO Use colors to describe locations.  TODO dimensions.
\begin{figure}[htb]
  \centering
  \includegraphics[width=0.7\linewidth]{pincell_mesh_png}
\end{figure}

The neutronics is calculated using 8 energy groups and takes as uncertain inputs 671 material cross
sections, including fission, capture, scattering, and neutron multiplication factor  Each cross section is
perturbed by 10\% of its original value, distributed normally.  The scattering cross
sections for each material in each group are not perturbed individually; rather, a scattering scaling factor
for each group is determined, and the scattering cross sections are all scaled by that factor.  The fission
and capture cross sections are perturbed individually.  The critical output of the neutronics calculation is
power shapes for use in the fuel performance code, as well as the $k$-eigenvalue for the rod.

The fuel performance code models mechanics such as heat conduction in the fuel, clad, and gap, clad stresses, grain radius
growth, and fuel expansion through the depletion steps of the fuel. The code takes power shapes
from the neutronics code as input and produces several key characteristics, such as peak clad temperature, maximum
fuel centerline temperature, and maximum clad stress.

The uncertain input space is highly correlated, so a Karhunen-Loeve (KL) component analysis is performed as
the first step in a two-part reduction \cite{physor2016}. The
covariance matrix is obtained via cross section construction in \texttt{scale}\cite{scale} using random sampling.  
Table \ref{tab:pcarank} gives the first several eigenvalues in the KL expansion.  Surrogate (or
\emph{latent}) dimensions identified by the KL expansion will be used as input variables for demonstration of
the various UQ methods.

\begin{table}[H]
  \centering
  \begin{tabular}{c|c}
Index & Eigenvalue \\ \hline
1 & 0.974489839965 \\
2 & 0.0183147250746 \\
3 & 0.00271597405394 \\
4 & 0.00260939137165 \\
5 & 0.000486257522596 \\
6 & 0.000431957645049 \\
7 & 0.000253683187786 \\
8 & 0.000228044411204 \\
9 & 0.000124030638175 \\
10 & 7.14328494102e-05 \\
11 & 6.30833696364e-05 \\
12 & 3.87071149672e-05 \\
13 & 3.51066363934e-05 \\
14 & 2.48699823434e-05 \\
15 & 1.98915286765e-05 \\
16 & 1.35985387253e-05 \\
17 & 1.128896325e-05 \\
18 & 9.59426898684e-06 \\
19 & 8.11612567548e-06 \\
20 & 7.16508951777e-06 \\
21 & 6.53366817241e-06 \\
22 & 4.50006575957e-06 \\
23 & 4.19287192651e-06 \\
24 & 3.7671309151e-06 \\
25 & 2.61683536224e-06 \\
26 & 2.22099981728e-06 \\
27 & 1.6360971709e-06 \\
28 & 1.13245742809e-06 \\
29 & 9.92282537141e-07
\end{tabular}
\caption{KL Expansion Eigenvalues for Pin Cell Problem}
\label{tab:pcarank}
\end{table}
     % 2
% Chapter Template

\chapter{Methods} % Main chapter title

\label{Chapter3} % Change X to a consecutive number; for referencing this chapter elsewhere, use \ref{ChapterX}

\lhead{Chapter 3. \emph{Methods}} % Change X to a consecutive number; this is for the header on each page - perhaps a shortened title

%----------------------------------------------------------------------------------------
%	SECTION: INTRO
%----------------------------------------------------------------------------------------

\section{Todo}
todo.
    % 3
% Chapter Template

\chapter{Results, Static} % Main chapter title

\label{Chapter4} % Change X to a consecutive number; for referencing this chapter elsewhere, use \ref{ChapterX}

\lhead{Chapter 4. \emph{Results, Static}} % Change X to a consecutive number; this is for the header on each page - perhaps a shortened title

%----------------------------------------------------------------------------------------
%	SECTION: INTRO
%----------------------------------------------------------------------------------------

\section{Todo}
todo.
    % 4 
% Chapter Template

\chapter{Engineering Demonstration} % Main chapter title

\label{ch:mammoth} % Change X to a consecutive number; for referencing this chapter elsewhere, use \ref{ChapterX}

\lhead{Chapter 5. \emph{Engineering Demonstration}} % Change X to a consecutive number; this is for the header on each page - perhaps a shortened title

%----------------------------------------------------------------------------------------
%	SECTION: INTRO
%----------------------------------------------------------------------------------------

\section{Introduction}
While analytic models provide insight to the operation of stochastic collocation for generalized polynomial chaos and high-density
model reduction methods, we are chiefly interested in applying these methods to engineering applications that lead to
decision making in real-world activities.  To this end, we selected multiphysics simulation code \mammoth{}, a \moose{}-based
\emph{MultiApp} that couples neutronics code \rattlesnake{}, fuel performance code \bison{}, and thermal hydraulics code
\texttt{relap-7}.  \mammoth{} solves these three sets of physics nonlinearly using picard iterations to feed back values until
convergence is achieved.

\section{Problem}
The model we selected is a two-dimensional slice of a pressurized water reactor fuel pin, including the fuel, gap, clad, and
moderator.  The fuel is separated into 23 axial rings, each with the same material properties but independent input uncertainty.
The response value of interest is the neutronics multiplication factor $k$-effective.  TODO how many groups, etc.
The uncertain inputs are group-, burnup-, and temperature-dependent cross sections, as well as TODO three \bison{} inputs.
The macroscopic cross sections are calculated using \texttt{scale} (TODO cite) along with a covariance matrix which provides
the correlation between cross sections.  Karhunen-Loevre (TODO cite) is used to de-couple the resulting 671 inputs and
reduce them to 10 representative independent inputs.

The simulation only considers neutronics (\rattlesnake{}) and fuel performance (\bison{}), so the thermal hydraulics is neglected.
The feedback from \rattlesnake{} to \bison{} is the power shape, and the feedback from \bison{} to \rattlesnake{} is the
temperature, which in turn affects the cross sections.  For performance, the number of picard iterations between the separate apps
is limited to 6 per time step.  TODO time steps, other input parameters, input file in appendix?, commit of MAMMOTH used

\section{Limitations}
During the collection of data, it was discovered that the performance of \bison{} can fluctuate depending on
the way is is parallelized.  There were instances where \bison{} would fail to converge, but report an unconverged
temperature as a converged solution.  As a result, there is artifical numerical error that is difficult to track or
account for.  Regardless, we demonstrate the performance of various methods on this model, as this behavior indicates
true simulation behavior.

\section{Results}

todo.

%Notes:
%
%Run times
%
%For Picard 6 and SN (3 azimuthal, 3 polar Gauss Chebyshev):
%
%MPI 24: 11m 29.452s = 689.452 sec =  16546.848 single equivalent (0.424 efficient)
%MPI 12: 19m 41.703s = 1181.703 sec = 14180.436 single equivalent (0.495 efficient)
%MPI  6: 27m 50.373s = 1670.373 sec = 10022.238 single equivalent (0.701 efficient)
%MPI  1: 117m 0.756s = 7020.756 sec =  7020.756 single equivalent (1.000 efficient)
    % 5 
% Chapter Template

\chapter{Conclusions} % Main chapter title

\label{Chapter8} % Change X to a consecutive number; for referencing this chapter elsewhere, use \ref{ChapterX}

\lhead{Chapter 8. \emph{Conclusions}} % Change X to a consecutive number; this is for the header on each page - perhaps a shortened title

%----------------------------------------------------------------------------------------
%	SECTION: INTRO
%----------------------------------------------------------------------------------------

\section{Todo}
todo.
 % 6 
% Chapter Template

\chapter{Future Work} % Main chapter title

\label{Chapter9} % Change X to a consecutive number; for referencing this chapter elsewhere, use \ref{ChapterX}

\lhead{Chapter 9. \emph{Future Work}} % Change X to a consecutive number; this is for the header on each page - perhaps a shortened title

%----------------------------------------------------------------------------------------
%	SECTION: INTRO
%----------------------------------------------------------------------------------------

\section{Introduction}
The results in this work lead to several interesting areas of improvement.  We discuss some of these briefly
here.

\section{Impact Inertia in Adaptive Samplers}
One weakness demonstrated in the adaptive sampling techniques is the phenomenon of purely-even or purely-odd
polynomial representations.  This is seen clearly in the Ishigami \ref{mod:ishigami} and Gauss Peak
\ref{mod:gausspeak} models.  Consider the Taylor development of a sine function,
\begin{equation}\label{eq:sine}
  \sin x = x - \frac{x^3}{6} + \frac{x^5}{120} + \mathcal{O}(x^7),
\end{equation}
and for a square exponential,
\begin{equation}\label{eq:sine}
  e^{-x^2} = 1 - x^2 + \frac{x^4}{2} - \frac{x^6}{6} + \frac{x^8}{24} + \mathcal{O}(x^10).
\end{equation}
Unique to both of these functions is ``skipping'' certain polynomial orders (evens for sine, odds for square
exponential).

In a single-dimension example, the current impact estimation expression is
\begin{equation}
  \tilde{\eta_k}= \eta_{k-1}.
\end{equation}

Because adaptive sampling currently relies on the previous-order polynomial to estimate the
importance of the current polynomial, it can be misled into thinking there is no additional information to
gather if certain polynomials are not present in the expansion.

For example, if the adaptive sampler finds the impacts of a one-dimensional problem to be 0.4 for $x$, it will
try $x^2$.  If the model is an odd function, it will find the impact for $x^2$ is actually zero.  As a result,
it will be very unlikely to try $x^3$, despite the fact that $x^3$ has significant real impact.

One resolution to this method is to apply some sort of \emph{impact inertia} to the estimation of impact
values; that is, in addition to considering the previous polynomial impact when estimating current polynomial
impact, several previous polynomials might be considered.  This kind of inertia is likely problem-dependent in
its effectiveness, and would be best controlled through an optional user input.  Some research would be
required to determine what default level of inertia is recommended.  The new impact estimation expression
would be something like the following:

\begin{equation}
  \tilde{\eta_k}= \frac{1}{\alpha}\sum_{n=1}^k \frac{1}{g(n)}\eta_{k-n},
\end{equation}
where $\alpha$ is a balancing parameter and $g(n)$ is a penalty function that grows as $k-n$ increases.

\section{Cross-Communication in Adaptive HDMR}
Todo, basically if I want the next step in (x,y), that next step shouldn't be a polynomial containing 0 in
either x or y.
     % 7
%\input{Chapters/Chapter6} 

%----------------------------------------------------------------------------------------
%	THESIS CONTENT - APPENDICES
%----------------------------------------------------------------------------------------

%\addtocontents{toc}{\vspace{2em}} % Add a gap in the Contents, for aesthetics

%\appendix % Cue to tell LaTeX that the following 'chapters' are Appendices

% Include the appendices of the thesis as separate files from the Appendices folder
% Uncomment the lines as you write the Appendices

%\input{Appendices/AppendixA}
%\input{Appendices/AppendixB}
%\input{Appendices/AppendixC}

%\addtocontents{toc}{\vspace{2em}} % Add a gap in the Contents, for aesthetics

\backmatter

%----------------------------------------------------------------------------------------
%	BIBLIOGRAPHY
%----------------------------------------------------------------------------------------

\label{Bibliography}

\lhead{\emph{Bibliography}} % Change the page header to say "Bibliography"

\bibliographystyle{unsrtnat} % Use the "unsrtnat" BibTeX style for formatting the Bibliography

\bibliography{Bibliography} % The references (bibliography) information are stored in the file named "Bibliography.bib"

\end{document}

% Appendix Template

\chapter{Quadratures, Polynomials, and Distributions} % Main appendix title

\label{apx:quads dists} % Change X to a consecutive letter; for referencing this appendix elsewhere, use \ref{AppendixX}

\lhead{Appendix A. \emph{Quadratures, Polynomials, and Distributions}} % Change X to a consecutive letter; this is for the header on each page - perhaps a shortened title



%\newcommand{\intzf}{\ensuremath{\int\limits_{0}^\infty}}
%\newcommand{\LargerCdot}{\raisebox{-0.25ex}{\scalebox{1.2}{$\cdot$}}}

\section{Introduction}
Thanks to the work of Xiu and Karniadakis \cite{xiu}, many distributions have corresponding polynomials and
integrating quadratures that are ideally suited for generalized Polynomial Chaos (gPC) expansion construction.
In this appendix, we consider the four continuous distributions with corresponding ideal polynomials and
quadratures, as well as an arbitrary case for other distributions.  In general, domain transformations must be
performed to shape a general distribution into a form that matches the quadrature integration scheme; we
discuss those transformations for each distribution here as well.  For clarity, we define $x$ as the
variable for standard distributions and domains, while $y$ is the variable for generic distributions and
domains.  Transformation involves the relationship between $x$ and $y$.

\subsection{General Syntax}
We use the following syntax when describing polynomials, distributions, normalizations, and quadratures.
\begin{itemize}
  \item $x$: Argument of standard distribution with specific domain.
  \item $y$: Argument of general distribution with generic domain.
  \item $h(y)$: Generic function with dependence on distributed variable $y$.
\item $\rho(y)$: Probability measure, or the functional part of the probability distribution.
\item $A$: Normalization factor, or scalar part of the probability distributions.
\item $\mu$: Distribution mean.
\item $\sigma$: Distribution standard deviation.
\end{itemize}
Note that because of the definitions above, for each distribution we require
\begin{equation}
  A \int\limits_a^b \rho(y) dy = 1,
\end{equation}
where $a$ and $b$ are the extreme values of the distribution, and might be infinite.
We also make use of the two standard functions, the Gamma function (not to be confused with the Gamma
distribution),
\begin{equation}
  \Gamma(x) = \int\limits_0^\infty z^{x-1}\exp(-z) dz,
\end{equation}
and the Beta function (not to be confused with the Beta distribution),
\begin{equation}
  \text{B}(z_1,z_2) = \int\limits_0^1 t^{z_1-1}(1-t)^{z_2-1} dt.
\end{equation}
In each example the quadrature nodes and weights $x_i,w_i$ will use subscripts that relate to the symbol of the
associated polynomials, for increased clarity.

\section{Uniform Distributions and Legendre Polynomials}
The uniform distribution is a single value between finite extrema $a$ and $b$ and zero everywhere else so that
$x\in[a,b]$. The uniform distribution has the following characteristics.
\begin{equation}
  \mu = \frac{a+b}{2},
\end{equation}
\begin{equation}
  \sigma = \frac{b-a}{2}
\end{equation}
\begin{equation}
  A = \frac{1}{2\sigma},
\end{equation}
\begin{equation}
  \rho(y)=1,
\end{equation}
\begin{equation}
  1 = \frac{1}{2\sigma}\int_a^b dy.
\end{equation}
Legendre polynomials $P_n(x)$ are defined on the range $x\in[-1,1]$ with polynomial order $n\in\mathbb{N}$.  
They are defined by the contour integral \cite{polys}
\begin{equation}
  P_n(x) = \frac{1}{2\pi i}\oint\qty(1-2tx + t^2)^{-1/2}t^{-n-1} dt,
\end{equation}
and are made orthonormal as
\begin{equation}
  \frac{2n-1}{2}\int\limits_{-1}^1 P_m(x)P_n(x) dx = \delta_{mn}.
\end{equation}
Legendre quadrature approximates the following integrals:
\begin{equation}
 \int\limits_{-1}^1 h(x)d(x) = \sum_{\ell=1}^\infty w_\ell h(x_\ell)
\end{equation}
In order to convert a general uniform distribution to share the domain of Legendre polynomials and quadrature,
the following conversion is necessary:
\begin{equation}
  y = \sigma x+\mu,
\end{equation}
\begin{equation}
  x = \frac{y-\mu}{\sigma}.
\end{equation}
As a result, Legendre quadrature integrates arbitrary uniform distributions as
\begin{equation}
  \int_a^b h(y)\rho(y)dy =\frac{1}{2} \sum_{\ell=1}^\infty w_\ell h(\sigma x_\ell+\mu).
\end{equation}


\section{Normal Distribution and Hermite Polynomials}
The normal distribution is a symmetric, bell-shaped distribution with domain $y\in(-\infty,\infty)$.  The
normal distribution has the following characteristics.
\begin{equation}
  \rho(y) = \exp\qty(-\frac{(y-\mu)^2}{2\sigma^2}),
\end{equation}
\begin{equation}
  A = \frac{1}{\sigma\sqrt{2\pi}},
\end{equation}
\begin{equation}
  1 =\frac{1}{\sigma\sqrt{2\pi}} \int_{-\infty}^\infty \exp\qty( -\frac{(y-\mu)^2}{2\sigma^2} )dy.
\end{equation}
Hermite polynomials $\text{He}_n(x)$ are defined on the range $x\in(-\infty,\infty)$ with polynomial order
$n\in\mathbb{N}$.  They can be defined through the contour integral \cite{polys}
\begin{equation}
  \text{He}_n(x) = \frac{n!}{2\pi i}\oint \exp(-t^2+2tx)t^{-n-1} dt,
\end{equation}
and are made orthonormal as
\begin{equation}
  \frac{1}{\sqrt{2\pi} n!}\int\limits_{-\infty}^\infty \text{He}_m(x)\text{He}_n(x) \exp(\frac{-x^2}{2}) dx = \delta_{m,n}.
\end{equation}
Hermite quadrature approximates the following integrals:
\begin{equation}
  \int\limits_{-\infty}^\infty h(x)\exp(\frac{-x^2}{2})dy = \sum_{h=1}^\infty w_h h(x_h).
\end{equation}
In order to convert a general normal distribution to share the domain of Hermite polynomials and quadrature,
the following conversion is necessary:
\begin{equation}
  y = \sigma x+\mu,
\end{equation}
\begin{equation}
  x = \frac{y-\mu}{\sigma},
\end{equation}
As a result, Hermite quadrature integrates arbitrary normal distributions as
\begin{equation}
  \intf h(y)\rho(y)dy =\frac{1}{\sqrt{2\pi}} \sum_{h=1}^\infty w_h h(\sigma x_h+\mu).
\end{equation}



\section{Gamma Distribution and Laguerre Polynomials}
The Gamma distribution has a finite lower bound $a$ and infinite upper bound so that $y\in[a,\infty)$.
This distribution also takes as an argument shape factors $\alpha$ and $\beta$.
The Gamma distribution has the following characteristics:
\begin{equation}
  \rho(y) = y^{\alpha-1}e^{-\beta y},
\end{equation}
\begin{equation}
  A = \frac{\beta^\alpha}{\Gamma(\alpha)},
\end{equation}
\begin{equation}
  1 = \frac{\beta^\alpha}{\Gamma(\alpha)}\int_a^\infty (y-a)^{\alpha-1}e^{-\beta (y-a)} dy.
\end{equation}
Laguerre polynomials $\mathcal{L}_n(x)^{(\tilde\alpha)}$ are defined on the range $x\in[0,\infty]$ with polynomial order
$n\in\mathbb{N}$.  An additional argument $\tilde\alpha$ is required to specify the polynomial family.
Note that we use $\tilde\alpha$ for the polynomial parameter, and $\alpha$ for the distribution parameter.
Laguerre polynomials can be defined through the contour integral \cite{polys}
\begin{equation}
  \mathcal{L}_n^{(\tilde\alpha)}(x) = \frac{1}{2\pi i}\oint\frac{1}{(1-t)^{\tilde\alpha+1)}t^{n+1}} 
      \exp(-\frac{xt}{1-t}) dt,
\end{equation}
and are made orthonormal as
\begin{equation}
  \frac{n!}{\Gamma(n+\tilde\alpha+1)}\int\limits_0^\infty x^\alpha e^{-x}
          \mathcal{L}_m^{(\tilde\alpha)}\mathcal{L}_n^{(\tilde\alpha)} dx = \delta_{mn}.
\end{equation}
Laguerre quadrature approximates the following integrals:
\begin{equation}
  \int\limits_0^\infty h(x) x^{\tilde\alpha} e^{-x} dx = \sum_{g=1}^\infty w_g h(x_g).
\end{equation}
In order to convert a general Gamma distribution to share the domain of Laguerre polynomials and quadrature,
the following conversion is necessary:
\begin{equation}
  y = \frac{x}{\beta}+L,
\end{equation}
\begin{equation}
  x = (y-L)\beta.
\end{equation}
As a result,
\begin{equation}
  \int_{L}^\infty h(y)\rho(y)dy =\frac{1}{(\alpha-1)!} \sum_{g=1}^\infty w_g h\qty(\frac{x_g}{\beta}+L),
\end{equation}
where points and weights are obtained using $\tilde\alpha = \alpha-1$ for Laguerre polynomials and quadrature.



\section{Beta Distribution and Jacobi Polynomials}
The Beta distribution is a flexible distribution with finite range $y\in[a,b]$ and shaping parameters $\alpha$
and $\beta$.  In the event $\alpha=\beta$, the distribution is symmetric.  If $\alpha=\beta=0$, the uniform
distribution is recovered.  The Beta distribution has the
following characteristics:
\begin{equation}
  \rho(y) = y^{\alpha-1}(1-y)^{\beta-1},
\end{equation}
\begin{equation}
  A =\frac{ \Gamma(\alpha+\beta)}{\Gamma(\alpha)\Gamma(\beta)},
\end{equation}
\begin{equation}
  1 = \frac{ \Gamma(\alpha+\beta)}{\Gamma(\alpha)\Gamma(\beta)}\int_a^b y^{\alpha-1}(1-y)^{\beta-1} dy.
\end{equation}
Jacobi polynomials $J_n(x)^{(\tilde\alpha,\tilde\beta)}$ are defined on the range $[-1,1]$ with polynomial
order $n\in\mathbb{N}$.  Two shaping arguments $\tilde\alpha$ and $\tilde\beta$ are used to uniquely define the
polynomial family.
Note that we use $\tilde\alpha,\tilde\beta$ for the polynomial parameters, and $\alpha,\beta$ for the
distribution parameters.
Jacobi polynomials can be defined through solution of the recurrence relation as
\begin{equation}
  J_n^{(\tilde\alpha,\tilde\beta)}(x) = \frac{(-1)^n}{2^n n!}(1-x)^{-\tilde\alpha}(1+x)^{-\tilde\beta}
  \frac{d^n}{d x^n}\qty[(1-x)^{\tilde\alpha+n}(1+x)^{\tilde\beta+n}],
\end{equation}
where both $\tilde\alpha$ and $\tilde\beta$ are greater than -1.  Jacobi polynomials are made orthonormal as
\begin{equation}
  \xi(\tilde\alpha,\tilde\beta,n)\int\limits_{-1}^1(1-x)^{\tilde\alpha}(1+x)^{\tilde\beta}
    J_m^{(\tilde\alpha,\tilde\beta)}(x)J_n^{(\tilde\alpha,\tilde\beta)}(x)dx = \delta_{mn},
\end{equation}
where
\begin{equation}
  \xi(\tilde\alpha,\tilde\beta,n) = \frac{2n+\tilde\alpha+\tilde\beta+1}{2^{\tilde\alpha+\tilde\beta+1}}
      \frac{\Gamma(n+\tilde\alpha+1)\Gamma(n+\tilde\beta+1)}{\Gamma(n+\tilde\alpha+\tilde\beta+1)n!}.
\end{equation}
Jacobi quadrature approximates the following integrals:
\begin{equation}
  \int_{-1}^1 h(x)(1-x)^{\tilde\alpha} (1+x)^{\tilde\beta} dx = \sum_{j=1}^\infty w_j h(x_j)
\end{equation}
Transforming general Beta distributions to compatible domain with Jacobi polynomials and quadrature will be
done in two steps, first to standard Beta distribution (using variable argument $z$) and then to Jacobi quadrature domain.
To convert to standard Beta:
\begin{equation}
  z=\frac{y-a}{b-L},\hspace{10pt}y=(b-a)z+a,\hspace{10pt} dy=(b-a)dz,
\end{equation}
\begin{equation}
  1=\frac{1}{\text{B}(\alpha,\beta)}\int_0^1 z^{\alpha-1}(1-z)^{\beta-1}dz,
\end{equation}
To convert to same form as the Jacobi probability weight,
\begin{equation}
  z=\frac{1+x}{2},\hspace{10pt}x=2z-1,\hspace{10pt}dz=\frac{1}{2}dx,
\end{equation}
so that
\begin{equation}
  1=\frac{1}{2^{\alpha+\beta-1}\text{B}(\alpha,\beta)}\int_{-1}^1 (1+x)^{\alpha-1}(1-x)^{\beta-1} dx.
\end{equation}
Combining the transformations,
\begin{equation}
  y=\frac{b-a}{2}x+\frac{b+a}{2},\hspace{10pt}x=\qty(y-\frac{b+a}{2})\qty(\frac{2}{b-a})
\end{equation}
In a potentially confusing twist, the Jacobi polynomial characteristic $\tilde\alpha$ is related to the Beta
distribution parameter $\beta$, and the Jacobi polynomial characteristic $\tilde\beta$ is related to the Beta
distribution parameter $\alpha$, as
\begin{equation}
  \tilde\alpha = \beta-1,\hspace{15pt}\tilde\beta = \alpha-1.
\end{equation}
As a result,
\begin{equation}
  \int_a^b h(y)\rho(y)dy = \frac{1}{2^{\tilde\alpha+\tilde\beta-1}\text{B}(\tilde\alpha,\tilde\beta)}
      \sum_{B=1}^\infty w_B h\qty(\frac{b-a}{2}x_B+\frac{b+a}{2}).
\end{equation}




\section{Arbitrary Distributions and Legendre Polynomials}
In general there is not a family of polynomials and quadratures that corresponds nicely for every
probabilistic distribution.  However, many continuous distributions have a CDF and inverse CDF that map the
distribution to and from the domain $[0,1]$.  As a result, we can apply Legendre quadrature to the
converted space.  This does not perfectly preserve the integration properties of Gaussian quadrature, but does
allow for general distributions to be covered.  We require for arbitrary distributions that
$y$ is finite, or $-\infty < a \leq y \leq b < \infty$.  In this case, the distribution has the following properties:
\begin{equation}
  \rho(y) = \rho(y),
\end{equation}
\begin{equation}
  F(y) = \int_{a}^y \rho(y')dy',
\end{equation}
\begin{equation}
  1 = \int_{a}^b \rho(y)dy. 
\end{equation}
We now consider transforming to the domain $[0,1]$ using the CDF.
Let $u\in[0,1]$, and note $F(y)\in[0,1]$.
Let
\begin{equation}
  du = dF(y) = \rho(y)dy,
\end{equation}
then
\begin{equation}
  F(y)=u\hspace{10pt}\therefore\hspace{10pt}y=F^{-1}(u),
\end{equation}
\begin{equation}
  dy = \frac{1}{\rho(y)}du,
\end{equation}
\begin{align}
\int_a^b h(y)\rho(y)dy &= \int_0^1 h\qty(F^{-1}(u))\rho\qty(F^{-1}(u))\frac{1}{\rho\qty(F^{-1}(u))}du,\\
  &=\int_0^1 h\qty(F^{-1}(u))du.
\end{align}
\begin{equation}
x = \frac{u-\mu}{\sigma}\therefore u=\sigma x+\mu,
\end{equation}
\begin{equation}
u = \frac{\hat b-\hat a}{2}x+\frac{\hat b+\hat a}{2},\hspace{10pt}\hat b=1,\hat a=0,
\end{equation}
\begin{equation}
u = \frac{1}{2}(x+1),
\end{equation}
\begin{align}
\int_a^b h(y)\rho(y)dy&= \int_0^1 h\qty(F^{-1}(u))du,\\
  &=\frac{1}{2}\sum_{\ell=1}^\infty w_\ell h\qty(F^{-1}\qty(\frac{1}{2}(x_\ell+1)))du.
\end{align}
In this manner, arbitrary distributions with a continuous CDF and inverse CDF can be expanded using Legendre
polynomials and integrated using Legendre quadrature.  This work was presented in \cite{xiu}, and we have
added the transformation algorithms explicitly.

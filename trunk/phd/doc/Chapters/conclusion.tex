% Chapter Template

\chapter{Conclusions} % Main chapter title

\label{ch:concl} % Change X to a consecutive number; for referencing this chapter elsewhere, use \ref{ChapterX}

\lhead{Chapter 8. \emph{Conclusions}} % Change X to a consecutive number; this is for the header on each page - perhaps a shortened title

%----------------------------------------------------------------------------------------
%	SECTION: INTRO
%----------------------------------------------------------------------------------------

\section{Introduction}
In this work we have explored advanced uncertainty quantification methods and their application to a variety of models.
In addition to implementing existing algorithms, new predictive methods for adaptive algorithms have been introduced
and demonstrated.  We have compared convergence performance to traditional analog Monte Carlo, and observed cases both
when collocation-based methods are desirable and when Monte Carlo is preferable.  We have also demonstrated performance
of these methods on a multiphyics engineering problem, as well as in time-dependent analysis of and OECD benchmark.  Here we
summarize the obsesrved results and generalize them, as well as discuss limitations discovered during this work.

\section{Performance Determination}
As seen in several analytic models as well as engineering performance models, the convergence rate of both stochastic collocation
for generalized polynomial chaos as well as high-density model reduction (using stochastic collocation for subsets) depend primarily
on two factors. 
First, despite many tools to combat the curse of dimensionality, grid-based collocation methods still degrade
significantly as the size of the input space increases.  For any more than approximately 10 independent inputs, collocation-based
methods perform little better than Monte Carlo until many thousands of samples are taken.  Second, SCgPC and HDMR perform much
better for models with a high level of continuity than discontinuous models.  As seen in the Sobol G-Function, the collocation
methods have great difficulty representing the absolute value function.  
However, for models with high levels of continuity and low dimensionality, collocation methods prove very effective in comparison
to traditional Monte Carlo methods.

Between different collocation-based methods, we also see several trends.  First, static HDMR methods never outperform their
corresponding SCgPC methods; that is, second-order polynomial expansions in SCgPC always match or outperform HDMR methods
that are limited to second-order polynomials.  This is expected because HDMR at any truncation is a subset of the SCgPC
expansion.  However, the static HDMR method is still valuable, as even in larger input dimensionality problems some solution
can be obtained with few runs.  For example, for first-order HDMR using first-order polynomials, only three times the input
dimensionality samples are required to obtain a solution.  For very costly models, it may not be possible to use SCgPC without
HDMR.

Second, we observe for all continuous functions, the total degree polynomial construction method significantly outperforms
the hyperbolic cross method, as expected by its design.  Since polynomial expansion methods struggle to perform well for
discontinuous models anyway, total degree is a good method to use if the response is expected to be smooth.

Finally, we note that in general the adaptive methods seldom completely outperform all the other collocation methods.  Because the
prediction algorithm is imperfect, there will always be a static choice of polynomials that is more effecient.  However, if
the nature of the response in polynomial representation is not well-known, the adaptive algorithms can be effective tools in
exploring the response polynomial space.

\section{Limitations Discovered}
One limitation that has not yet been discussed is the reliablity of model algorithms.  Because many engineering codes are
complicated and involve a great number of opttions to assure particular realizations can be solved, they are also often
somewhat fragile.  Changes in the input space can require changes in other solution options, such as preconditioning tools,
spatial and temportal step sizes, and so on.  The changes required are often not predictable, and if not applied, can result
in regular failure to converge a solution.  Traditional Monte Carlo methods overcome this issue by rejecting failed points and
chosing a new sample.  This introduces some bias, but hopefully a small amount relative to the overall sample size.  For
collocation-based methods, however, re-sampling is not a valid option, and failure to converge results in a failure of the method.
In the process of searching for a suitable engineering demonstration model, many problems were considering using a variety of
codes; however, after extensive collaboration, it was determined many of these codes accepted as much as a 10\% failure rate
in random perturbations of the input space.  This failure rate almost surely renders the collocation-based methods
unusable.  Thus, in addition to considering the dimnsionality of the input space and regularity of the response, the
robustness of the algorithms used to solve the model responses must be considered before applying stochastic collocation for
generalized polynomial chaos expansion or high-density model reductions methods.

\section{Final Comments}
Needed?

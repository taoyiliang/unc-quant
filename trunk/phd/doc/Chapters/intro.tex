% Chapter 1

\chapter{Introduction} % Main chapter title

\label{ch:intro} % For referencing the chapter elsewhere, use \ref{Chapter1} 

\lhead{1. \emph{Introduction}} % This is for the header on each page - perhaps a shortened title

%----------------------------------------------------------------------------------------

%\section{Welcome and Thank You}

%problem description
In simulation modeling, we attempt to capture the behavior of a physical system by describing it in a series
of equations, often partial differential equations.  These equations may be time-dependent, and capture
physics of interest for understanding the system.  A \emph{solver} is then written that can solve the series
of equations and determine quantities of interest (QoI).  A traditional solver accepts a set of inputs and
produces a set of single-valued outputs.  For instance, a solver might solve equations related to the
attenuation of a beam of photons through a material, and the QoI might be the strength of the beam exiting the
material.  A single run of the solver usually results in a single value, or realization, of the quantity of
interest.

This single realization might be misleading, however.  In most systems there is some degree of uncertainty in
the input parameters to the solver.  Some of these uncertainties may be epistemic, or systematic uncertainty
originating with inexact measurements or measurable unknowns.  Other uncertainties might be aleatoric,
intrinsic uncertainty in the system itself, such as probabilistic interactions or random motion.  Taken
together, the input parameter uncertainties exist within a multidimensional probabilistic space.  While some
points in that space may be more likely than others, the possible range of values for the QoI is only
understood when the uncertain input space is considered as a whole.  We note here that while it is possible
that some of the input parameters are correlated in their probabilistic distribution, it is also possible to
decouple them into uncorrelated variables.  Throughout this work we will assume the input parameters 
are uncorrelated.

%Monte Carlo
One traditional method for exploring the uncertain input space is through random sampling, such as in analog Monte
Carlo sampling.  In this method, a point in the input space is chosen at random based on probability.  This
point represents values for the input parameters to the solver.  The solver is executed with these inputs, and
the QoIs are collected.  Then, another point in the input space is chosen at random.  This process continues
until the properties of the QoIs, or \emph{response}, are well understood.

There are some beneficial properties to random sampling approaches like Monte Carlo.  
Significantly, they are unintrusive:
 there is no need to modify the solver in order to use these methods.  This allows a framework of
algorithms to be developed which know only the input space and QoI of a solver, but need no further knowledge
about its operation.  Unintrusive methods are desirable because the uncertainty quantification algorithms can
be developed and maintained separately from the solver.

Monte Carlo and similar sampling strategies are relatively slow to converge on the response surface.  For
example, with Monte Carlo sampling, in order to reduce the standard error of the mean of the response by a factor
of two, it is necessary to take at least four times as many samples.  If a solver is sufficiently computationally
inexpensive, running additional solutions is not a large concern; however, for lengthy and expensive solvers,
it may not be practical to obtain sufficient realizations to obtain a clear response.

% expensive solvers need low-sample UQ
In this work, we will assume solvers are computationally expensive, requiring many hours per solve, and that
computational resource availability requires as few solves as possible.  As such, we consider several methodologies 
for quantifying the uncertainty in expensive solver
calculations.  In order to demonstrate clearly the function of these methods, we apply them first on
several simpler problems, such as polynomial evaluations and analytic attenuation.  These models have a high
degree of regularity, and their analyticity provides for straightforward benchmarking.  Through gradual
increasing complexity, we investigate the behavior of the UQ methods.

Finally, we apply the methods to an engineering-scale solver that
models the neutronics and performance of nuclear fuel.  This will
demonstrate the practical application of the uncertainty quantification methods, where the regularity and
other properties of the model are not well understood.

The first uncertainty quantification method we consider
is traditional analog Monte Carlo (MC) analysis, wherein random sampling of the input space generates a view of
the response.  MC is used as a benchmark methodology; if other methods converge on moments of the quantities
of interest more quickly and consistently than MC, we consider them ``better'' for our purposes.

The second method we consider is stochastic collocation for generalized polynomial
chaos (SCgPC)\cite{sparseSC,sparse1,sparse2,xiu}, whereby deterministic collocation points 
are used to develop a polynomial-interpolated reduced-order model
of the response as a function of the inputs.  This method algorithmically expands the solver as the sum of
orthogonal multidimensional polynomials with scalar coefficients.  The scalar coefficients are obtained by
numerical integration using multidimensional collocation (quadrature) points.  The chief distinction between
SCgPC and Monte Carlo methods is that SCgPC is deterministic, in that the realizations required from the
solver are predetermined instead of randomly sampled.  There are two major classes of deterministic
uncertainty quantification methods: intrusive and unintrusive.  Like Monte Carlo, SCgPC is unintrusive
and performs well without any need to access the operation of the solver.  This behavior is desirable for
construction black-box approach algorithms for uncertainty quantification.  Other intrusive methods such as
stochastic Galerkin exist \cite{galerkin}, but require solver modification to operate.  This makes them
solver-dependent and undesirable for an independent uncertainty quantification framework.

The other methods we present here expand on
SCgPC.  First, we introduce non-tensor-product methods for determining the set of polynomial bases to
use in the expansion.  Because a tensor product grows exponentially with increasing cardinality of the input
space, we combat this curse of dimensionality using the 
alternative polynomial set construction methods\cite{hctd}.
These bases will then be used to construct Smolyak-like sparse grids \cite{smolyak} to provide collocation
points that in turn calculate the coefficients in the polynomial expansion.  Second, we consider
anisotropic sparse grids,
allowing higher-order polynomials for particular input parameters.  We also consider methods for
obtaining weights that determine the level of anisotropic preference to give parameters, and explore the effects of a
variety of anisotropic choices.

The second method group we consider is high-dimension model representation (HDMR), which correlates with Sobol
decomposition \cite{hdmr}.  This method is useful both for developing sensitivities of the quantity of interest to subsets
of the input space, as well as constructing a reduced-order model itself.  We demonstrate the strength of HDMR
as a method to inform anisotropic sensitivity weights for SCgPC.

Finally, we consider adaptive algorithms to construct both SCgPC and HDMR expansions using second-moment
convergence criteria.  We analyze these for potential efficiencies and shortcomings.  We also propose future
work to further improve the adaptive methods.

We implement all these methods in Idaho National Laboratory's \raven{}\cite{raven}
uncertainty quantification framework. \raven{} is a Python-written framework that non-intrusively provides
tools for analysts to quantify the uncertainty in their simulations with minimal development.  To demonstrate
the application of the method developed, we use a complex non-linear multiphysics system solver simulating
the operation of a fuel pin within a nuclear reactor core, including both neutronics and fuel performance
physics kernals.  For this solver, we use the coupled \rattlesnake{}\cite{rattlesnake} and 
\bison{} \cite{bison,mammoth} production codes.
Both of these codes are developed and maintained within the \moose{}\cite{moose} environment.  The
multiphysics nonlinear system provides a challenge with unknown response properties for the uncertainty
quantification methods discussed in this proposal.

%outline chapters
The remainder of this work will proceed as follows:
\begin{itemize}
  \item Chapter 2: We describe the analytic test problems and engineering-scale problem solved by the simulations 
    we will be running, along with their properties and inferences about the algorithms developed.
    We discuss potential approaches to model solving and applications of the models.
  \item Chapter 3: We describe methods for uncertainty quantification, including Monte Carlo (MC),
    stochastic collocation for generalized Polynomial Chaos (SCgPC), and high-dimension model reduction (HDMR).
  \item Chapter 4: We analyze results obtained for the static UQ methods on analytic models, and contrast 
    them with traditional Monte Carlo convergence on statistical moments.
  \item Chapter 5: We describe implementation of adaptive methods for both SCgPC and HDMR.  We discuss the
    strengths, weaknesses, and options for these adaptive methods.
  \item Chapter 6: We evaluate results for the adaptive methods in comparison with static methods and Monte
    Carlo for the analytic models.  We again consider the strengths and weaknesses of the adaptive methods in 
    light of the results. 
  \item Chapter 7: We perform analysis on the engineering-scale multiphysics coupled problem, and analyze
    results.
  \item Chapter 8: We draw conclusions from the evaluations performed, and offer some suggestions for
    applicability as well as future development for the UQ methods demonstrated.
\end{itemize}
%----------------------------------------------------------------------------------------

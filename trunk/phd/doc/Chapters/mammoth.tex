% Chapter Template

\chapter{Engineering Demonstration} % Main chapter title

\label{Chapter5} % Change X to a consecutive number; for referencing this chapter elsewhere, use \ref{ChapterX}

\lhead{Chapter 5. \emph{Engineering Demonstration}} % Change X to a consecutive number; this is for the header on each page - perhaps a shortened title

%----------------------------------------------------------------------------------------
%	SECTION: INTRO
%----------------------------------------------------------------------------------------

\section{Introduction}
While analytic models provide insight to the operation of stochastic collocation for generalized polynomial chaos and high-density
model reduction methods, we are chiefly interested in applying these methods to engineering applications that lead to
decision making in real-world activities.  To this end, we selected multiphysics simulation code \mammoth{}, a \moose{}-based
\emph{MultiApp} that couples neutronics code \rattlesnake{}, fuel performance code \bison{}, and thermal hydraulics code
\texttt{relap-7}.  \mammoth{} solves these three sets of physics nonlinearly using picard iterations to feed back values until
convergence is achieved.

\section{Problem}
The model we selected is a two-dimensional slice of a pressurized water reactor fuel pin, including the fuel, gap, clad, and
moderator.  The fuel is separated into 23 axial rings, each with the same material properties but independent input uncertainty.
The response value of interest is the neutronics multiplication factor $k$-effective.  TODO how many groups, etc.
The uncertain inputs are group-, burnup-, and temperature-dependent cross sections, as well as TODO three \bison{} inputs.
The macroscopic cross sections are calculated using \texttt{scale} (TODO cite) along with a covariance matrix which provides
the correlation between cross sections.  Karhunen-Loevre (TODO cite) is used to de-couple the resulting 671 inputs and
reduce them to 10 representative independent inputs.

The simulation only considers neutronics (\rattlesnake{}) and fuel performance (\bison{}), so the thermal hydraulics is neglected.
The feedback from \rattlesnake{} to \bison{} is the power shape, and the feedback from \bison{} to \rattlesnake{} is the
temperature, which in turn affects the cross sections.  For performance, the number of picard iterations between the separate apps
is limited to 6 per time step.  TODO time steps, other input parameters, input file in appendix?, commit of MAMMOTH used

\section{Limitations}
During the collection of data, it was discovered that the performance of \bison{} can fluctuate depending on
the way is is parallelized.  There were instances where \bison{} would fail to converge, but report an unconverged
temperature as a converged solution.  As a result, there is artifical numerical error that is difficult to track or
account for.  Regardless, we demonstrate the performance of various methods on this model, as this behavior indicates
true simulation behavior.

\section{Results}

todo.

%Notes:
%
%Run times
%
%For Picard 6 and SN (3 azimuthal, 3 polar Gauss Chebyshev):
%
%MPI 24: 11m 29.452s = 689.452 sec =  16546.848 single equivalent (0.424 efficient)
%MPI 12: 19m 41.703s = 1181.703 sec = 14180.436 single equivalent (0.495 efficient)
%MPI  6: 27m 50.373s = 1670.373 sec = 10022.238 single equivalent (0.701 efficient)
%MPI  1: 117m 0.756s = 7020.756 sec =  7020.756 single equivalent (1.000 efficient)

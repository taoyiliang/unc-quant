% Chapter Template

\chapter{Time-Dependent Analysis} % Main chapter title

\label{ChapterTime} % Change X to a consecutive number; for referencing this chapter elsewhere, use \ref{ChapterX}

\lhead{Chapter X. \emph{Time-Dependent Analysis}} % Change X to a consecutive number; this is for the header on each page - perhaps a shortened title

%----------------------------------------------------------------------------------------
%	SECTION 1
%----------------------------------------------------------------------------------------

\section{Introduction}

Up to now in this work we have restricted ourselves to models with a set of single-valued responses.  One of the strengths
of the \raven{} framework is its innate ability to extend reduced-order models (such as the stochastic 
collocation for generalized polynomial chaos and high-density model reduction expansions) to
include time-dependent analysis.  \raven{} does this by taking snapshots in time and interpolating between them to evaluate
the reduced-order model at any time.  As part of this work we added the algorithms necessary to do this snapshot-based
time-dependent analysis for both the stochastic collocation for generalized polynomial chaos and the high-density model
reduction methods. Conveniently, no additional quadrature points are required to perform transient instead of static
uncertainty quantification using SCgPC or HDMR methods.  It should be noted, however, that adaptive SCgPC and adaptive
HDMR are not well-suited to time-dependent analysis because of the plethora of responses; effectively, there is a
full set of responses for each snapshot in time, making it difficult for the adaptive algorithms to determine the
ideal polynomials to add.

To demonstrate performance of this feature, we need a time-dependent response with transient behavior.
Because none of the analytical models nor the \mammoth{} pincell problem have notable transient behavior, we consider
a \bison{} simulation of an OECD benchmak (TODO citations) where the performance of light-water reactor fuel 
through several power transients is
analyzed.

\section{Problem Description}
TODO problem description

The responses of interest for this model are the axial maximum fuel centerline temperature, the maximum cladding temperature,
fission gas released, elongation of the cladding, and stress on the cladding.

Of particular interest in this problem is analysis of sensitivity coefficients as they devolop in time.  As the
simulation progresses, there is a shift in the dominant physics behind response values.  For example, one of the more
dramatic physics transitions occurs as the fuel expands enough to make contact with the cladding.

\section{Results}
TODO results

\section{Conclusion}
Time-dependent sensitivity analysis provides means to better understand uncertainty propogation throughout a transient simulation.
As physics shift throughout the simulation, so too does the sensitivity of the response to the input parameters.  If this same
simulation were performed using only static analysis on time-averaged responses, the ability to make clear decisions would be
reduced because of the lack of time-dependent information.  The addition of time-dependent analysis is quite beneficial to 
analysts considering time-dependent simulations.


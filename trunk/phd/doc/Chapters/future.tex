% Chapter Template

\chapter{Future Work} % Main chapter title

\label{Chapter9} % Change X to a consecutive number; for referencing this chapter elsewhere, use \ref{ChapterX}

\lhead{Chapter 9. \emph{Future Work}} % Change X to a consecutive number; this is for the header on each page - perhaps a shortened title

%----------------------------------------------------------------------------------------
%	SECTION: INTRO
%----------------------------------------------------------------------------------------

\section{Introduction}
The results in this work lead to several interesting areas of improvement.  We discuss some of these briefly
here.

\section{Impact Inertia in Adaptive Samplers}
One weakness demonstrated in the adaptive sampling techniques is the phenomenon of purely-even or purely-odd
polynomial representations.  This is seen clearly in the Ishigami (\ref{mod:ishigami}) and Gauss Peak
(\ref{mod:gausspeak}) models.  Consider the Taylor development of a sine function,
\begin{equation}\label{eq:sine}
  \sin x = x - \frac{x^3}{6} + \frac{x^5}{120} + \mathcal{O}(x^7),
\end{equation}
and for a square exponential,
\begin{equation}\label{eq:sine}
  e^{-x^2} = 1 - x^2 + \frac{x^4}{2} - \frac{x^6}{6} + \frac{x^8}{24} + \mathcal{O}(x^10).
\end{equation}
Unique to both of these functions is ``skipping'' certain polynomial orders (evens for sine, odds for square
exponential).

In a single-dimension example, the current impact estimation expression is
\begin{equation}
  \tilde{\eta_k}= \eta_{k-1}.
\end{equation}

Because adaptive sampling currently relies on the previous-order polynomial to estimate the
importance of the current polynomial, it can be misled into thinking there is no additional information to
gather if certain polynomials are not present in the expansion.

For example, if the adaptive sampler finds the impacts of a one-dimensional problem to be 0.4 for $x$, it will
try $x^2$.  If the model is an odd function, it will find the impact for $x^2$ is actually zero.  As a result,
it will be very unlikely to try $x^3$, despite the fact that $x^3$ has significant real impact.

One resolution to this method is to apply some sort of \emph{impact inertia} to the estimation of impact
values; that is, in addition to considering the previous polynomial impact when estimating current polynomial
impact, several previous polynomials might be considered.  This kind of inertia is likely problem-dependent in
its effectiveness, and would be best controlled through an optional user input.  Some research would be
required to determine what default level of inertia is recommended.  The new impact estimation expression
would be something like the following:

\begin{equation}
  \tilde{\eta_k}= \frac{1}{\alpha}\sum_{n=1}^k \frac{1}{g(n)}\eta_{k-n},
\end{equation}
where $\alpha$ is a balancing parameter and $g(n)$ is a penalty function that grows as $k-n$ increases.

\section{Cross-Communication in Adaptive HDMR}
Todo, basically if I want the next step in (x,y), that next step shouldn't be a polynomial containing 0 in
either x or y.

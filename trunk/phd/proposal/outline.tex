\documentclass{article}
\usepackage{amsmath}
\usepackage{amssymb}
%\usepackage{amsthm}
\usepackage{amscd}
%\usepackage{amsfonts}
\usepackage{graphicx}%
\usepackage{fancyhdr}
\usepackage{color}
\usepackage{cite}
\usepackage{fullpage}
%\usepackage[T1]{fontenc}
%\usepackage[utf8]{inputenc}
%\usepackage{authblk}
\usepackage{physics}
\usepackage{float}
\usepackage{caption}
\usepackage{subcaption}

\setlength{\columnsep}{0.5 in}

\newcommand{\expv}[1]{\ensuremath{\mathbb{E}[ #1]}}
\newcommand{\xs}[2]{\ensuremath{\Sigma_{#1}^{(#2)}}}
\newcommand{\intO}{\ensuremath{\int\limits_{4\pi}}}
\newcommand{\intz}{\ensuremath{\int\limits_0^1}}
\newcommand{\intf}{\ensuremath{\int\limits_{-\infty}^\infty}}
\newcommand{\intzf}{\ensuremath{\int\limits_{0}^\infty}}
\newcommand{\LargerCdot}{\raisebox{-0.25ex}{\scalebox{1.2}{$\cdot$}}}

%\textwidth6.6in
%\textheight9in


%\setlength{\topmargin}{0.3in} \addtolength{\topmargin}{-\headheight}
%\addtolength{\topmargin}{-\headsep}

%\setlength{\oddsidemargin}{0in}

%\oddsidemargin  0.0in \evensidemargin 0.0in \parindent0em

%\pagestyle{fancy}\lhead{MATH 579 (UQ for PDEs)} \rhead{02/24/2014}
%\chead{Project Proposal} \lfoot{} \rfoot{\bf \thepage} \cfoot{}
\title{Adaptive Sparse-Grid Stochastic Collocation Uncertainty Quantification \\ \vspace{10pt}\normalsize A Ph.D. Research Proposal}

\author{Paul W. Talbot\\talbotp@unm.edu\\Department of Nuclear Engineering\\University of New Mexico}
%\date{}


\begin{document}
\maketitle
\section{Motivation}
\begin{itemize}
\item With the increase of numerical models, UQ is critical.
\item For low-dimension uncertainty space, MC is inefficient.
\item Generalized polynomial chaos (gPC) is a suitable surrogate approximation.
\item Stochastic collocation (SC) is very efficient compared to MC for small input spaces.
\item Sobol decomposition is a natural extension of gPC.
\end{itemize}

\section{Theory}
\subsection{gPC}
  \begin{itemize}
  \item Represent stochastic process as sum of product of weighted sets of orthonormal polynomials
  \item Index Sets determine combination of polynomial orders to use based on truncation level $L$
  \item Further, use anisotropy to increase polynomial focus where sensitivity is highest
  \end{itemize}
\subsection{SC}
  \begin{itemize}
  \item To determine coefficients for polynomial products, use quadrature integration
  \item Use Smoljak-like sparse grids to alleviate curse of dimensionality
  \end{itemize}
\subsection{HDMR/Sobol decomposition}
  \begin{itemize}
  \item First decompose QoI into reference, singlet, duplet, etc. terms
  \item Second, evaluate each sub-term using SC for gPC
  \item Doesn't improve on gPC generally for convergence, but provides sensitivities
  \item For large dimension, can use nested quadrature to reduce work necessary
  \end{itemize}

\section{Demonstrated Results}
\subsection{Static Sparse Grid}
\subsubsection{Analytical Function, $N=5$}
  \begin{itemize}
  \item TD best, HC next, MC last (5 inputs)
  \item Exponential convergence for TD
  \end{itemize}
\subsection{Projectile, $N=8$}
  \begin{itemize}
  \item HC best, TD next, MC last
  \item Demonstrate lower regularity
  \end{itemize}
\subsection{Reactor Core}
  \begin{itemize}
  \item $N=5$
    \begin{itemize}
    \item HC best, TD next, MC last
    \item Similar convergence between index sets
    \end{itemize}
  \item $N=14$
    \begin{itemize}
    \item On-par with Monte Carlo
    \end{itemize}
  \end{itemize}
\subsection{Anisotropic Sparse Grid}
  \begin{itemize}
  \item Correctly chosen, always improves convergence
  \item Incorrectly chosen, slower convergence
  \item Can unexpectedly increase total computations for the same polynomial level $L$ for large anisotropy (compared to isotropic sparse grid)
  \end{itemize}
\subsection{Sobol Decomposition (HDMR)}
  \begin{itemize}
  \item Sparse grid accomplishes nearly the same effect as HDMR
  \item However, HDMR still good for sensitivity analysis and creating anisotropic weights
  \end{itemize}
  
  \newpage
\section{Intended Research}
\subsection{Adaptivity}
  \begin{itemize}
  \item Polynomial selection - add polynomial choices in any dimension/combination of dimensions until contribution is less than a relative tolerance.
  \begin{itemize}
     \item Algorithm:
     \item Loop: start with minimal index set, e.g. $\{(0,0,0),(1,0,0),(0,1,0),(0,0,1)\}$, calculate SC for gPC variance
     \item In each dimension, and each combination of dimensions that are only 1 greater than each existing dimension order, add a point
     \item See the contribution from each point to the new variance, and close off any dimensions that are converged to tolerance
     \item Continue until all dimensions/combinations of dimensions are converged
     \item Using a database of nested points, few additional evaluations should be needed each iteration.
     \end{itemize}
  \item HDMR - Similarly, choose component Sobol factors to be included, then solve each one using adaptive SC for gPC.
     \begin{itemize}
     \item Algorithm:
     \item Outer loop: add terms in HDMR
     \item Inner loop: create surrogate model for each term using AASC for gPC
     \item If new term contributes less than tolerance, consider dimensions varied to be converged.
     \item Stop adding terms in a dimension if all the lower-order terms for the dimension are converged.
     \item Continue until all dimensions and combinations are converged.
     \end{itemize}
  \end{itemize}
  
\subsection{Surrogate Modeling}
  \begin{itemize}
  \item Feature of interest: limit search algorithm - This particular existing \texttt{RAVEN} tool adaptively chooses points to solve in order to define a limiting surface, e.g. between ``success'' and ``failure'' regions.  Failure probability is integral of failure region.
  \item Computationally expensive to directly solve algorithm each sample, so use gPC as surrogate model instead
  \item Metric for Adaptive Anisotropic Sparse Grid Stochastic Collocation for Generalized Polynomial Chaos expansion: ability to determine limit surface and failure probability accurately.
  \item Suggested problem: irradiated fuel in cladding, power ramp up until fuel touches cladding, do limit search around limiting Von Mises stress.
  \end{itemize}
\end{document}
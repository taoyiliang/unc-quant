% Chapter Template

\chapter{Proposed Work} % Main chapter title

\label{Chapter4} % Change X to a consecutive number; for referencing this chapter elsewhere, use \ref{ChapterX}

\lhead{Chapter 4. \emph{Proposed Work}} % Change X to a consecutive number; this is for the header on each page - perhaps a shortened title

%----------------------------------------------------------------------------------------
%	SECTION 1
%----------------------------------------------------------------------------------------

\section{Proposal}
There are several opportunities to continue this preliminary work.  First, it is possible \cite{Gerstner} to
adaptively construct polynomial index sets instead of using deterministic index sets.  Similarly, HDMR can be
expanded with adaptive algorithms to build up the representation in consecutive subsets\cite{Ayres}.  These adaptive
algorithms have been shown to greatly affect the curse of dimensionality, to the point where HDMR based on
SCgPC is competitive with Monte Carlo sampling for input spaces with dimensionality on the order of hundreds.

In concert with this, while Gaussian quadrature is effective at integrating polynomials, it has been shown
\cite{goodclenshaw} that other quadratures with convenient properties can compete effectively with them.  In
particular, when performing adaptive sampling, nested quadratures offer a significant reduction in necessary
computational cost for convergence, since quadrature points are re-used in successive layers.  We propose
implementing and considering some nested quadrature strategies such as Clenshaw-Curtis and Gauss-Patterson and
analyzing their impact on convergence when used in SCgPC.

Similarly, it would be of interest to contrast the use of Lagrange polynomials in the gPC expansion with
versus the Wiener-Askey polynomials used thus far.  We expect to see little change in convergence rates except
in cases where the simulation model is easily-replicated by either set of polynomial families.

Lastly, we propose implementing all the methods described in this work in the \raven{} uncertainty
quantification framework, and perform thorough analysis on a selected \bison{} case.  Similar analysis has
been performed recently \cite{bigbison} but without the benefit of some of the adaptive and combined
strategies established and proposed in this work.  Such implementation would demonstrate significant viability
for the methods and make them more readily available to uncertainty quantification analysts.

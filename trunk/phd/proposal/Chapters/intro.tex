% Chapter 1

\chapter{Introduction} % Main chapter title

\label{ch:intro} % For referencing the chapter elsewhere, use \ref{Chapter1} 

\lhead{1. \emph{Introduction}} % This is for the header on each page - perhaps a shortened title

%----------------------------------------------------------------------------------------

%\section{Welcome and Thank You}

%problem description
Fuels performance codes are numerical simulations intended to characterize the performance of a set of
materials in a particular geometry under a certain environment, over time.  The environmental considerations
might include temperature, neutron flux, external pressure, and similar factors.  In many cases, the
performance is quantified by considering the maximum stress undergone by cladding around the fuel as it
expands and makes contact.  By varying the construction materials and geometry of the fuel, its cladding, and
the gap between them, fuel can be designed for optimal performance without experiencing a rupture or similar
break.

%introduce uq for problem
There are a plethora of parameters that go into simulating fuel performance.  The fuel itself is made up of
many constituent materials with a variety of densities and structures, as well as behavior under irradiation.
The contents of the fuel-cladding gap determine how effectively heat can conduct out of the fuel and to the
cladding, then out to a moderator, and the thickness of this gap determines the amount of fuel expansion
allowed before contact is made and outward pressure begins increasing.  The material and geometry of the
cladding determine limits on stress and efficiency of heat transfer.  Any of the material properties in the
fuel, gap, or cladding, along with the environmental conditions, can be a source of uncertainty in determining
the maximum stress applied to the cladding.

%explain nature of uncertainty
There are two categories into which sources of uncertainty fall: aleatoric, or the statistical uncertainty inherent in a
system; and epistemic, or the systematic uncertainty due to imprecision in measurement or existence of
measurable unknowns.  While there are aleatoric uncertainties in fuel performance (such as the neutronics of
irradiated fuel), in this work we consider mostly epistemic uncertainties surrounding the material properties
and geometries of the problems.  For an example case, we can consider the overall reactor power, fuel mesoscale
grain growth, and fuel thermal expansion coefficient as uncertain input parameters, with maximum Von Mises stress in the 
axial center of a fuel rod as a quantity of interest in the output space.

%explain scope of proposal
In this work, we consider several methodologies for quantifying the uncertainty in fuel performance
calculations.  In order to demonstrate clearly the function of these methods, we demonstrate them first on
several simpler problems, such as polynomial evaluations or projectile motion.  The first method we consider
is traditional, analog Monte Carlo (MC) analysis, wherein random sampling of the input space generates a view of
the output space.  MC is used as a benchmark methodology; if other methods converge on quantities of interest
more quickly and accurately than MC, we consider them ``better'' for our purposes.

The second method we consider is isotropic, tensor-product (TP) stochastic collocation for generalized polynomial
chaos (SCgPC)\cite{sparseSC}\cite{sparse1}\cite{sparse2}\cite{xiu}, whereby deterministic collocation points are used to develop a polynomial reduced-order model
of the output quantities of interest as a function of the inputs.  The other methods we consider expand on
this method.  First, we introduce non-tensor-product methods for determining polynomial bases, using the 
total degree (TD) and hyperbolic cross (HC) polynomial set construction methods\cite{hctd}.
These bases will then be used to construct Smolyak-like sparse grids for collocation\cite{smolyak}.  Second, we consider
anisotropic sparse grids,
allowing additional collocation points for preferential input parameters.  We also consider methods for
determining weights that determine the level of preference to give parameters, and explore the effects of a
variety of anisotropic choices.

The third method we consider is high-dimension model representation (HDMR), which correlates with Sobol
decomposition \cite{hdmr}.  This method is useful both for developing sensitivities of the quantity of interest to subsets
of the input space, as well as constructing a reduced-order model itself.  We demonstrate the strength of HDMR
as a method to inform anisotropic sensitivity weights for SCgPC.

Additionally, we propose continued work on developing adaptive algorithms for both SCgPC and HDMR\cite{Ayres}.  In adaptive
SCgPC, the polynomial basis is constructed level-by-level based on the highest-impact subset polynomials.  In
adaptive HDMR, the constituent subset input spaces are developed similarly, based on the highest-impact input
subset.  The crowning achievement we propose is combining HDMR and SCgPC to develop both the subset input
space as well as the overall reduced-order model adaptively in an attempt to construct a
competitively-efficient method for uncertainty quantification.

Finally, we propose all these methods be developed within Idaho National Laboratory's \raven{}\cite{raven}
uncertainty quantification framework. \raven{} is a Python-written framework that non-intrusively provides
tools for analysts to quantify the uncertainty in their simulations with minimal impact. \raven{}  has already
been shown to work seamlessly with \moose{}-based fuel performance code \bison{}\cite{moose}\cite{bison}, on which we propose to demonstrate the various
methods described above.

%outline chapters
The remainder of this work will proceed as follows:
\begin{itemize}
  \item Chapter 2: We mathematically describe the problems solved by the simulations we will be running,
    including polynomial evaluations, attenuation, projectile, diffusion, and fuel performance.  We discuss
    their potential applications and approach using random sampling.
  \item Chapter 3: We describe several methods for uncertainty quantification, including Monte Carlo, Latin
    Hypercube sampling, generalized Polynomial Chaos, and high-dimension model representation.  We also
    discuss methods to accelerate the convergence of SCgPC and HDMR models and.
  \item Chapter 4: We discuss proposed work extending both SCgPC and HDMR to be constructed adaptively.  We
    also discuss the predicted shortfalls in the adaptive algorithms and some potential methods to address
    them.
\end{itemize}
%----------------------------------------------------------------------------------------

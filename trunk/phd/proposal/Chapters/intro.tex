% Chapter 1

\chapter{Introduction} % Main chapter title

\label{ch:intro} % For referencing the chapter elsewhere, use \ref{Chapter1} 

\lhead{1. \emph{Introduction}} % This is for the header on each page - perhaps a shortened title

%----------------------------------------------------------------------------------------

%\section{Welcome and Thank You}

%problem description
In simulation modeling, we attempt to capture the behavior of a physical system by describing it in a series
of equations, often partial differential equations.  These equations may be time-dependent, and capture
physics of interest for understanding the system.  A \emph{solver} is then written that can solve the series
of equations and determine quantities of interest (QoI).  Typically, a solver accepts a set of inputs and
produces a set of outputs, often single-valued.  For instance, a solver might solve equations related to the
attenuation of a beam of photons through a material, and the QoI might be the strength of the beam exiting the
material.  A single run of the solver usually results in a single value, or realization, of the quantity of
interest.

This single realization might be misleading, however.  In most systems there is some degree of uncertainty in
the input parameters to the solver.  Some of these uncertainties may be epistemic, or systematic uncertainty
originating with inexact measurements or measurable unknowns.  Other uncertainties might be aleatoric,
intrinsic uncertainty in the system itself, such as probabilistic interactions or random motion.  Taken
together, the input parameter uncertainties exist within a multidimensional probabilistic space.  While some
points in that space may be more likely than others, the possible range of values for the QoI is only
understood when the uncertain input space is considered as a whole.  We note here that while it is possible
that some of the input parameters are correlated in their probabilistic distribution, it is also possible to
decouple them into uncorrelated variables.  Throughout this work we will assume the input parameters have been
uncorrelated from one another.

One traditional method for exploring the uncertain input space is through random sampling, such as in analog Monte
Carlo sampling.  In this method, a point in the input space is chosen at random based on probability.  This
point represents values for the input parameters to the solver.  The solver is executed with these inputs, and
the QoI is collected.  Then, another point in the input space is chosen at random.  This process continues
until the properties of the QoI or \emph{response} is well understood.

While Monte Carlo sampling is effective at developing an understanding of the response, the sheer volume of
samples needed to improve that understanding is sometimes prohibitive.  One way to reduce the number of
samples is by a Latin Hypercube Sampling (LHS) approach.  In this method, the input domain is divided into
orthogonal hypervolumes.  For instance, for a problem with two uncertain input parameters, the corresponding
input space is two-dimensional, and the LHS grid would appear as a Cartesian grid of the input space.  Once
the grid is constructed, a random sample is taken.  Once a sample is taken in one of the grid locations, no
additional samples can be taken that are within the same uncertainty range in any dimension.  In the two
dimensional example, once a point is sampled, no additional points can be sampled in the same row or column as
the sampled points.  In this way, the samples are distributed more thoroughly throughout the input space, and
may provide a more clear response with less realizations than analog Monte Carlo.

There are some beneficial properties to both Monte Carlo and LHS methods.  Significantly, they are unintrusive; that
is, there is no need to modify the solver in order to use these methods.  This allows a framework of
algorithms to be developed which know only the input space and QoI of a solver, but need no further knowledge
about its operation.  Unintrusive methods are desirable because the uncertainty quantification algorithms can
be developed and maintained separately from the solver.

However, both Monte Carlo and LHS sampling are relatively slow to converge on the response surface.  For
example, with Monte Carlo sampling, in order to reduce the error in the mean value of the response by a factor
of two, it is necessary to take four times as many samples.  If a solver is sufficiently computationally
inexpensive, running additional solutions is not a large concern; however, for lengthy and expensive solvers,
it may not be practical to run sufficient realizations to obtain a clear response.

%explain scope of proposal
In this work, we consider several methodologies for quantifying the uncertainty in expensive solver
calculations.  In order to demonstrate clearly the function of these methods, we apply them first on
several simpler problems, such as polynomial evaluations and analytic attenuation.  These models have a high
degree of regularity, and their analyticity provides for straightforward benchmarking.  We then apply the
methods to an intermediate-level solver for range of a projectile including drag.  This model is nonlinear and
has finite regularity, and has no analytic solution.  These properties challenge the methods while still
maintaining a fast and simple solver.  Finally, we apply the methods to an engineering-scale solver that
models the neutronics of a nuclear reactor core.  While not as complex as a multiphysics model, this will
demonstrate the ``real life'' application of the uncertainty quantification methods, where the regularity and
other properties of the model are not well understood.

The first uncertainty quantification method we consider
is traditional analog Monte Carlo (MC) analysis, wherein random sampling of the input space generates a view of
the response.  MC is used as a benchmark methodology; if other methods converge on moments of the quantities
of interest more quickly and consistently than MC, we consider them ``better'' for our purposes.

The second method we consider is isotropic, tensor-product stochastic collocation for generalized polynomial
chaos (SCgPC)\cite{sparseSC}\cite{sparse1}\cite{sparse2}\cite{xiu}, whereby deterministic collocation points 
are used to develop a polynomial-interpolated reduced-order model
of the response as a function of the inputs.  This method algorithmically expands the solver as the sum of
orthogonal multidimensional polynomials with scalar coefficients.  The scalar coefficients are obtained by
numerical integration using multidimensional collocation (quadrature) points.  The chief distinction between
SCgPC and Monte Carlo methods is that SCgPC is deterministic, in that the realizations required from the
solver are predetermined instead of randomly sampled.  There are two major classes of deterministic
uncertainty quantification methods: intrusive and unintrusive.  Like Monte Carlo and LHS, SCgPC is unintrusive
and performs well without any need to access the operation of the solver.  This behavior is desirable for
construction black-box approach algorithms for uncertainty quantification.  Other intrusive methods such as
stochastic Galerkin exist \cite{galerkin}, but require solver modification to operate.  This makes them
solver-dependent and undesirable for an independent uncertainty quantification framework.

The other methods we present here expand on
SCgPC.  First, we introduce non-tensor-product methods for determining the set of polynomial bases to
use in the expansion.  Because a tensor product grows exponentially with increasing cardinality of the input
space, we combat this curse of dimensionality using the 
total degree (TD) and hyperbolic cross (HC) polynomial set construction methods\cite{hctd}.
These bases will then be used to construct Smolyak-like sparse grids \cite{smolyak} to provide collocation
points that in turn calculate the coefficients in the polynomial expansion.  Second, we consider
anisotropic sparse grids,
allowing higher-order polynomials for particular input parameters.  We also consider methods for
obtaining weights that determine the level of anisotropic preference to give parameters, and explore the effects of a
variety of anisotropic choices.

%The third method we consider is high-dimension model representation (HDMR), which correlates with Sobol
%decomposition \cite{hdmr}.  This method is useful both for developing sensitivities of the quantity of interest to subsets
%of the input space, as well as constructing a reduced-order model itself.  We demonstrate the strength of HDMR
%as a method to inform anisotropic sensitivity weights for SCgPC.

Based on the preliminary work demonstrated here, we propose several improvements.  First, we propose
implementing high-dimension model representation (HDMR) as a method to decompose the input uncertainty space
into low-dimensional subspaces \cite{hdmr}.  This method, sometimes referred to as Sobol decomposition,
provides two significant functions.  First, this method provides the \emph{Sobol sensitivities}, which are
parameters based on analysis of variance (ANOVA).  Sobol sensitivities indicate the sensitivity of the
response QoI to changes in each subset space.  The first-order sensitivities can be used to determine suitable
anisotropic weighting for SCgPC.  Additionally, HDMR acts as a reduced-order model for the solver, potentially
providing more efficient evaluations.

Additionally, we propose continued work on developing adaptive algorithms for both SCgPC and HDMR\cite{Ayres}.  In adaptive
SCgPC, the polynomial basis is constructed level-by-level based on the highest-impact subset polynomials.  In
adaptive HDMR, the constituent subset input spaces are developed similarly, based on the highest-impact input
subset.  The crowning achievement we propose is combining HDMR and SCgPC to develop both the subset input
space as well as the overall reduced-order model adaptively in an attempt to construct a
competitively-efficient method for uncertainty quantification.

Finally, we propose all these methods be developed within Idaho National Laboratory's \raven{}\cite{raven}
uncertainty quantification framework. \raven{} is a Python-written framework that non-intrusively provides
tools for analysts to quantify the uncertainty in their simulations with minimal development.  To demonstrate
the application of the method developed, we propose a complex non-linear multiphysics system solver simulating
the operation of a fuel pin within a nuclear reactor core, including both neutronics and fuel performance
physics kernals.  For this solver, we propose to use the coupled \rattlesnake{}\cite{rattlesnake} and 
\bison{}\cite{bison}\cite{mammoth} production codes.
Both of these codes are developed and maintained within the \moose{}\cite{moose} environment.  The
multiphysics nonlinear system will provide a challenge with unknown response properties for the uncertainty
quantification methods discussed in this proposal.

%outline chapters
The remainder of this work will proceed as follows:
\begin{itemize}
  \item Chapter 2: We mathematically describe the problems solved by the simulations we will be running,
    including polynomial evaluations, attenuation, projectile, neutronics, and rector performance.  We discuss
    their potential approach and applications.
  \item Chapter 3: We describe several methods for uncertainty quantification, including Monte Carlo, Latin
    Hypercube sampling, and generalized Polynomial Chaos.  We also
    discuss methods to accelerate the convergence of SCgPC.
  \item Chapter 4: We analyze results obtained thus far for SCgPC methods, and contrast them with traditional
    Monte Carlo convergence on statistical moments.
  \item Chapter 5: We discuss proposed work both with HDMR and extending SCgPC to be constructed adaptively.  We
    also discuss the predicted shortfalls in the adaptive algorithms and some potential methods to address
    them.
\end{itemize}
%----------------------------------------------------------------------------------------

%%%%%%%%%%%%%%%%%%%%  OLD  %%%%%%%%%%%%%%%%%%%%%%
%
%Fuels performance codes are numerical simulations intended to characterize the performance of a set of
%materials in a particular geometry under a certain environment, over time.  Environmental considerations
%for nuclear fuels might include temperature, neutron flux, and external pressure.  In many cases, the
%performance is quantified by considering the maximum stress undergone by cladding around the fuel as it
%expands and makes contact.  By varying the construction materials and geometry of the fuel, its cladding, and
%the gap between them, fuel can be designed for optimal performance without experiencing a rupture or similar
%break.
%
%%introduce uq for problem
%There are a plethora of parameters that go into simulating fuel performance.  The fuel itself is made up of
%many constituent materials with a variety of densities and structures, as well as behavior under irradiation.
%The contents of the fuel-cladding gap determine how effectively heat can conduct out of the fuel and to the
%cladding, then out to a moderator, and the thickness of this gap determines the amount of fuel expansion
%allowed before contact is made and outward pressure begins increasing.  The material and geometry of the
%cladding determine limits on stress and efficiency of heat transfer.  Any of the material properties in the
%fuel, gap, or cladding, along with the environmental conditions, can be a source of uncertainty in determining
%the maximum stress applied to the cladding.
%
%%explain nature of uncertainty
%There are two categories into which sources of uncertainty fall: aleatoric, or the statistical uncertainty inherent in a
%system; and epistemic, or the systematic uncertainty due to imprecision in measurement or existence of
%measurable unknowns.  While there are aleatoric uncertainties in fuel performance (such as the neutronics of
%irradiated fuel), in this work we consider mostly epistemic uncertainties surrounding the material properties
%and geometries of the problems.  For an example case, we can consider the overall reactor power, fuel mesoscale
%grain growth, and fuel thermal expansion coefficient as uncertain input parameters, with maximum Von Mises stress in the 
%axial center of a fuel rod as a quantity of interest in the output space.
%

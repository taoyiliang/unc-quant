% Chapter Template

\chapter{Stochastic Collocation and generalized Polynomial Chaos} % Main chapter title

\label{Chapter3} % Change X to a consecutive number; for referencing this chapter elsewhere, use \ref{ChapterX}

\lhead{Chapter 3. \emph{SC and gPC}} % Change X to a consecutive number; this is for the header on each page - perhaps a shortened title

%----------------------------------------------------------------------------------------
%	SECTION 1
%----------------------------------------------------------------------------------------

\section{Uncertainty Quantification Methods}
We consider a few methods for uncertainty quantification (UQ).  One method to classify UQ methods is by their
interaction level with a simulation code.  Non-intrusive methods treat the code as a black box, perturbing the
inputs and collecting the outputs without modifying the code.  These methods are ideal for generic application
frameworks, where the simulation code may be unknown or precompiled.  Examples of non-intrusive methods
include analog Monte Carlo (MC), Latin Hypercube sampling (LHS), and deterministic collocation methods.
Alternatively, intrusive methods require access to the solution algorithm itself.  Sometimes this can provide
more efficient solutions.  In particular, adjoint methods require the solution operator to be reversible, and
provide very efficient methods to determine sensitivities and analyze output-input dependence.  While many
intrusive methods have benefits, they lack the flexibility and universal applicability of non-intrusive
methods, and so we neglect them in this work.

\section{Correlation}
An assumption we make going forward is that the uncertain input parameters are independent or at least
uncorrelated.  If this is not the case, methods such as Karhunen-Loeve expansion can be used to develop an orthogonal
input space to replace the original.  Similarly, principle component analysis can be used to attempt to find
the fundamental uncorrelated space that maps into the correlated variable space.

\section{Monte Carlo}
In analog Monte Carlo uncertainty quantification, a single point in the input space is selected randomly,
and an output collected, until an appropriately accurate view of the output space is acquired.  While few
samples result in a poor understanding of the quantity of interest, sufficiently large samples converge on a
correct solution.  The convergence rate of Monte Carlo is consistent, as
\begin{equation}
  \epsilon = \frac{c}{\sqrt{\eta}},
\end{equation}
where $c$ is a constant and $\eta$ is the number of samples taken.  While this convergence rate is slow, it is possibly one of the
most reliable methods available.  This makes MC a good choice for benchmarking.

One of the downsides of MC (and LHS) when compared with other methods is that they do not generate a
reduced-order model as part of the evaluation; however, interpolation methods can be used to generate
additional samples.

\section{Latin Hypercube Sampling}
Latin hypercube sampling\cite{lhs} is another stochastic method that specializes in multidimensional
distributions.  In this method
